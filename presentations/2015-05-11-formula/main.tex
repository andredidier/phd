%% LaTeX Beamer presentation template (requires beamer package)
%% see http://bitbucket.org/rivanvx/beamer/wiki/Home
%% idea contributed by H. Turgut Uyar
%% template based on a template by Till Tantau
%% this template is still evolving - it might differ in future releases!

\documentclass{beamer}

\mode<presentation>
{
\usetheme{CIn20131120}

\setbeamercovered{transparent}
}

\usepackage[english]{babel}
\usepackage[utf8]{inputenc}
\usepackage[T1]{fontenc} 

% font definitions, try \usepackage{ae} instead of the following
% three lines if you don't like this look
\usepackage{mathptmx}
\usepackage[scaled=.90]{helvet}
\usepackage{courier}

\DeclareGraphicsExtensions{.png,.eps}
%\DeclareGraphicsExtensions{.png}
\graphicspath{{./images/},{./}}


\usepackage[T1]{fontenc}


\title{Fault Tree Analysis with Temporal Faults Algebra}
\subtitle{A formal approach to verify safety requirements}

%\subtitle{}

% - Use the \inst{?} command only if the authors have different
%   affiliation.
%\author{F.~Author\inst{1} \and S.~Another\inst{2}}
\author{André Didier}

% - Use the \inst command only if there are several affiliations.
% - Keep it simple, no one is interested in your street address.


\date{11 May 2015}


% This is only inserted into the PDF information catalog. Can be left
% out.
\subject{Talks}



% If you have a file called "university-logo-filename.xxx", where xxx
% is a graphic format that can be processed by latex or pdflatex,
% resp., then you can add a logo as follows:

% \pgfdeclareimage[height=0.5cm]{university-logo}{university-logo-filename}
% \logo{\pgfuseimage{university-logo}}



% Delete this, if you do not want the table of contents to pop up at
% the beginning of each subsection:
% \AtBeginSubsection[]
% {
% \begin{frame}<beamer>
% \frametitle{Outline}
% \tableofcontents[currentsection,currentsubsection]
% \end{frame}
% }

% If you wish to uncover everything in a step-wise fashion, uncomment
% the following command:

%\beamerdefaultoverlayspecification{<+->}

\begin{document}

\begin{frame}
\titlepage
\end{frame}

\begin{frame}
\frametitle{Concepts in this presentation}

\begin{itemize}
  \item Systems modelling with Simulink
  \item Faults injection
  \item Faults traces
  \item Fault Trees -- Static, Temporal and Dynamic
  \item Free Boolean Algebra
  \item Lists lattice
  \item Temporal Faults Algebra
  \item Activation Algebra and Predicates
  \item Safety requirements and fault tree acceptance criteria
\end{itemize}
\end{frame}

\section{Master dissertation review}

\begin{frame}
\frametitle{System modelling with Simulink}
\includegraphics[width=\linewidth]{acsBlockDiagrams}
\end{frame}

\begin{frame}
\frametitle{Fault Tolerance, Failure Logic}
\includegraphics[height=0.4\textheight]{blockDiagramMonitorInternals}\par
\includegraphics[width=\linewidth]{acsAnnotations}
\end{frame}

\section{Fault Trees}

\begin{frame}
\frametitle{Static Fault Tree}
\includegraphics[height=0.7\textheight]{monitorFT.png}\par
\hfill{\tiny\color{100PANTONECOOLGRAY10CV}
\url{http://www.fault-tree-analysis-software.com}}\hspace{9mm}
\end{frame}

\subsection[Short First Subsection Name]{First Subsection Name}

\begin{frame}
\frametitle{}
\framesubtitle{Subtitles are optional}

\begin{itemize}
  \item
  \item
\end{itemize}
\end{frame}

\begin{frame}
\frametitle{}

% You can create overlays
\begin{itemize}
  \item using the \texttt{pause} command:
  \begin{itemize}
    \item First item.
    \pause
    \item Second item.
  \end{itemize}
  \item using overlay specifications:
  \begin{itemize}
    \item<3-> First item.
    \item<4-> Second item.
  \end{itemize}
  \item using the general \texttt{uncover} command:
  \begin{itemize}
    \uncover<5->{\item First item.}
    \uncover<6->{\item Second item.}
  \end{itemize}
\end{itemize}
\end{frame}

\section*{Summary}

\begin{frame}
\frametitle<presentation>{Summary}

\begin{itemize}
  \item The \alert{first main message} of your talk in one or two lines.
\end{itemize}

% The following outlook is optional.
\vskip0pt plus.5fill
\begin{itemize}
  \item Outlook
  \begin{itemize}
    \item Something you haven't solved.
    \item Something else you haven't solved.
  \end{itemize}
\end{itemize}
\end{frame}

\end{document}
