\DeclareMathOperator{\doo}{d}
\DeclareMathOperator{\nibefore}{\lhd}
\DeclareMathOperator{\simultaneous}{\triangle}
\DeclareMathOperator{\ibefore}{\unlhd}
\DeclareMathOperator{\inter}{\cap}
\DeclareMathOperator{\union}{\cup}
\DeclareMathOperator{\tvar}{\mathsf{tvar}}
\DeclareMathOperator{\distinct}{\mathsf{distinct}}
\DeclareMathOperator{\set}{\mathsf{set}}
\DeclareMathOperator{\xbefore}{\rightarrow}
\DeclareMathOperator{\listt}{\mathsf{list}}
\DeclareMathOperator{\sett}{\mathsf{set}}
\DeclareMathOperator{\tformulat}{\mathsf{tformula}}
\DeclareMathOperator{\BasicEventMinLevel}{\mathsf{BasicEventMinLevel}}
\DeclareMathOperator{\RootProbability}{\mathsf{RootProbability}}
\DeclareMathOperator{\evaluateRule}{\mathsf{evaluateRule}}
\DeclareMathOperator{\minBasicEventLevel}{\mathsf{minBasicEventLevel}}
\DeclareMathOperator{\ftProbability}{\mathsf{ftProbability}}
\DeclareMathOperator{\defs}{\doteq}
\DeclareMathOperator{\concat}{@}
%\newcommand{\aaexp}[2]{^{#1}/_{#2}}
\newcommand{\aaexp}[2]{{\left[#1\right]}^{\left[#2\right]}}
\newcommand{\nominalvalue}[2]{\mathsf{N}^{#2}\left(#1\right)}
\newcommand{\failurevalue}[2]{\mathsf{#1}^{#2}}
\newcommand{\component}[1]{\mathsf{C}^{#1}}
\newcommand{\outvalue}[2]{\mathsf{#1}^{#2}}
\newcommand{\outvalueof}[1]{\rho\left(#1\right)}
\def\fba{fba}
\def\tfa{tfa}
\newcommand{\tikznodemark}[1]{\tikz[na]\node [coordinate] (#1) {};}

\newcommand{\includeautosizegraphics}[1]{%
\includegraphics[height=0.8\textheight,width=\textwidth,keepaspectratio]{#1}%
}

\def\tikzoverlay{%
   \tikz[baseline,overlay]\node[every overlay node]
}

\tikzstyle{every overlay node}=[draw=black,fill=white,rounded corners,anchor=north west]

\tikzstyle{event text}=[text centered, 
  execute at begin node={\begin{varwidth}{2cm}},
  execute at end node={\end{varwidth}}
  ]
  
\tikzstyle{event}=[event text, rectangle, draw=black, fill=yellow!20, anchor=north]
%\tikzstyle{fault tree}=[edge from parent fork down, sibling distance=7cm, level distance=1.4cm, circuit logic US, growth parent anchor=south,nodes=event]
\tikzstyle{fault tree}=[circuit logic US]

\tikzstyle{level 1}=[sibling distance=7cm, level distance=1.4cm, growth parent anchor=south, nodes=event]
\tikzstyle{level 2}=[sibling distance=5cm]
\tikzstyle{level 3}=[sibling distance=5cm]
\tikzstyle{level 4}=[sibling distance=3cm]

\tikzstyle{gate}=[rotate=90, anchor=east, xshift=-1mm]
\tikzstyle{or gate}=[gate, or gate US,fill=blue!60]
\tikzstyle{and gate}=[gate, and gate US, fill=red!60]
\tikzstyle{basic}=[circle, fill=green!60, anchor=north, minimum width=0.5cm]

\tikzstyle{spare gate}=[shape=spare gate, fill=orange!40]
  
\tikzset{
  >=stealth',
  edge from parent path={(\tikzparentnode.south) -- ++(0,-0.95cm)
      -| (\tikzchildnode.north)},
  global scale/.style={
    scale=#1,
    every node/.style={scale=#1}
  }
}
\makeatletter
\pgfdeclareshape{spare gate}{
  \inheritsavedanchors[from=circle]
  \inheritanchor[from=circle]{center}
  \inheritanchor[from=circle]{north}
  \inheritanchor[from=circle]{south}
  \inheritanchor[from=circle]{east}
  \inheritanchor[from=circle]{west}
  \backgroundpath{%
      % Save radius to x
      \pgf@x=\radius
      % Radius is also containing the "minimum width" and "minimum height"
      % This ensures that even with no text the shape will be drawn.
      % Unless of course that min are set to 0pt
      % So no need to check for that
      % Save radius
      \pgfutil@tempdima=\pgf@x%

      % west triangle corner "b"
      \pgf@xb=-3\pgf@x%
      \pgf@yb=-4\pgf@x%
      % east triangle corner "c"
      \pgf@xc= 3\pgf@x%
      \pgf@yc=-4\pgf@x%

      % If text is present shift shape to center 
      % You need to shift more, but to get the idea
      \centerpoint
      \advance\pgf@xb by\pgf@x
      \advance\pgf@yb by\pgf@y
      \advance\pgf@xc by\pgf@x
      \advance\pgf@yc by\pgf@y

      % Save centerpoint in "a" (top triangle point)
      \pgf@xa=\pgf@x 
      \pgf@ya=\pgf@y

      % Below are good for debugging purposes.
      %\message{^^JTop : \the\pgf@xa,\the\pgf@ya}
      %\message{^^JWest: \the\pgf@xb,\the\pgf@yb}
      %\message{^^JEast: \the\pgf@xc,\the\pgf@yc}
      %\message{^^JCent: \the\pgf@x,\the\pgf@y}

      % draw triangle..
      \pgfpathmoveto{\pgfpoint{\pgf@xa}{\pgf@ya}}%
      \pgfpathlineto{\pgfpoint{\pgf@xb}{\pgf@yb}}%
      \pgfpathlineto{\pgfpoint{\pgf@xc}{\pgf@yc}}%
      \pgfpathclose

      % The radius of the small circles
      % Read in from option TODO
      \pgfutil@tempdimb=3pt

      % Move top triangle to head circle
      \advance\pgf@ya by.25\pgfutil@tempdimb
      % Move west triangle corner to west circle center
      \advance\pgf@xb by 1.5\pgfutil@tempdima
      \advance\pgf@yb by -\pgfutil@tempdimb
      % For handling line thickness if you wish "edge touch" and not "overlap"
      %\advance\pgf@yb by -.5\pgflinewidth 
      % Move east triangle corner to east circle center
      \advance\pgf@xc by-1.5\pgfutil@tempdima
      \advance\pgf@yc by -\pgfutil@tempdimb
      % For handling line thickness if you wish "edge touch" and not "overlap"
      %\advance\pgf@yc by -.5\pgflinewidth

      % This saves underlying "stuff" when you have the explicit `\pgfqpoint` and is thus a little faster
      \edef\pgf@marshal{%
          \noexpand\pgfpathcircle{%
              \noexpand\pgfqpoint{\the\pgf@xa}{\the\pgf@ya}}
          {\the\pgfutil@tempdimb}%
          \noexpand\pgfpathcircle{%
              \noexpand\pgfqpoint{\the\pgf@xb}{\the\pgf@yb}}
          {\the\pgfutil@tempdimb}%
          \noexpand\pgfpathcircle{%
              \noexpand\pgfqpoint{\the\pgf@xc}{\the\pgf@yc}}
          {\the\pgfutil@tempdimb}%
      }\pgf@marshal
  }
}
\makeatother

\def\CSPM{CSP$_M$\xspace}


\newenvironment{snippetcspm}[1][2]
{
\ifthenelse{\equal{#1}{0}}
    {\tiny}
    {
    \ifthenelse{\equal{#1}{1}}
        {\scriptsize}
        {
        \ifthenelse{\equal{#1}{2}}
            {\footnotesize}
            {\small}
        }
    }
%\begin{samepage}
\verbatim
}
{
\endverbatim
%\end{samepage}
}

