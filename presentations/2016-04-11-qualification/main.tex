\documentclass{beamer}

\mode<presentation>
{
\usetheme{CIn20131120}
\setbeamercovered{transparent}
}

%\AtBeginSection[]
%{
%	\begin{frame}<beamer>
%		\frametitle{Outline}
%		\tableofcontents[currentsection,hideothersubsections]
%	\end{frame}
%}

\usepackage[british]{babel}
\usepackage[utf8]{inputenc}
\usepackage[T1]{fontenc} 
\usepackage{xspace}
\usepackage{graphicx}
\usepackage{amsmath,amsfonts,amssymb,amsthm}

%\DeclareGraphicsExtensions{.pdf,.png,.jpg}
\DeclareGraphicsExtensions{.pdf,.eps,.png}
%\DeclareGraphicsExtensions{.png}
\graphicspath{{./../../thesis/},{./}}

\title{An Algebra of Temporal Faults}
\subtitle{PhD Qualification}
\date{April 2016}

\author{André Didier}

\institute{
\inst{1}Federal University of Pernambuco\\
Centre of Informatics
}

\subject{Qualification presentation}

%%%%%%%%%%%%%%%%%%% Acronyms
\def\FTA{Fault Tree Analysis (FTA)\gdef\FTA{FTA\xspace}\xspace}
\def\FT{%
	Fault Tree (FT)%
	\gdef\FT{FT\xspace}%
	\gdef\FTs{FTs\xspace}%
	\xspace%
}
\def\FTs{%
	Fault Trees (FTs)%
	\gdef\FT{FT\xspace}%
	\gdef\FTs{FTs\xspace}%
	\xspace%
}
\def\MCS{%
	Minimal Cut Set (MCS)%
	\gdef\MCS{MCS\xspace}%
	\gdef\MCSs{MCSs\xspace}%
	\xspace%
}
\def\MCSs{%
	Minimal Cut Sets (MCSs)%
	\gdef\MCS{MCS\xspace}%
	\gdef\MCSs{MCSs\xspace}%
	\xspace%
}
\def\SFT{%
	Static Fault Tree (SFT)%
	\gdef\SFT{SFT\xspace}%
	\gdef\SFTs{SFTs\xspace}%
	\xspace%
}
\def\SFTs{%
	Static Fault Trees (SFTs)%
	\gdef\SFT{SFT\xspace}%
	\gdef\SFTs{SFTs\xspace}%
	\xspace%
}
\def\TFT{%
	Temporal Fault Tree (TFT)%
	\gdef\TFT{TFT\xspace}%
	\gdef\TFTs{TFTs\xspace}%
	\xspace%
}
\def\TFTs{%
	Temporal Fault Trees (TFTs)%
	\gdef\TFT{TFT\xspace}%
	\gdef\TFTs{TFTs\xspace}%
	\xspace%
}
\def\DFT{%
	Dynamic Fault Tree (DFT)%
	\gdef\DFT{DFT\xspace}%
	\gdef\DFTs{DFTs\xspace}%
	\xspace%
}
\def\DFTs{%
	Dynamic Fault Trees (DFTs)%
	\gdef\DFT{DFT\xspace}%
	\gdef\DFTs{DFTs\xspace}%
	\xspace%
}
\def\FBA{%
	Free Boolean Algebra (FBA)%
	\gdef\FBA{FBA\xspace}%
	\gdef\FBAs{FBAs\xspace}%
	\gdef\iFBA{an FBA\xspace}%
	\xspace%
}
\def\FBAs{%
	Free Boolean Algebras (FBAs)%
	\gdef\FBA{FBA\xspace}%
	\gdef\FBAs{FBAs\xspace}%
	\gdef\iFBA{an FBA\xspace}%
	\xspace%
}
\def\iFBA{%
	a Free Boolean Algebra (FBA)%
	\gdef\FBA{FBA\xspace}%
	\gdef\FBAs{FBAs\xspace}%
	\gdef\iFBA{an FBA\xspace}%
	\xspace%
}
\def\ATF{%
	Algebra of Temporal Faults (ATF)%
	\gdef\ATF{ATF\xspace}%
	\gdef\iATF{an ATF\xspace}%
	\gdef\IATF{An ATF\xspace}%
	\xspace%
}
\def\IlATF{%
	An Algebra of Temporal Faults (ATF)%
	\gdef\ATF{ATF\xspace}%
	\gdef\iATF{an ATF\xspace}%
	\gdef\IATF{An ATF\xspace}%
	\xspace%
}
\def\iATF{%
	an Algebra of Temporal Faults (ATF)%
	\gdef\ATF{ATF\xspace}%
	\gdef\iATF{an ATF\xspace}%
	\gdef\IATF{An ATF\xspace}%
	\xspace%
}
\def\IATF{%
	An Algebra of Temporal Faults (ATF)%
	\gdef\ATF{ATF\xspace}%
	\gdef\iATF{an ATF\xspace}%
	\gdef\IATF{An ATF\xspace}%
	\xspace%
}
%%%%%%%%%%%%%%%%%%% Acronyms (end)
%%%%%%%%%%%%%%%%%%% Commands
\newcommand{\includegraphicsaspectratio}[2][1]{%
  \includegraphics[width=#1\textwidth,height=#1\textheight,keepaspectratio]{#2}%
}
\def\pand{\ensuremath\operatorname{<}}
\def\por{\ensuremath\operatorname{|}}
\def\sand{\ensuremath\operatorname{\&}}
\def\nibefore{\ensuremath\operatorname{\lhd}}
\def\ibefore{\ensuremath\operatorname{\unlhd}}
\def\simultaneous{\ensuremath\operatorname{\bigtriangleup}}
\def\probabilityop{\ensuremath \Pr}
\newcommand{\probability}[1]{\ensuremath \probabilityop\left({#1}\right)}
\def\varop{\ensuremath\operatorname{\mathbf{var}}}
\newcommand{\var}[1]{\ensuremath\varop #1}
\def\union{\ensuremath\operatorname{\cup}}
\def\inter{\ensuremath\operatorname{\cap}}
%\def\ftcoherencyop{\ensuremath\operatorname{\Phi}}
\def\ftcoherencyop{\ensuremath\operatorname{\phi}}
\newcommand{\ftcoherency}[1]{\ensuremath\ftcoherencyop\left(#1\right)}
\newcommand{\replace}[2]{\ensuremath\left[#1\middle/#2\right]}
%%%%%%%%%%%%%%%%%%% Commands (end)

\begin{document}

\begin{frame}[label=title]
%If you want to restart, goto \hyperlink{title}{\beamergotobutton{title}}.
\titlepage
\end{frame}

\section{Introduction}
\begin{frame}
	\frametitle{Introduction}
	\begin{itemize}
		\item Safety is a crucial concern on critical systems
		\item \FTA is a deductive method to assess safety: premisses are leaves (basic events) and conclusions are roots (top events)
		\item Structure expressions represent the structure of \FTs
		\begin{itemize}
			\item \MCSs discover and top-event probability calculus use these expressions
		\end{itemize}
		\item Other kinds of \FTs: \TFT and \DFT
		\begin{itemize}
			\item \SFT is used to distinguish original \FTs from the other kinds of \FTs
		\end{itemize}
	\end{itemize}
\end{frame}

\begin{frame}
	\frametitle{Example of structure expression}
	TODO: colocar figura de árvore e sua expressão de estrutura
\end{frame}

\begin{frame}
	\frametitle{\FTs and structure expressions}
	\begin{itemize}
		\item \SFTs: structure expressions, Shannon method (later BDDs) in the Fault Tree Handbook
		\item \TFTs: Pandora, structure expressions, and the full implementation of the Fault Tree Handbook (Papadopoulos, Walker)
		\begin{itemize}
			\item Operators: PAND, POR, SAND
		\end{itemize}
		\item \DFTs: defined as a visual representation and to improve expressiveness of Markov chains (Boyd and Dugan); structure expressions and calculus of top-event probability for \DFTs shown in the work of Merle
		\begin{itemize}
			\item Operators: CSp, SEQ, FDEP
		\end{itemize}
	\end{itemize}
\end{frame}

\subsection{Research questions}

\begin{frame}
	\frametitle{Mathematical models}
	\begin{itemize}
		\item \SFTs have a mathematical model based on \FBA (the expressions evaluate to a set of sets)
		\item There are different mathematical models for \TFTs and \DFTs, none is similar to \FBA
		\begin{itemize}
			\item \TFTs: sequence value function (evaluates to a discrete order natural number)
			\item \DFTs: date-of-occurrence function (evaluates to a date)
		\end{itemize}
		\item Models for \TFTs and \DFTs are not ready for NOT gates, although it has its importance
	\end{itemize}
\end{frame}

\begin{frame}
	\frametitle{Research questions}
	\begin{itemize}
		\item Is there a mathematical model that unifies the representation of \SFTs, \TFTs and \DFTs?
		\item Can this model represent ordering of events for \TFTs and \DFTs? 
		\item Does this model allow formula reduction to a normal form?
	\end{itemize}
\end{frame}

\subsection{Proposed solution}
\begin{frame}
	\begin{center}
		\includegraphicsaspectratio[1]{StrategyOverview}
	\end{center}
\end{frame}

\begin{frame}
	\frametitle{\IlATF}
	\begin{itemize}
		\item The algebra has a denotational semantics that evaluates to a set of lists
		\begin{itemize}
			\item In \FBAs, each set corresponds to a ``true'' value
			\item In the \ATF, each \emph{list} corresponds to a ``true'' value
		\end{itemize}
		\item The \ATF is a conservative extension of Boolean Algebra (\SFT)
		\item \TFTs and \DFTs gates can be expressed with gates that are order-based
	\end{itemize}
\end{frame}

\begin{frame}
	\frametitle{Agenda}
	\begin{itemize}
		\item Background
		\begin{itemize}
			\item systems, dependability, and fault modelling; 
			\item time relation of fault events;
		\end{itemize}
		\item Analysis and tools
		\begin{itemize}
			\item \FTA, 
			\item structure expressions, 
			\item \FBAs, 
			\item importance of the NOT operator
		\end{itemize}
		\item Contributions
		\begin{itemize}
			\item the algebra, 
			\item temporal properties, 
			\item XBefore laws, 
			\item propositions
		\end{itemize}
		\item Case study
	\end{itemize}
\end{frame}

\section{Background}

\subsection{Systems, dependability, and fault modelling}

\begin{frame}
	\frametitle{Systems and dependability}
	\begin{itemize}
		\item Systems are characterized by five properties: functionality, performance, \emph{cost}, \emph{dependability}, and security
		\item Dependability
		\begin{itemize}
			\item Threats: fault-error-failure chain
			\item Attributes: availability, reliability, \emph{safety}, integrity, maintainability
			\item Means to attain: Fault tolerance and removal by injecting faults (model-checking), theorem proving, and symbolic execution
		\end{itemize}
		\item Hardware and software are connected: software faults may cause a failure in a software-controlled hardware, and hardware faults may send incorrect date, causing a failure
	\end{itemize}
\end{frame}

\begin{frame}
	\frametitle{Fault modelling}
	\begin{itemize}
		\item SAE ARP4761 (safety assessment process for civil airborne systems) describe development cycles and methods to ``clearly identify each failure condition''
		\item \SFTs, Dependence Diagrams, and Markov chains are involved in failure identification
		\begin{itemize}
		%TODO there is a relaction of DD and SFTs, as well as DRBD abd DFTs
			\item Recall that Markov chains are the basis for \DFTs
		\end{itemize}
		\item \FTs are present in several stages of systems' modelling (which depends on fault modelling)
	\end{itemize}
\end{frame}

\subsection{Time relation of fault events}

\begin{frame}
	\frametitle{Background: time relations}
	\begin{center}
		\includegraphicsaspectratio[0.5]{time-relations}
	\end{center}
\end{frame}

\section{Analysis and tools}
\subsection{Fault Tree Analysis and structure expressions}
\begin{frame}
	\frametitle{\SFT}
	\begin{center}
		\includegraphicsaspectratio[0.9]{sft-example-ald-software}
	\end{center}
\end{frame}

\begin{frame}
	\frametitle{\SFT structure expression}
	\begin{center}
		\includegraphicsaspectratio[0.45]{ex-fault-tree1}\\
		$TOP = a \land b$
	\end{center}
\end{frame}

\begin{frame}
	\frametitle{\TFT}
	\begin{center}
		\includegraphicsaspectratio[0.55]{tft-small-example}\\
		$TOP = (A < C) \lor (A \land B)$
	\end{center}
\end{frame}

\begin{frame}
	\frametitle{\TFT's temporal truth table}
	\begin{center}
	\scriptsize
		$TOP = (A < C) \lor (A \land B)$\\
		\begin{tabular}{cccccc}
			$A$ & $B$ & $C$ & $A < C$ & $A \land B$ & $TOP$\\
			\hline
			0 & 0 & 0 & 0 & 0 & \textbf{0}\\
			\ldots\\
			1 & 0 & 2 & 2 & 0 & \textbf{2}\\
			1 & 1 & 0 & 0 & 1 & \textbf{1}\\
			\ldots\\
			1 & 1 & 2 & 2 & 1 & \textbf{1}\\
			1 & 2 & 1 & 0 & 2 & \textbf{2}\\
			1 & 2 & 2 & 2 & 2 & \textbf{2}\\
			\ldots\\
			1 & 3 & 2 & 2 & 3 & \textbf{2}\\
			\ldots\\
			2 & 1 & 3 & 3 & 2 & \textbf{2}\\
			2 & 2 & 1 & 0 & 2 & \textbf{2}\\
			\ldots\\
			3 & 1 & 2 & 0 & 3 & \textbf{3}\\
			\ldots\\
			\hline
		\end{tabular}
	\end{center}
\end{frame}

\begin{frame}[fragile]
	\frametitle{\DFT}
	\begin{center}
		\scriptsize
		\includegraphicsaspectratio[0.60]{dft-example-mrl2014}
		\begin{align*}
		SYSTEM =& CS \lor SS \lor (M \land MC) \lor \\
			&(P \land (B_d \nibefore P)) \lor (B_a \land (P \nibefore B_a)) \lor \\
			&(BP_a \land (P2 \nibefore P1) \land (P1 \nibefore BP_a)) \lor
			(P2 \land (P1 \nibefore BP_a) \land (BP_a \nibefore P2))
		\end{align*}
	\end{center}
\end{frame}

\subsection{Free Boolean algebras}

\begin{frame}
	\frametitle{\FBA}
	\begin{itemize}
		\item Generators of \iFBA are independent statements (or independent events). Some examples:
		\begin{itemize}
			\item ``valve A is stuck closed''
			\item ``motor M is malfunctioning''
		\end{itemize}
		\item Let $E$ be a set of generators. An algebra is constructed from the power set of $E$: the set of sets of the elements of $E$
	\end{itemize}
\end{frame}

\begin{frame}[fragile]
	\frametitle{\FBA}
	Inductive definition of \iFBA:
	\begin{subequations}
		\begin{align}
		\var s = \{X | s \in X\} & \implies \var s \in B & \text{(variable)}\\
		X \in B & \implies -X \in B                           & \text{(complement)}\\
		X \in B \land Y \in B & \implies X \inter Y \in B     & \text{(intersection)}
		\end{align}
	\end{subequations}
	Bottom and top of the algebra's lattice, and the union operator are obtained from DeMorgan laws.
\end{frame}

\subsection{NOT operator in SFTs}

\begin{frame}
	\frametitle{NOT operator in SFTs: coherency}
	Definition of coherency:
	\begin{itemize}
		\item $\ftcoherency{x}$ is monotonic (non-decreasing) in each variable;
		\item Each $x_i$ is relevant, which means that $\ftcoherency{x}\replace{x_i}{1} \neq \ftcoherency{x}\replace{x_i}{0}$ for some vector $x$.
	\end{itemize}
\end{frame}

\begin{frame}
	\frametitle{Non-coherent tree}
	\begin{center}
		\includegraphicsaspectratio[0.55]{non-coherent-ft-example}\\
		$S = \left(x_1 \land x_3\right) \lor \left(x_2 \land \lnot x_3\right)$\\
		$\ftcoherency{1,1,x_3}\replace{x_3}{1}=\ftcoherency{1,1,x_3}\replace{x_3}{0} \implies \text{non-coherent}$
	\end{center}
\end{frame}

\begin{frame}
	\frametitle{Usefulness of the NOT gate}
	\begin{center}
		\begin{minipage}{0.6\textwidth}
			\includegraphicsaspectratio[1]{ft-generic-failure-gas-detection-system}
		\end{minipage}
		\begin{minipage}{0.39\textwidth}
			\footnotesize
			\begin{description}
				\item[$L$:] lamp (informative)
				\item[$R_1$, $R_2$:] relays (actuators)
				\item[$D_1$, $D_2$:] sensors
				\item[$LU$:] logic control unit
			\end{description}
			\begin{align*}
				G_1 & = L \lor LU \lor \left(D_1 \land D_2\right)\\
				G_2 & = R_1 \lor LU \lor \left(D_1 \land D_2\right)\\
				G_3 & = R_2 \lor LU \lor \left(D_1 \land D_2\right)
			\end{align*}
		\end{minipage}
	\end{center}
\end{frame}

\begin{frame}
	\frametitle{Most critical outcome}
	\footnotesize
	\begin{itemize}
		\item Process isolation shuts down ($G_2$), power supply isolation fails ($G_3$) and operator is not informed ($G_1$) but the operator information system is working (lamp and siren are off, but they are operational)
	\end{itemize}
	\begin{minipage}{0.38\textwidth}
		\footnotesize
		\includegraphicsaspectratio[0.95]{outcome-4-coherent-ft}
		\begin{itemize}
			\item Coherent tree, without event ``operator not informed''
			\item Minimal cut sets: $\left\{R_1, R_2\right\}, \left\{D_1, D_2\right\}, \left\{LU\right\} $
		\end{itemize}
	\end{minipage}
	\begin{minipage}{0.52\textwidth}
		\footnotesize
		\includegraphicsaspectratio[0.95]{outcome-4-non-coherent-ft}
		\begin{itemize}
			\item Structure expression of the non-coherent tree: $\lnot L \land \lnot LU \land R_1 \land R_2 \land \left(\lnot D_1 \lor \lnot D_2\right)$
			\item Minimal cut set after approximation: $\left\{R_1, R_2\right\} $
		\end{itemize}
	\end{minipage}
\end{frame}

\subsection{Nominal model and fault injection}
\subsection{Isabelle/HOL}

\section{Contributions}
\subsection{A free algebra to express structure expressions}
\subsection{Temporal properties}
\subsection{XBefore laws}
\subsection{Propositions}

\section{Case study}
\subsection{Structure expressions with Boolean operators}
\subsection{Structure expressions with XBefore}

\section{Final remarks}
\subsection{Conclusion}

\begin{frame}
\frametitle{Status}
TODO
\end{frame}

\begin{frame}
\frametitle{Next steps in the thesis}
TODO
\end{frame}

\begin{frame}
\frametitle{Future work, out of the scope of this thesis}
TODO
\end{frame}

%\begin{frame}
%\frametitle{Motivation}
%\end{frame}


\end{document}