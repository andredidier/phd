\documentclass{beamer}

\mode<presentation>
{
\usetheme{CIn20131120}
\setbeamercovered{transparent}
}

%\AtBeginSection[]
%{
%	\begin{frame}<beamer>
%		\frametitle{Outline}
%		\tableofcontents[currentsection,hideothersubsections]
%	\end{frame}
%}

\usepackage{lmodern}
\usepackage[british]{babel}
\usepackage[utf8]{inputenc}
\usepackage[T1]{fontenc} 
\usepackage{xspace}
\usepackage{graphicx}
\usepackage{amsmath,amsfonts,amssymb,amsthm}
\usepackage{verbatim}
\usepackage{appendixnumberbeamer} 

%\DeclareGraphicsExtensions{.pdf,.png,.jpg}
\DeclareGraphicsExtensions{.pdf,.eps,.png}
%\DeclareGraphicsExtensions{.png}
\graphicspath{{./../../thesis/},{./}}

\title{An Algebra of Temporal Faults}
\subtitle{PhD Qualification}
\date{April 2016}

\author{André Didier}
\supervisor{Alexandre Cabral Mota}
\cosupervisor{Alexander Romanovsky}

\institute{
\inst{1}Federal University of Pernambuco\\
Centre of Informatics
}

\subject{Qualification presentation}

%%%%%%%%%%%%%%%%%%% Acronyms
\def\FTA{Fault Tree Analysis (FTA)\gdef\FTA{FTA\xspace}\xspace}
\def\FT{%
	Fault Tree (FT)%
	\gdef\FT{FT\xspace}%
	\gdef\FTs{FTs\xspace}%
	\gdef\IFT{An FT\xspace}%
	\xspace%
}
\def\FTs{%
	Fault Trees (FTs)%
	\gdef\FT{FT\xspace}%
	\gdef\FTs{FTs\xspace}%
	\gdef\IFT{An FT\xspace}%
	\xspace%
}
\def\IFT{%
	A Fault Tree (FT)%
	\gdef\FT{FT\xspace}%
	\gdef\FTs{FTs\xspace}%
	\gdef\IFT{An FT\xspace}%
	\xspace%
}
\def\MCS{%
	Minimal Cut Set (MCS)%
	\gdef\MCS{MCS\xspace}%
	\gdef\MCSs{MCSs\xspace}%
	\xspace%
}
\def\MCSs{%
	Minimal Cut Sets (MCSs)%
	\gdef\MCS{MCS\xspace}%
	\gdef\MCSs{MCSs\xspace}%
	\xspace%
}
\def\MCSeq{%
	Minimal Cut Sequence (MCSeq)%
	\gdef\MCSeq{MCSeq\xspace}%
	\gdef\MCSeqs{MCSeqs\xspace}%
	\xspace%
}
\def\MCSeqs{%
	Minimal Cut Sequences (MCSeqs)%
	\gdef\MCS{MCSeq\xspace}%
	\gdef\MCSs{MCSeqs\xspace}%
	\xspace%
}
\def\SFT{%
	Static Fault Tree (SFT)%
	\gdef\SFT{SFT\xspace}%
	\gdef\SFTs{SFTs\xspace}%
	\xspace%
}
\def\SFTs{%
	Static Fault Trees (SFTs)%
	\gdef\SFT{SFT\xspace}%
	\gdef\SFTs{SFTs\xspace}%
	\xspace%
}
\def\TFT{%
	Temporal Fault Tree (TFT)%
	\gdef\TFT{TFT\xspace}%
	\gdef\TFTs{TFTs\xspace}%
	\xspace%
}
\def\TFTs{%
	Temporal Fault Trees (TFTs)%
	\gdef\TFT{TFT\xspace}%
	\gdef\TFTs{TFTs\xspace}%
	\xspace%
}
\def\DFT{%
	Dynamic Fault Tree (DFT)%
	\gdef\DFT{DFT\xspace}%
	\gdef\DFTs{DFTs\xspace}%
	\xspace%
}
\def\DFTs{%
	Dynamic Fault Trees (DFTs)%
	\gdef\DFT{DFT\xspace}%
	\gdef\DFTs{DFTs\xspace}%
	\xspace%
}
\def\FBA{%
	Free Boolean Algebra (FBA)%
	\gdef\FBA{FBA\xspace}%
	\gdef\FBAs{FBAs\xspace}%
	\gdef\iFBA{an FBA\xspace}%
	\xspace%
}
\def\FBAs{%
	Free Boolean Algebras (FBAs)%
	\gdef\FBA{FBA\xspace}%
	\gdef\FBAs{FBAs\xspace}%
	\gdef\iFBA{an FBA\xspace}%
	\xspace%
}
\def\iFBA{%
	a Free Boolean Algebra (FBA)%
	\gdef\FBA{FBA\xspace}%
	\gdef\FBAs{FBAs\xspace}%
	\gdef\iFBA{an FBA\xspace}%
	\xspace%
}
\def\ATF{%
	Algebra of Temporal Faults (ATF)%
	\gdef\ATF{ATF\xspace}%
	\gdef\iATF{an ATF\xspace}%
	\gdef\IATF{An ATF\xspace}%
	\xspace%
}
\def\IlATF{%
	An Algebra of Temporal Faults (ATF)%
	\gdef\ATF{ATF\xspace}%
	\gdef\iATF{an ATF\xspace}%
	\gdef\IATF{An ATF\xspace}%
	\xspace%
}
\def\iATF{%
	an Algebra of Temporal Faults (ATF)%
	\gdef\ATF{ATF\xspace}%
	\gdef\iATF{an ATF\xspace}%
	\gdef\IATF{An ATF\xspace}%
	\xspace%
}
\def\IATF{%
	An Algebra of Temporal Faults (ATF)%
	\gdef\ATF{ATF\xspace}%
	\gdef\iATF{an ATF\xspace}%
	\gdef\IATF{An ATF\xspace}%
	\xspace%
}
\def\activation{%
	Activation Algebra (ActA)%
	\gdef\activation{ActA\xspace}%
	\xspace%
}
%%%%%%%%%%%%%%%%%%% Acronyms (end)
%%%%%%%%%%%%%%%%%%% Commands
\newcommand{\includegraphicsaspectratio}[2][1]{%
  \includegraphics[width=#1\textwidth,height=#1\textheight,keepaspectratio]{#2}%
}
\def\pand{\ensuremath\operatorname{<}}
\def\por{\ensuremath\operatorname{|}}
\def\sand{\ensuremath\operatorname{\&}}
\def\nibefore{\ensuremath\operatorname{\lhd}}
\def\ibefore{\ensuremath\operatorname{\unlhd}}
\def\simultaneous{\ensuremath\operatorname{\bigtriangleup}}
\def\probabilityop{\ensuremath \Pr}
\newcommand{\probability}[1]{\ensuremath \probabilityop\left({#1}\right)}
\def\varop{\ensuremath\operatorname{\mathbf{var}}}
\newcommand{\var}[1]{\ensuremath\varop #1}
\newcommand{\sliceright}[2]{\ensuremath #1_{\left[..#2\right]}}
\newcommand{\sliceleft}[2]{\ensuremath #1_{\left[#2..\right]}}
\newcommand{\slice}[3]{\ensuremath #1_{\left[#2..#3\right]}}
\def\xbeforeop{\ensuremath\rightarrow}
\newcommand{\xbefore}[2]{\ensuremath #1 \xbeforeop #2 }
\newcommand{\xbeforedef}[2]{\ensuremath \left\{ zs | \exists i \bullet \sliceright{zs}{i} \land \sliceleft{zs}{i}  \right\}}
\def\union{\ensuremath\operatorname{\cup}}
\def\inter{\ensuremath\operatorname{\cap}}
%\def\ftcoherencyop{\ensuremath\operatorname{\Phi}}
\def\ftcoherencyop{\ensuremath\operatorname{\phi}}
\newcommand{\ftcoherency}[1]{\ensuremath\ftcoherencyop\left(#1\right)}
\newcommand{\replace}[2]{\ensuremath\left[#1\middle/#2\right]}
\def\CSPm{CSP$_M$\xspace}
\newenvironment{snippetcspm}[1][2]
{
\ifthenelse{\equal{#1}{0}}
    {\tiny}
    {
    \ifthenelse{\equal{#1}{1}}
        {\scriptsize}
        {
        \ifthenelse{\equal{#1}{2}}
            {\footnotesize}
            {\small}
        }
    }
%\begin{samepage}
\verbatim
}
{
\endverbatim
%\end{samepage}
}
\def\tempoop{\ensuremath\operatorname{\mathbf{tempo}}}
\newcommand{\tempo}[2][1-4]{\ensuremath\tempoop_{#1} #2}
\def\algebraset{\ensuremath\operatorname{\mathbf{atf}}}
\def\tracetobool{\ensuremath\operatorname{\stackrel{\leadsto}{\text{\tiny B}}}}
\def\tracetofba{\ensuremath\operatorname{\stackrel{\leadsto}{\text{\tiny F}}}}
\def\tracetoalgebra{\ensuremath\operatorname{\stackrel{\leadsto}{\text{\tiny XB}}}}
\newcommand{\append}[2]{\ensuremath #1 \operatorname{\mathbf{@}} #2}
%%%%%%%%%%%%%%%%%%% Commands (end)

\begin{document}

\begin{frame}[label=title]
%If you want to restart, goto \hyperlink{title}{\beamergotobutton{title}}
\titlepage
\end{frame}

\section{Introduction}
\begin{frame}{Motivation}
	\begin{itemize}
		\item Safety is a crucial concern on critical systems
		%\item \FTA is a deductive method to assess safety: premisses are leaves (basic events) and conclusions are roots (top events)
		\item \FTA is a deductive method to assess safety: from premisses (leaves, basic events) it deduces the conclusions (roots, top events)
		\item Structure expressions represent the structure of \FTs
		\begin{itemize}
			\item \MCSs discover and top-event probability calculus use these expressions
		\end{itemize}
		\item Other kinds of \FTs: \TFT and \DFT
		\begin{itemize}
			\item Sequences of events---\MCSeqs---that lead to top-events
			\item \SFT is used to distinguish original \FTs from the other kinds of \FTs
		\end{itemize}
	\end{itemize}
\end{frame}

\begin{frame}{\FTs and structure expressions}
	\begin{itemize}
		\item \SFTs: structure expressions, Shannon's method (later, BDDs) in the Fault Tree Handbook
		\item \TFTs: Pandora, structure expressions, and the full implementation of the Fault Tree Handbook (Papadopoulos and Walker)
		\begin{itemize}
			\item Gates: PAND, POR, SAND
		\end{itemize}
		\item \DFTs: defined as a visual representation and to improve expressiveness of Markov chains (Boyd and Dugan); structure expressions and calculus of top-event probability for \DFTs shown in the work of Merle
		\begin{itemize}
			\item Gates: CSp, SEQ, FDEP
		\end{itemize}
	\end{itemize}
\end{frame}

\subsection{Research questions}

\begin{frame}{Mathematical models}
	\begin{itemize}
		\item \SFTs have a mathematical model based on \FBA (the expressions evaluate to a set of sets)
		\item There are different mathematical models for \TFTs and \DFTs, none is similar to \FBA
		\begin{itemize}
			\item \TFTs: sequence value function (evaluates to a discrete order natural number)
			\item \DFTs: date-of-occurrence function (evaluates to a date)
		\end{itemize}
		\item Models for \TFTs and \DFTs are not ready for NOT gates, although it has its importance
	\end{itemize}
\end{frame}

\begin{frame}{Research questions}
	\begin{itemize}
		\item Is there a mathematical model that unifies the representation of \SFTs, \TFTs and \DFTs?
		\item Can this model represent ordering of events for \TFTs and \DFTs? 
		\item Does this model allow formula reduction to a normal form?
	\end{itemize}
\end{frame}

\subsection{Proposed solution}
\begin{frame}
	\begin{center}
		\includegraphicsaspectratio[1]{StrategyOverview}
	\end{center}
\end{frame}

\begin{frame}{\IlATF}
	\begin{itemize}
		\item The algebra has a denotational semantics that evaluates to a set of lists
		\begin{itemize}
			\item In \FBAs, each set corresponds to a ``true'' value
			\item In the \ATF, each \emph{list} corresponds to a ``true'' value
		\end{itemize}
		\item The \ATF is a conservative extension of Boolean Algebra (\SFT)
		\item \TFTs and \DFTs gates can be expressed with an order-based operator in the algebra
	\end{itemize}
\end{frame}

\begin{frame}{Agenda}
	\begin{itemize}
		\item Background
		\begin{itemize}
			\item systems, dependability, and fault modelling; 
			\item time relation of fault events;
		\end{itemize}
		\item Analysis and tools
		\begin{itemize}
			\item \FTA, 
			\item structure expressions, 
			\item \FBAs, 
			\item importance of the NOT operator
		\end{itemize}
		\item Contributions
		\begin{itemize}
			\item the algebra, 
			\item temporal properties, 
			\item XBefore laws, 
		\end{itemize}
		\item Case study
	\end{itemize}
\end{frame}

\section{Background}

\subsection{Systems, dependability, and fault modelling}

\begin{frame}{Systems and dependability}
	\begin{itemize}
		\item Systems are characterized by five properties: functionality, performance, \emph{cost}, \emph{dependability}, and security
		\item Dependability
		\begin{itemize}
			\item Threats: fault-error-failure chain
			\item Attributes: availability, reliability, \emph{safety}, integrity, maintainability
			\item Means to attain: Fault tolerance and removal by injecting faults (model-checking), theorem proving, and symbolic execution
		\end{itemize}
		\item Hardware and software are connected: software faults may cause a failure in a software-controlled hardware, and hardware faults may send incorrect date, causing a failure
	\end{itemize}
\end{frame}

\begin{frame}{Fault modelling}
	\begin{itemize}
		\item SAE ARP4761 (safety assessment process for civil airborne systems) describe development cycles and methods to ``clearly identify each failure condition''
		\item \SFTs, Dependence Diagrams, and Markov chains are involved in failure identification
		\begin{itemize}
		%TODO there is a relaction of DD and SFTs, as well as DRBD abd DFTs
			\item Recall that Markov chains are the basis for \DFTs
		\end{itemize}
		\item \FTs are present in several stages of systems' modelling (which depends on fault modelling)
	\end{itemize}
\end{frame}

\subsection{Time relation of fault events}

\begin{frame}{Background: time relations}
	\begin{center}
		\includegraphicsaspectratio[0.5]{time-relations}
	\end{center}
\end{frame}

\section{Analysis and tools}
\subsection{Fault Tree Analysis and structure expressions}
\begin{frame}{\SFT}
	\begin{center}
		\includegraphicsaspectratio[0.9]{sft-example-ald-software}
	\end{center}
\end{frame}

\begin{frame}{\SFT and structure expression}
	\begin{center}
		\includegraphicsaspectratio[0.45]{ex-fault-tree1}\\
		$TOP = a \land b$
	\end{center}
\end{frame}

\begin{frame}{\TFT and structure expression}
	\begin{center}
		\includegraphicsaspectratio[0.55]{tft-small-example}\\
		$TOP = (A < C) \lor (A \land B)$
	\end{center}
\end{frame}

\begin{frame}{\TFT's temporal truth table}
	\begin{center}
	\scriptsize
		$TOP = (A < C) \lor (A \land B)$\\
		\begin{tabular}{cccccc}
			$A$ & $B$ & $C$ & $A < C$ & $A \land B$ & $TOP$\\
			\hline
			0 & 0 & 0 & 0 & 0 & \textbf{0}\\
			\ldots\\
			1 & 0 & 2 & 2 & 0 & \textbf{2}\\
			1 & 1 & 0 & 0 & 1 & \textbf{1}\\
			\ldots\\
			1 & 1 & 2 & 2 & 1 & \textbf{1}\\
			1 & 2 & 1 & 0 & 2 & \textbf{2}\\
			1 & 2 & 2 & 2 & 2 & \textbf{2}\\
			\ldots\\
			1 & 3 & 2 & 2 & 3 & \textbf{2}\\
			\ldots\\
			2 & 1 & 3 & 3 & 2 & \textbf{2}\\
			2 & 2 & 1 & 0 & 2 & \textbf{2}\\
			\ldots\\
			3 & 1 & 2 & 0 & 3 & \textbf{3}\\
			\ldots\\
			\hline
		\end{tabular}
	\end{center}
\end{frame}

\begin{frame}[fragile]{\DFT structure expression}
	\begin{center}
		\scriptsize
		\includegraphicsaspectratio[0.60]{dft-example-mrl2014}
		\begin{align*}
		SYSTEM =& CS \lor SS \lor (M \land MC) \lor \\
			&(P \land (B_d \nibefore P)) \lor (B_a \land (P \nibefore B_a)) \lor \\
			&(BP_a \land (P2 \nibefore P1) \land (P1 \nibefore BP_a)) \lor
			(P2 \land (P1 \nibefore BP_a) \land (BP_a \nibefore P2))
		\end{align*}
	\end{center}
\end{frame}

\subsection{Free Boolean algebras}

\begin{frame}[label=fba]{Definition of \iFBA}
	\begin{itemize}
		\item Generators of \iFBA are independent statements (or independent events). Some examples:
		\begin{itemize}
			\item ``valve A is stuck closed''
			\item ``motor M is malfunctioning''
		\end{itemize}
		\item Let $E$ be a set of generators. An algebra is constructed from the power set of $E$: the set of sets of the elements of $E$
	\end{itemize}
	\hyperlink{fbadef}{\beamergotobutton{FBA}}
\end{frame}

\begin{frame}{Formulas in \FBA}
	\begin{itemize}
		\item \IFT event is a statement in \FBA
		\item For example, fault $F_1$ and $F_2$ becomes $\var F_1$ and $\var F_2$ in \FBA (the set of generators contains only the two events)
		\item If a top event is given by $F_1 \land F_2$, the expression in \FBA is $\var F_1 \inter \var F_2$
		\begin{itemize}
			\item The resulting set is: 
			$\left\{
				\left\{F_1, F_2\right\}
			\right\}$
		\end{itemize}
		\item The NOT operator needs the Universal set
		\begin{itemize}
			\item $\lnot \left(F_1 \land F_2\right)$ is equivalent to the expression $- \left(\var F_1 \inter \var F_2\right)$, which results in the set:
			$\left\{
				\left\{\right\},
				\left\{F_1\right\},
				\left\{F_2\right\}
			\right\}$
		\end{itemize}
	\end{itemize}
\end{frame}

\subsection{NOT operator in SFTs}

\begin{frame}[label=not]{NOT operator in SFTs: coherency}
	\begin{itemize}
		\item Coherent trees are those that dot not have NOT gates or have them removed
		\item Removal of negated elements is possible due to an approximation: if the probability of occurrence is close to 1, the probability of non-occurrence is negligible (it depends on the probability of the other events)
		\item In a coherent tree, each $x_i$ is relevant, which means that $\ftcoherency{x}\replace{x_i}{1} \neq \ftcoherency{x}\replace{x_i}{0}$ for some vector $x$
	\end{itemize}
	\hyperlink{coherencydef}{\beamergotobutton{Coherency}}
\end{frame}

\begin{frame}{Non-coherent tree}
	\begin{center}
		\includegraphicsaspectratio[0.50]{non-coherent-ft-example}\\
	\end{center}
	\footnotesize
	\begin{itemize}
		\item $S = \left(x_1 \land x_3\right) \lor \left(x_2 \land \lnot x_3\right)$ ($x_1 = \text{take a ride}$, $x_2 = \text{take the metro}$, $x_3 = \text{wake up early}$)
		\item $\ftcoherency{1,1,x_3}\replace{x_3}{1}=\ftcoherency{1,1,x_3}\replace{x_3}{0} \implies \text{non-coherent}$
		\item The tree misleads. One possibility from the tree is that: the college student would take a ride AND take the metro ($x_1 \land x_2$)
	\end{itemize}
\end{frame}

\begin{frame}{Usefulness of the NOT gate}
	\begin{center}
		\begin{minipage}{0.6\textwidth}
			\includegraphicsaspectratio[1]{ft-generic-failure-gas-detection-system}
		\end{minipage}
		\begin{minipage}{0.39\textwidth}
			\footnotesize
			\begin{description}
				\item[$L$:] lamp and siren (informative)
				\item[$R_1$, $R_2$:] relays (actuators)
				\item[$D_1$, $D_2$:] sensors
				\item[$LU$:] logic control unit
			\end{description}
			\begin{align*}
				G_1 & = L \lor LU \lor \left(D_1 \land D_2\right)\\
				G_2 & = R_1 \lor LU \lor \left(D_1 \land D_2\right)\\
				G_3 & = R_2 \lor LU \lor \left(D_1 \land D_2\right)
			\end{align*}
		\end{minipage}
	\end{center}
\end{frame}

\begin{frame}{Most critical outcome}
	\footnotesize
	\begin{itemize}
		\item Process shut down fails ($G_2$), power supply isolation fails ($G_3$) and operator is not informed ($G_1$), but the operator information system is working (lamp and siren are off, but they are operational)
	\end{itemize}
	\begin{minipage}{0.38\textwidth}
		\footnotesize
		\includegraphicsaspectratio[0.95]{outcome-4-coherent-ft}
		\begin{itemize}
			\item Coherent tree, without event ``operator not informed''
			\item Minimal cut sets: $\left\{R_1, R_2\right\}, \left\{D_1, D_2\right\}, \left\{LU\right\} $
		\end{itemize}
	\end{minipage}
	\begin{minipage}{0.52\textwidth}
		\footnotesize
		\includegraphicsaspectratio[0.95]{outcome-4-non-coherent-ft}
		\begin{itemize}
			\item Structure expression of the non-coherent tree: $\lnot L \land \lnot LU \land R_1 \land R_2 \land \left(\lnot D_1 \lor \lnot D_2\right)$
			\item Minimal cut set after approximation: $\left\{R_1, R_2\right\} $
		\end{itemize}
	\end{minipage}
\end{frame}

\subsection{Nominal model and fault injection}

\begin{frame}[fragile]{Monitor component nominal model}
	\begin{itemize}
		\item A monitor is a component commonly used to improve fault tolerance
	\end{itemize}
	\begin{center}
		\includegraphicsaspectratio[0.60]{blockDiagramMonitorInternals}
	\end{center}
	\begin{itemize}
		\item Each component of the monitor has a \CSPm process
		\item The model is the synchronization of the components' processes
		\item Temporal dependency of events through a \verb$tick$ event
	\end{itemize}
\end{frame}

\begin{frame}[fragile]{Monitor component breakable model}
	\footnotesize
	\begin{center}
		\includegraphicsaspectratio[0.60]{monitor-breakable}
	\end{center}
	\begin{itemize}
		\item For each hardware signal, a new signal process is created and synchronizes in inputs and outputs of the components
		\item The breakable version allows the hardware signals to behave arbitrarily, changing output values or omitting them
		\item The failure cases are those when the observed output on the breakable version is different from the nominal one
	\end{itemize}
\end{frame}

\begin{frame}[fragile]{\SFT structure expressions from fault injection output traces}
	\begin{itemize}
		\item From the traces generated by the observer, we extracted failure expressions (local structure expressions)
		\item These expressions ignore order and consider only event occurrence
		\item Example of trace and resulting expression (\verb$MonIn1$ $=A$, \verb$RelationalOperator$ $=S$):
		\begin{snippetcspm}[1]
			TRACE 5:
			failure.Hardware.N04_MonIn1.1.EXP.I.5
			failure.Hardware.N04_MonIn1.1.ACT.OMISSION
			failure.Hardware.N04_RelationalOperator.1.EXP.B.false
			failure.Hardware.N04_RelationalOperator.1.ACT.B.true
			out.1.OMISSION
		\end{snippetcspm}
		$A \land S$
	\end{itemize}
\end{frame}

\section{Contributions}
\subsection{A free algebra to express structure expressions}

\begin{frame}{Representing three kinds of \FTs}
	\begin{itemize}
		\item Structure expressions of \SFTs are not related to order
		\item Existing structure expressions of \TFTs and \DFTs can represent \SFTs' structure expressions, but non can represent NOT gates
		\item \TFT's structure expressions is based on a sequence value function
		\item \DFT's is based on a date-of-occurrence function
		\item All three kinds can be represented by an order-based operator
	\end{itemize}
\end{frame}

\begin{frame}[label=thealgebra]{The algebra}
	\begin{itemize}
		\item Why lists? Built-in order representation.
		\item Boolean algebra lattice order: set inclusion
		\begin{itemize}
			\item In \FBAs, the formula for $A \lor B$ is the set of sets: 
			$\left\{
				\left\{A\right\},
				\left\{B\right\},
				\left\{A,B\right\}
			\right\}$
		\end{itemize}
		\item The algebra with lists is indeed a Boolean algebra? How to represent the same formula with lists? \hyperlink{inductiveatf}{\beamergotobutton{inductive \ATF}}
		\begin{itemize}
			\item Answer: all permutations
			\item The same formula with lists is the set: 
			$\left\{
				\left[A\right],
				\left[B\right],
				\left[A,B\right],
				\left[B,A\right]
			\right\}$
		\end{itemize}
		\item We show that it is valid for any other Boolean formula, including the NOT operator %by implementing the boolean class in Isabelle/HOL
		\item Normal form similar to a DNF: terms may contain ANDs, NOTs, and a new order-based operator.
	\end{itemize}
\end{frame}

\begin{frame}[label=expressorder]{How to express order?}
	\begin{itemize}
		\item The Boolean operators are not enough to distinguish order
		\begin{itemize}
			\item So we need new operators
			\item How many? Just one.
		\end{itemize}
		\item The exclusive before (XBefore) expresses that one variable comes before another \hyperlink{xbeforedefs}{\beamergotobutton{definition}}
		\item The result of $A\land B$ is
		$\left\{
			\left[A,B\right],
			\left[B,A\right]
		\right\}$
		\item The result of $\xbefore{A}{B}$ is
		$\left\{
			\left[A,B\right]
		\right\}$
		\item Not always trivial. What if there are three variables ($A$, $B$, and $C$)?
		\begin{itemize}
			\item $A\land B$ is 
				$\left\{
					\left[A,B\right],
					\left[A,B,C\right], 
					\left[A,C,B\right],
					\left[C,A,B\right],\right. $
					$\left.
						\left[B,A\right],
						\left[B,A,C\right], 
						\left[B,C,A\right],
						\left[C,B,A\right]
					\right\}$
			\item $\xbefore{A}{B}$ is 
				$\left\{
					\left[A,B\right], 
					\left[A,B,C\right], 
					\left[A,C,B\right],
					\left[C,A,B\right]
				\right\} $
		\end{itemize}
		\item And the combination of operators?
	\end{itemize}
\end{frame}

%\subsection[Tempo properties]{\emph{Tempo} properties}
\subsection{Tempo properties}

\begin{frame}[label=tempo]{\emph{Tempo} properties}
	\begin{itemize}
		\item The properties are:
		\begin{enumerate}
			\item Disjoint split: if the first part of a list is in the set, then every remainder is not;
			\item If a sublist is in a list, then is must be in a sublist in the beginning or at the end;
			\item If a sublist is in the middle, then there a sublist that ends at the end of the sublist and there is another that starts at the beginning of the sublist
			\item If a generator belongs to a list, then there is a sublist of size one that contains the generator
		\end{enumerate}
		\item Odd \emph{tempo} properties are related to the AND operator and even \emph{tempo} properties are related to the OR operator
		\item The XBefore laws depend on these properties
		\item All properties apply for variables (the previous examples)
		% Provided the tempo properties apply, our work is equivalent to the work of Merle 
	\end{itemize}
	\hyperlink{tempodetail}{\beamergotobutton{definitions}}
\end{frame}

\subsection{XBefore laws}

\begin{frame}{The XBefore laws}
	\begin{itemize}
		\item The laws are divided into three categories
		\item Basic: absorption, non-idempotency ($\tempo[1]$)
		\item Relation with Boolean operators: intersection ($\tempo[1]$) and union ($\tempo$)
		\item Reduction: associativity ($\tempo[1]$) and distributivity ($\tempo$)
	\end{itemize}
\end{frame}

\section{Case study}

\begin{frame}{Case study: monitor}
	\begin{center}
		\includegraphicsaspectratio[0.60]{blockDiagramMonitorInternals}
	\end{center}
	\begin{align*}
		a &= \text{LowPower-In1}& A = \var a\\
		b &= \text{LowPower-In2}& B = \var b\\
		s &= \text{SwitchFailure}& S = \var s
	\end{align*}
\end{frame}

\subsection{Structure expressions with Boolean operators}

\begin{frame}[fragile]{Mapping each trace from the \CSPm model to \ATF}{Boolean operators only, without XBefore}
{\scriptsize
	\begin{align*}
		\text{\texttt{TRACE 1: }}&[s,b] \tracetobool S \inter B \inter -A & \left\{\left[s,b\right],\left[b,s\right]\right\}\\
		\text{\texttt{TRACE 2: }}&[b,s] \tracetobool B \inter S \inter -A & \left\{\left[s,b\right],\left[b,s\right]\right\}\\
		\text{\texttt{TRACE 3: }}&[a,b] \tracetobool A \inter B \inter -S & \left\{\left[a,b\right],\left[b,a\right]\right\}\\
		\text{\texttt{TRACE 4: }}&[b,a] \tracetobool B \inter A \inter -S & \left\{\left[a,b\right],\left[b,a\right]\right\}\\
		\text{\texttt{TRACE 5: }}&[a,s] \tracetobool A \inter S \inter -B & \left\{\left[a,s\right],\left[s,a\right]\right\}\\
		\text{\texttt{TRACE 6: }}&[a,s,b] \tracetobool A \inter S \inter B & \left\{\left[a,b,s\right],\left[a,s,b\right],\ldots,\left[s,b,a\right]\right\}\\
		\text{\texttt{TRACE 7: }}&[a,b,s] \tracetobool A \inter B \inter S & \left\{\left[a,b,s\right],\left[a,s,b\right],\ldots,\left[s,b,a\right]\right\}\\
		\text{\texttt{TRACE 8: }}&[b,a,s] \tracetobool B \inter A \inter S & \left\{\left[a,b,s\right],\left[a,s,b\right],\ldots,\left[s,b,a\right]\right\}
	\end{align*}
}

Combining the above sets with unions, we obtain a set that is equivalent to $(A \inter B) \union (S \inter (A \union B))$
\end{frame}

\subsection{Structure expressions with XBefore}
\begin{frame}[fragile]{Mapping each trace from the \CSPm model to \ATF}{With Boolean operators and the XBefore}
{\scriptsize
	\begin{align*}
		\text{\texttt{TRACE 1: }}&[s,b] \tracetoalgebra \left(\xbefore{S}{B}\right) \inter -A & \left\{[s,b]\right\}\\
		\text{\texttt{TRACE 2: }}&[b,s] \tracetoalgebra \left(\xbefore{B}{S}\right) \inter -A & \left\{[b,s]\right\}\\
		\text{\texttt{TRACE 3: }}&[a,b] \tracetoalgebra \left(\xbefore{A}{B}\right) \inter -S & \left\{[a,b]\right\}\\
		\text{\texttt{TRACE 4: }}&[b,a] \tracetoalgebra \left(\xbefore{B}{A}\right) \inter -S & \left\{[b,a]\right\}\\
		\text{\texttt{TRACE 5: }}&[a,s] \tracetoalgebra \left(\xbefore{A}{S}\right) \inter -B & \left\{[a,s]\right\}\\
		\text{\texttt{TRACE 6: }}&[a,s,b] \tracetoalgebra \xbefore{\xbefore{A}{S}}{B} & \left\{[a,s,b]\right\}\\
		\text{\texttt{TRACE 7: }}&[a,b,s] \tracetoalgebra \xbefore{\xbefore{A}{B}}{S} & \left\{[a,b,s]\right\}\\
		\text{\texttt{TRACE 8: }}&[b,a,s] \tracetoalgebra \xbefore{\xbefore{B}{A}}{S} & \left\{[b,a,s]\right\}
	\end{align*}}
\end{frame}

\begin{frame}{Analysing the expression obtained using \ATF}
	\begin{itemize}
		\item Combining the sets with unions, we obtain a set that is equivalent to 
$\left(B \inter S \inter -A\right) \union \left(B \inter A \inter -S\right) \union \left(\xbefore{A}{S}\right)$
	\end{itemize}
	\begin{center}
		\includegraphicsaspectratio[0.60]{blockDiagramMonitorInternals}
	\end{center}
	\begin{itemize}
		\item If the switcher fails and connects the second output, \emph{then} the first input fails, the system is in an operational state
	\end{itemize}
\end{frame}

\section{Final remarks}
\subsection{Conclusion}

\begin{frame}{Conclusion}
	\begin{itemize}
		\item A theory was developed to support a more precise representation of fault events compared to our previous strategy for injecting faults
		\item The theory can represent \SFTs, \TFTs and \DFTs with the XBefore operator and has a denotational semantics based on a set of lists
		%(still needs to prove the equivalence to existing algebras for \TFTs and \DFTs)
	\end{itemize}
\end{frame}

\begin{frame}{Status}
	\begin{center}
		\includegraphicsaspectratio[0.8]{StrategyOverview-Status}
	\end{center}
\end{frame}

\begin{frame}[label=nextsteps]{Next steps in the thesis}
	\begin{itemize}
		\item Soundness and completeness of the \ATF
		\item Mapping function from traces to \ATF
		\item Faults modelling (\activation) 
		\item Automatic \FTA (\ATF reduction and FT criteria verification)
		\item Relations of NOT and XBefore
	\end{itemize}
	\hyperlink{tasks}{\beamergotobutton{Tasks schedule}}
\end{frame}

\begin{frame}{Future work, out of the scope of this thesis}
	\begin{itemize}
		\item Apply a similar idea to Sequential BDDs to our approach to reduce expressions to a normal form
		\item Define and develop a theory of the automatic proposal of architectural model modifications based on \DFTs
	\end{itemize}
\end{frame}

\begin{frame}[label=end]
%If you want to restart, goto \hyperlink{title}{\beamergotobutton{title}}.
\titlepage
\end{frame}

\appendix
\section{Detailed information}

\begin{frame}
	The following slides are used to show detailed information about some topic in the presentation.
\end{frame}

\subsection{Definition of \iFBA}

\begin{frame}[fragile,label=fbadef]{Inductive definition of an FBA}
	The inductive definition of \iFBA is:
	\begin{subequations}
		\begin{align}
		\var s = \{X | s \in X\} & \implies \var s \in B & \text{(variable)}\\
		X \in B & \implies -X \in B                           & \text{(complement)}\\
		X \in B \land Y \in B & \implies X \inter Y \in B     & \text{(intersection)}
		\end{align}
	\end{subequations}
	Bottom and top of the algebra's lattice, and the union operator are obtained from the inductive definition.

	\hyperlink{fba}{\beamergotobutton{Back}}
\end{frame}

\subsection{Definition of coherency}

\begin{frame}[fragile,label=coherencydef]{Definition of coherency}
	\begin{itemize}
		\item The coherency function $\ftcoherency{x}$ is monotonic (non-decreasing) in each variable;
		\item Each $x_i$ is relevant, which means that $\ftcoherency{x}\replace{x_i}{1} \neq \ftcoherency{x}\replace{x_i}{0}$ for some vector $x$
	\end{itemize}
	\hyperlink{not}{\beamergotobutton{Back}}
\end{frame}

\subsection{\ATF as a Boolean algebra}

\begin{frame}[label=inductiveatf]{\ATF as a Boolean algebra}
	\begin{align*}
		\var x & \in \algebraset & \text{Variable}\\
		S \in \algebraset \implies -S & \in \algebraset & \text{Complement, Negation}\\
		S \in \algebraset \land T \in \algebraset \implies S \inter T & \in \algebraset & \text{Intersection, \emph{Infimum}}\\
		S \in \algebraset \land T \in \algebraset \implies \xbefore{S}{T} & \in \algebraset & \text{XBefore}\\
		\intertext{Following the definitions, the expressions below are also valid for $\algebraset$:}
		UNIV &\in \algebraset & \text{Universal set, True}\\
		\{\} &\in \algebraset & \text{Empty set, False}\\
		S \in \algebraset \land T \in \algebraset \implies S \union T &\in \algebraset & \text{Union, \emph{Supremum}}
	\end{align*}
	\hyperlink{thealgebra}{\beamergotobutton{Back}}
\end{frame}

\subsection{XBefore definitions}

\begin{frame}[label=xbeforedefs]{XBefore definitions}
	\begin{itemize}
		\item Using list concatenation:
			{\scriptsize
						$\xbefore{S}{T} =
									  \left\{
									    zs | \exists xs, ys \bullet \left(set~xs\right) \inter \left(set~ys\right) = \{\}
									      \land xs \in S \land ys \in T \land zs = \append{xs}{ys}
									  \right\}$}
		\item Using list split:
			{\scriptsize
				$\xbefore{S}{T} =
									\left\{
										zs | \exists i \bullet \sliceright{zs}{i} \in S \land \sliceleft{zs}{i} \in T
									\right\}$}
	\end{itemize}
	\hyperlink{expressorder}{\beamergotobutton{Back}}
\end{frame}

\subsection{Tempo properties definitions}

\begin{frame}[label=tempodetail]{\emph{Tempo} properties definitions}
	{\scriptsize\begin{align*}
			\tempo[1]{S} &= \forall i, j, zs \bullet
			  i \le j \implies
			  \lnot \left(
			    \sliceright{zs}{i} \in S \land \sliceleft{zs}{j} \in S
			  \right)\\
			\tempo[2]{S} &= \forall i, zs \bullet
			  zs \in S \iff
			  \sliceright{zs}{i} \in S \lor \sliceleft{zs}{i} \in S\\
			\tempo[3]{S} &= \forall i, j, zs \bullet
			  j < i \implies
			  \left(
			    \slice{zs}{j}{i} \in S \iff \sliceright{zs}{i} \in S \land \sliceleft{zs}{j} \in S
			  \right)\\
			\tempo[4]{S} &= \forall zs \bullet zs \in S \iff (\exists i \bullet \slice{zs}{i}{\left(i+1\right)} \in S)
			%\intertext{Replace $\tempo[4]{S}$ by (less restrictive):}
			%&= \forall zs \bullet zs \in S \iff \exists i,j \bullet i < j \land \slice{zs}{i}{j} \in S
		\end{align*}}
	\hyperlink{tempo}{\beamergotobutton{Back}}
\end{frame}

\subsection{Tasks schedule}

\begin{frame}[label=tasks]{Tasks schedule}
	\begin{center}
		\scriptsize
		\begin{tabular}{p{6.7cm}cccc}
			\hline
			Task & 2nd & 3rd & 4th & 1st\\
			\hline \hline
			Qualification & $\bullet$ &  &  & \\
			\hline
			Elaborate a theory for the \activation & $\bullet$ &  &  & \\
			\hline
			Elaborate a theory for the acceptance criteria & $\bullet$ &  &  & \\
			\hline
			Prepare a paper about \activation and acceptance criteria & $\bullet$ & $\bullet$ &  & \\
			\hline
			Submit paper about \activation and acceptance criteria &  & $\bullet$ &  & \\
			\hline
			Prove soundness and completeness theorems for the DNF of \ATF &  & $\bullet$ & $\bullet$ & \\
			\hline
			Define the mapping rules from traces to \ATF &  &  & $\bullet$ & \\
			\hline
			Demonstrate the relations of NOT and XBefore and other operators &  &  & $\bullet$ & \\
			\hline
			Define the conditions that cause non-coherent analysis with NOT &  &  & $\bullet$ & \\
			\hline
			Write the results in the thesis &  & $\bullet$ & $\bullet$ & \\
			\hline
			Prepare thesis' defence &  &  & $\bullet$ & $\bullet$\\
			\hline
			Defence &  &  &  & $\bullet$\\
			\hline
		\end{tabular}
	\end{center}
	{\footnotesize 2nd, 3rd, 4th: second, third, and fourth quarter of 2016; 1st: first quarter of 2017}
	\hyperlink{nextsteps}{\beamergotobutton{Next steps}}
\end{frame}


\end{document}