\documentclass{beamer}

\mode<presentation>
{
\usetheme{CIn20131120}
\setbeamercovered{transparent}
}

%\AtBeginSection[]
%{
%	\begin{frame}<beamer>
%		\frametitle{Outline}
%		\tableofcontents[currentsection,hideothersubsections]
%	\end{frame}
%}

\usepackage[british]{babel}
\usepackage[utf8]{inputenc}
\usepackage[T1]{fontenc} 
\usepackage{xspace}

\title{An Algebra of Temporal Faults}
\subtitle{PhD Qualification}
\date{April 2016}

\author{André Didier}

\institute{
\inst{1}Federal University of Pernambuco\\
Centre of Informatics
}

\subject{Qualification presentation}

%%%%%%%%%%%%%%%%%%% Acronyms
\def\FTA{Fault Tree Analysis (FTA)\gdef\FTA{FTA\xspace}\xspace}
\def\FT{%
	Fault Tree (FT)%
	\gdef\FT{FT\xspace}%
	\gdef\FTs{FTs\xspace}%
	\xspace%
}
\def\FTs{%
	Fault Trees (FTs)%
	\gdef\FT{FT\xspace}%
	\gdef\FTs{FTs\xspace}%
	\xspace%
}
\def\MCS{%
	Minimal Cut Set (MCS)%
	\gdef\MCS{MCS\xspace}%
	\gdef\MCSs{MCSs\xspace}%
	\xspace%
}
\def\MCSs{%
	Minimal Cut Sets (MCSs)%
	\gdef\MCS{MCS\xspace}%
	\gdef\MCSs{MCSs\xspace}%
	\xspace%
}
\def\SFT{%
	Static Fault Tree (SFT)%
	\gdef\SFT{SFT\xspace}%
	\gdef\SFTs{SFTs\xspace}%
	\xspace%
}
\def\SFTs{%
	Static Fault Trees (SFTs)%
	\gdef\SFT{SFT\xspace}%
	\gdef\SFTs{SFTs\xspace}%
	\xspace%
}
\def\TFT{%
	Temporal Fault Tree (TFT)%
	\gdef\TFT{TFT\xspace}%
	\gdef\TFTs{TFTs\xspace}%
	\xspace%
}
\def\TFTs{%
	Temporal Fault Trees (TFTs)%
	\gdef\TFT{TFT\xspace}%
	\gdef\TFTs{TFTs\xspace}%
	\xspace%
}
\def\DFT{%
	Dynamic Fault Tree (DFT)%
	\gdef\DFT{DFT\xspace}%
	\gdef\DFTs{DFTs\xspace}%
	\xspace%
}
\def\DFTs{%
	Dynamic Fault Trees (DFTs)%
	\gdef\DFT{DFT\xspace}%
	\gdef\DFTs{DFTs\xspace}%
	\xspace%
}
\def\FBA{%
	Free Boolean Algebra (FBA)%
	\gdef\FBA{FBA\xspace}%
	\gdef\FBAs{FBAs\xspace}%
	\xspace%
}
\def\FBAs{%
	Free Boolean Algebras (FBAs)%
	\gdef\FBA{FBA\xspace}%
	\gdef\FBAs{FBAs\xspace}%
	\xspace%
}
%%%%%%%%%%%%%%%%%%% Acronyms (end)

\begin{document}

\begin{frame}[label=title]
%If you want to restart, goto \hyperlink{title}{\beamergotobutton{title}}.
\titlepage
\end{frame}

\section{Introduction}
\begin{frame}
	\frametitle{Introduction}
	\begin{itemize}
		\item Safety is a crucial concern on critical systems
		\item \FTA is a deductive method to assess safety: premisses are leaves (basic events) and conclusions are roots (top events)
		\item Structure expressions represent the structure of \FTs
		\begin{itemize}
			\item \MCSs discover and top-event probability calculus use these expressions
		\end{itemize}
		\item Other kinds of \FTs: \TFT and \DFT
		\begin{itemize}
			\item \SFT is used to distinguish original \FTs from the other kinds of \FTs
		\end{itemize}
	\end{itemize}
\end{frame}

\begin{frame}
	\frametitle{Example of structure expression}
	TODO: colocar figura de árvore e sua expressão de estrutura
\end{frame}

\begin{frame}
	\frametitle{\FTs and structure expressions}
	\begin{itemize}
		\item \SFTs: structure expressions, Shannon method (later BDDs) in the Fault Tree Handbook
		\item \TFTs: Pandora, structure expressions, and the full implementation of the Fault Tree Handbook (Papadopoulos, Walker)
		\begin{itemize}
			\item Operators: PAND, POR, SAND
		\end{itemize}
		\item \DFTs: defined as a visual representation and to improve expressiveness of Markov chains (Boyd and Dugan); structure expressions and calculus of top-event probability for \DFTs shown in the work of Merle
		\begin{itemize}
			\item Operators: CSp, SEQ, FDEP
		\end{itemize}
	\end{itemize}
\end{frame}

\subsection{Research questions}

\begin{frame}
	\frametitle{Mathematical models}
	\begin{itemize}
		\item \SFTs have a mathematical model based on \FBA
		\item There are different mathematical models for \TFTs and \DFTs, none is similar to \FBA
		\item Models for \TFTs and \DFTs are not ready for NOT gates, although it has its importance
	\end{itemize}
\end{frame}

\begin{frame}
	\frametitle{Research questions}
	\begin{itemize}
		\item Is there a mathematical model that unifies the representation of \SFTs, \TFTs and \DFTs?
		\item Can this model represent ordering of events for \TFTs and \DFTs? 
		\item Does this model allow formula reduction to a normal form?
	\end{itemize}
\end{frame}

\subsection{Proposed solution}
\begin{frame}
	\frametitle{Proposed solution overview}
\end{frame}

\section{Background}
\subsection{Systems, dependability, and fault modelling}
\subsection{Time relation of fault events}

\section{Analysis and tools}
\subsection{Fault Tree Analysis}
\subsection{Structure expressions analysis}
\subsection{Free Boolean algebrass}
\subsection{NOT operator in SFTs}
\subsection{Nominal model and fault injection}
\subsection{Isabelle/HOL}

\section{Contributions}
\subsection{A free algebra to express structure expressions}
\subsection{Temporal properties}
\subsection{XBefore laws}
\subsection{Propositions}

\section{Case study}
\subsection{Structure expressions with Boolean operators}
\subsection{Structure expressions with XBefore}

\section{Final remarks}
\subsection{Conclusion}

\begin{frame}
\frametitle{Status}
\end{frame}

\begin{frame}
\frametitle{Next steps in the thesis}
\end{frame}

\begin{frame}
\frametitle{Future work, out of the scope of this thesis}
\end{frame}

%\begin{frame}
%\frametitle{Motivation}
%\end{frame}


\end{document}