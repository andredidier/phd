\documentclass{beamer}

\mode<presentation>
{
\usetheme{CIn20131120}
\setbeamercovered{transparent}
}

%\AtBeginSection[]
%{
%	\begin{frame}<beamer>
%		\frametitle{Outline}
%		\tableofcontents[currentsection,hideothersubsections]
%	\end{frame}
%}

\usepackage[british]{babel}
\usepackage[utf8]{inputenc}
\usepackage[T1]{fontenc} 
    
\title{An Algebra of Temporal Faults}
\subtitle{PhD Qualification}
\date{April 2016}

\author{André Didier}

\institute{
\inst{1}Federal University of Pernambuco\\
Centre of Informatics
}

\subject{Qualification presentation}

\begin{document}

\begin{frame}[label=title]
%If you want to restart, goto \hyperlink{title}{\beamergotobutton{title}}.
\titlepage
\end{frame}

\section{Introduction}
\subsection{Research questions}
\subsection{Proposed solution}

\section{Background}
\subsection{Systems, dependability, and fault modelling}
\subsection{Time relation of fault events}

\section{Analysis and tools}
\subsection{Fault Tree Analysis}
\subsection{Structure expressions analysis}
\subsection{Free Boolean algebrass}
\subsection{NOT operator in SFTs}
\subsection{Nominal model and fault injection}
\subsection{Isabelle/HOL}

\section{Contributions}
\subsection{A free algebra to express structure expressions}
\subsection{Temporal properties}
\subsection{XBefore laws}
\subsection{Propositions}

\section{Case study}
\subsection{Structure expressions with Boolean operators}
\subsection{Structure expressions with XBefore}

\section{Final remarks}
\subsection{Conclusion}

\begin{frame}
\frametitle{Status}
\end{frame}

\begin{frame}
\frametitle{Next steps in the thesis}
\end{frame}

\begin{frame}
\frametitle{Future work, out of the scope of this thesis}
\end{frame}

%\begin{frame}
%\frametitle{Motivation}
%\end{frame}


\end{document}