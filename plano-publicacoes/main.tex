%%This is a very basic article template.
%%There is just one section and two subsections.
\documentclass[a5paper,portuges]{article}
\usepackage{babel}
\usepackage[T1]{fontenc}
\usepackage[utf8]{inputenc}
\usepackage{cleveref}
\usepackage{plano-publicacoes}

\title{Plano de publicações}

\crefname{artigoi}{artigo}{artigos}
\Crefname{artigoi}{Artigo}{Artigos}

\begin{document}

\maketitle

\begin{artigo}
\label{sec:msc-traces-tfa}
\begin{objetivo}
O objetivo deste artigo é modificar o artigo do SBMF sobre o mestrado e acrescentar a relação dos traces de falha com o reticulado de listas booleano. Mostrar a teoria fundamental para representação dos traces de falha em uma álgebra que estende a álgebra booleana, usando o mesmo conceito de FBA.
\end{objetivo}
\begin{fluxo}
Mestrado $\rightarrow$ Traces de falha $\rightarrow$ Álgebra Temporal de Defeitos (TFA)
\end{fluxo}

\begin{paraconseguir}
Artigo do SBMF + álgebra temporal de defeitos
\end{paraconseguir}

\begin{submissao}
  \item FMi'2015 (submetido)
\end{submissao}

\end{artigo}

\begin{artigo}
\label{sec:criterios-aceitacao}
\begin{objetivo}
O objetivo deste artigo é modificar o \cref{sec:msc-traces-tfa} sobre o mestrado e a teoria fundamental dos traces de defeitos. A partir dos traces de defeito, utilizar a TFA para obter os predicados e verificar os critérios de aceitação de uma árvore de falha.
\end{objetivo}
\begin{fluxo}
Mestrado => Traces de falha => TFA => Predicado => Critérios de aceitação
\end{fluxo}
\begin{paraconseguir}
\Cref{sec:msc-traces-tfa} + Predicados e critérios de aceitação (20-25% a mais)
\end{paraconseguir}
\begin{submissao}
  \item Special issue do FMi'2015 (aguardando convite)
\end{submissao}
\end{artigo}

\begin{artigo}
\label{sec:padroes-tolerancia}
\begin{objetivo}
O objetivo principal deste artigo é apresentar a etapa inicial do doutorado através da modelagem de falhas usando a álgebra de ativação e padrões de tolerância a falhas para obter os mesmos resultados do \cref{sec:criterios-aceitacao}. Um objetivo secundário do artigo é apresentar e mostrar que existe um homomorfismo do reticulado de listas booleano para qualquer álgebra booleana, assim como é definido em Isabelle/HOL.
\end{objetivo}
\begin{fluxo}
Álgebra de ativação (F + O) através de padrões => Reticulado de listas booleano => Predicado => Critérios de aceitação

Reticulado de listas booleano => Homomorfismo para qualquer álgebra booleana
\end{fluxo}
\begin{paraconseguir}
\Cref{sec:criterios-aceitacao} + padrões de tolerância a falhas + homomorfismo para qualquer álgebra booleana 
\end{paraconseguir}
\end{artigo}

\begin{artigo}
\label{sec:algebra-merle}
\begin{objetivo}
O objetivo deste artigo é apresentar a ideia central do doutorado mostrando que o reticulado de listas atende aos teoremas da álgebra temporal de Merle. 

Obs.: Convidar Sascha para a escrita.
\end{objetivo}

\begin{fluxo}
Reticulado de listas temporal => Formalização de álgebra temporal de Merle => equivalência do reticulado de listas para a álgebra temporal de Merle
\end{fluxo}

\begin{paraconseguir}
Teoria sobre álgebra temporal de Merle e teoria sobre o reticulado de listas temporal. 
\end{paraconseguir}
\begin{submissao}
  \item Hindawi (falta escrever)
\end{submissao}
\end{artigo}

\begin{artigo}
\end{artigo}

\begin{artigo}
\end{artigo}

\end{document}
