\documentclass[11pt,a4paper]{article}
\usepackage{isabelle,isabellesym}

% this should be the last package used
\usepackage{pdfsetup}

% urls in roman style, theory text in math-similar italics
\urlstyle{rm}
\isabellestyle{it}


\begin{document}

\title{Abstract Completeness}
\author{Jasmin Christian Blanchette, Andrei Popescu, and Dmitriy Traytel}

\maketitle

\begin{abstract}
  This is a formalization of an abstract property of possibly infinite
  derivation trees (modeled by a codatatype), that represents the core of a
  Beth--Hintikka-style proof of the first-order logic completeness theorem and
  is independent of the concrete syntax or inference rules. This work is
  described in detail in a publication by the authors \cite{bla-compl}.

  The abstract proof can be instantiated for a wide range of Gentzen and tableau
  systems as well as various flavors of FOL---e.g., with or without predicates,
  equality, or sorts. Here, we give only a toy example instantiation with
  classical propositional logic. A more serious instance---many-sorted FOL with
  equality---is described elsewhere \cite{bla-mech}.
\end{abstract}

\bibliographystyle{abbrv}
\bibliography{root}

\tableofcontents

% sane default for proof documents
\parindent 0pt\parskip 0.5ex

% generated text of all theories
%
\begin{isabellebody}%
\setisabellecontext{Basic}%
%
\isadelimtheory
%
\endisadelimtheory
%
\isatagtheory
%
\endisatagtheory
{\isafoldtheory}%
%
\isadelimtheory
%
\endisadelimtheory
%
\begin{isamarkuptext}%
From the site\footnote{Accessed 27/jan/2016: \url{https://isabelle.in.tum.de/overview.html}} of the 
creators:
%
\begin{quote}
Isabelle is a generic proof assistant. It allows mathematical formulas to be expressed in a formal 
language and provides tools for proving those formulas in a logical calculus. 
%
The main application 
is the formalization of mathematical proofs and in particular formal verification, which includes 
proving the correctness of computer hardware or software and proving properties of computer 
languages and protocols.
\end{quote}%
\end{isamarkuptext}%
\isamarkuptrue%
%
\begin{isamarkuptext}%
Isabelle/HOL\index{Isabelle/HOL} is the most widespread instance of Isabelle. 
\acs{HOL} stands for \acl{HOL}.
%
Isabelle/HOL provides a \ac{HOL} proving environment ready to use, which include: (co)datatypes, 
inductive definitions, recursive functions, locales, custom syntax definition, etc.
%
Proofs can be written in both human\footnote{By human we mean that anyone with mathematics and logic 
basic knowledge---it means that deep programming knowledge is not essential.} and machine-readable 
language based in \ac{Isar}.
%
The tool also includes the \emph{sledgehammer}, a port to call external first-order provers to find
proofs automatically.
%
The user interface is based in jEdit\footnote{Accessed 27/jan/2016: \url{http://www.jedit.org/}}, 
which provides a text editor, syntax parser, shortcuts, etc (see \cref{fig:basic-symmetry-isabelle-window}).

\begin{figure}[t]
  \centering
  \includegraphicsaspectratio[0.8]{basic-symmetry-isabelle-window}
  \caption{Isabelle/HOL window, showing the basic symmetry theorem}
  \label{fig:basic-symmetry-isabelle-window}
\end{figure}

Theories in Isabelle/HOL are based in a few axioms.
%
Isabelle/HOL Library's theories---that comes with the installer---and user's theories are based on 
these axioms.
%
This design decision avoids inconsistencies and paradoxes.

Besides the provided theories, its active community provides a comprehensive \ac{AFP}.
%
Each entry in this archive can be cited and usually contains an \emph{abstract}, a document, and a 
theory file.
%
For example, \iacl{FBA} theory is available in~\cite{Huffm2010}.
%
To use it, it is enough to download and put on the same directory of your own theory files.

Bellow we show an example and explain the overall syntax of the human and machine-readable language.%
\end{isamarkuptext}%
\isamarkuptrue%
\isacommand{theorem}\isamarkupfalse%
\ basic{\isacharunderscore}symmetry{\isacharcolon}\isanewline
\ \ \isakeyword{assumes}\ {\isachardoublequoteopen}x\ {\isacharequal}\ y{\isachardoublequoteclose}\ %
\isamarkupcmt{Assumptions%
}
\isanewline
\ \ \isakeyword{shows}\ {\isachardoublequoteopen}y\ {\isacharequal}\ x{\isachardoublequoteclose}\ %
\isamarkupcmt{Hypothesis%
}
\isanewline
%
\isadelimproof
%
\endisadelimproof
%
\isatagproof
\isacommand{proof}\isamarkupfalse%
\ {\isacharminus}%
\newcommand{\bsproof}{%
\ \ \isacommand{have}\isamarkupfalse%
\ {\isachardoublequoteopen}x\ {\isacharequal}\ x{\isachardoublequoteclose}\ \isacommand{{\isachardot}{\isachardot}}\isamarkupfalse%
\ %
\isamarkupcmt{Proof step%
}
\isanewline
\ \ \isacommand{from}\isamarkupfalse%
\ assms\ %
\isamarkupcmt{Using assumptions%
}
\ \isanewline
\ \ \ \ \isacommand{show}\isamarkupfalse%
\ {\isachardoublequoteopen}y\ {\isacharequal}\ x{\isachardoublequoteclose}\ \isacommand{{\isachardot}{\isachardot}}\isamarkupfalse%
\ %
\isamarkupcmt{Show thesis%
}
%
}\\\bsproof\\
\isacommand{qed}\isamarkupfalse%
%
\endisatagproof
{\isafoldproof}%
%
\isadelimproof
%
\endisadelimproof
%
\begin{isamarkuptext}%
Finally, Isabelle/HOL provides \LaTeX{} syntax sugar and allow easy document preparation: this 
entire section was written in a theory file mixing Isabelle's and \LaTeX{}'s syntax).
%
The above theorem can be written using Isabelle's quotation and anti-quotations.
%
For example, we can write it using usual \LaTeX{} theorem environment:
%
\begin{Theo}[Basic symmetry]
Assuming \isa{x\ {\isacharequal}\ y}, thus:

\begin{center}
\isa{y\ {\isacharequal}\ x}
\end{center}
\end{Theo}
%
\begin{proof}
\bsproof
\end{proof}

Otherwise specified, in the next sections we will omit proofs because they are all verified using
Isabelle/HOL.%
\end{isamarkuptext}%
\isamarkuptrue%
%
\isadelimtheory
%
\endisadelimtheory
%
\isatagtheory
%
\endisatagtheory
{\isafoldtheory}%
%
\isadelimtheory
%
\endisadelimtheory
\end{isabellebody}%
%%% Local Variables:
%%% mode: latex
%%% TeX-master: "root"
%%% End:


%%% Local Variables:
%%% mode: latex
%%% TeX-master: "root"
%%% End:


\end{document}

%%% Local Variables:
%%% mode: latex
%%% TeX-master: t
%%% End:
