\chapter{Formalisation of Hazard Management}

This chapter formalises our hazard management approach using \CSP.

Hazard properties, extracted from~\cite{EricsonII2005}:
%
\begin{enumerate}
  \item Hazardous element (HE);
  \item Initiating mechanisms (IM): sequence of events to activate the hazard;
  \item Target and Threat (T/T): the person or system and its corresponding threat;
  \item Status: open or closed. A hazard can only be closed if it has been verified through analysis, inspection or testing that the safety requirements are implemented in the design and successfully tested for effectiveness.
  \item Mitigation: lists the recommended actions obtained by all analysis
  \begin{itemize}
    \item Eliminate through design selection;
    \item Incorporate safety devices;
    \item Provide warning devices;
    \item Develop procedures and training;
    \item Control hazard through design methods.
  \end{itemize}
  \item Mishap Risk Index (initial, current, final) -- qualitative measurement:
  \begin{itemize}
    \item Severity
    \begin{enumerate}
      \item Catastrophic;
      \item Critical;
      \item Marginal;
      \item Negligible.
    \end{enumerate}
    \item Probability
    \begin{enumerate}
      \item Frequent;
      \item Probable;
      \item Occasional;
      \item Remote;
      \item Improbable.
    \end{enumerate}
  \end{itemize} 
\end{enumerate}

Scope of system to which Hazard Analysis make sense: those that have signal, material and energy.
%
All system functional elements~\cite[p. 47]{KSS+2011} are:
\begin{description}
  \item[Signal.] Generate, transmit, distribute, and receive signals used in passive or active sensing and in communications;
  \item[Data.]Analyse, interpret, organize, query, and/or convert data and information into forms desired by the user or other systems;
  \item[Material.] Provide system structural support or enclosure, or transform the shape, composition, or location of material substances;
  \item[Energy.] Provide and convert energy or propulsive power to the system.
\end{description}

\begin{definition}[Identifier]
\hmreg{Id}{\mathrm{Id}}
It is modelled as any user-defined data type. It can be a sequence of alphanumeric symbols (a descriptive text of the element) or a natural number (for example: an identification of a system element). It is a basic data type $\hmuse{Id}$ used in other definitions.
\end{definition}

\begin{definition}[Component]
\hmreg{Component}{\mathrm{C}}
\hmreg{subcomponents}{\mathop{\mathrm{subcomponents}}}
A component is any functional or physical block within a system that may contain a hazard. 
%
It is defined as an identifier and a set of identifiers: 
%
\\$\hmuse{Component}: \hmcartesian{\hmuse{Id}}{\hmpowerset{\hmuse{Id}}}$.
%
\\Every subcomponent cannot contain any ancestor.
%
Given a component $C=\left(c_0,CS_0\right)$, the subcomponents set is obtained by a folding operation.
%
\\$\hmsubcomponents{C} = \hmsetcomprehension{\hmunion{\hmsetenum{\hmfirst{C_i}}}{\hmsubcomponents{C_i}}}{C_i \in \hmsecond{C}}$.
%
\\For every component $C$: $\hmforall{C_i \in CS_0}{c_0 \notin \hmsubcomponents{C_i}}$ 
\end{definition}

\begin{definition}[Hazardous element]
A hazardous element represents an element that can potentially harm someone or something.
$\hmdatatypeHE: \hmdatatypeId$.
\end{definition}

%\hmregisterbinop{OPA}{\mathrm{OPA}}
%$\hmop{OPA}{X}{Y}$

\begin{definition}[Initiating mechanism]
An initiating mechanism is an event\footnote{It is not a \CSP event; when we refer to a \CSP event we will always refer to a channel.} on the system.
%
$\hmdatatypeIM: \hmdatatypeId$
%it can be a CSP process. 
\end{definition}

\begin{definition}[Target/Threat]
A target and threat pair represents the person or system that is harmed if a hazardous element is activated by the initiating mechanisms and the nature of the harm. 
%
$\hmdatatypeTT: \hmcartesian{\hmdatatypeId}{\hmdatatypeId}$.
\end{definition}

\begin{definition}[Hazard]
Given a hazardous element $HE$, a sequence of $n$ initiating mechanisms $\hmseqenum{IM_1, \ldots, IM_n}$ and a target-threat pair $T/T$, a hazard $H$ is defined as: $H: \hmdatatypeId$ and is obtained by a total, bijective function $\hmhazardsymbol$: 
%
\\$\hmhazardsymbol: \hmtotalbijectivefunction{\hmcartesian{\hmcartesian{\hmdatatypeHE}{\hmdatatypeTT}}{\hmseq{\hmdatatypeIM}}}{\hmdatatypeId}$
%
\\which is written as:
%
\\$H = \hmhazard{HE}{\hmseqenum{IM_1, \ldots, IM_n}}{T/T}$.
\end{definition}

\begin{definition}[Probability function]
A probability function $\hmprobabilitysymbol$ is used to associate values to elements in the system.
%
It is defined through analysis. 
%
$\hmprobabilitysymbol: \hmpartialfunction{\hmdatatypeId}{\hmdatatypeR}$
\end{definition}

\begin{definition}[Hazard analysis type]
A hazard analysis type is a categorization of the analysis methods as follows:
%
\newcommand{\HATdescr}[1]{\item[\expandafter\csname #1presentation\endcsname.] \expandafter\csname #1expanded\endcsname\csname #1CHECKtrue\endcsname}
%
\begin{description}
  \HATdescr{CDHAT}
  \HATdescr{PDHAT}
  \HATdescr{DDHAT}
  \HATdescr{SDHAT}
  \HATdescr{ODHAT}
  \HATdescr{HDHAT}
  \HATdescr{RDHAT} 
\end{description}
\end{definition}

\begin{definition}[Analysis method]
An analysis method $\hmdatatypeAM$, such that $\hmdatatypeAM: \hmdatatypeId$, describes a method used to analyse an element. 
%
It is defined as an enumerated set with the value correspondence shown in \cref{tbl:analysis-methods}.
%
\newcounter{analysismethods}
\newcommand{\amcount}{\addtocounter{analysismethods}{1}\arabic{analysismethods}}
\newcommand{\amline}[3]{\amcount & \csname #1presentation\endcsname#3 & #2 & \csname #1expanded\endcsname}
%
\begin{table}
\centering
\begin{tabular}{c|l|l|l}
Seq. & Abbrev. & Hazard analysis type & Description\\
\hline
\hline
\amline{PHL}{\CDHAT}{*}\\
\hline
\amline{PHA}{\PDHAT}{*}\\
\hline
\amline{SSHA}{\DDHAT}{*}\\
\hline
\amline{SHA}{\SDHAT}{*}\\
\hline
\amline{OSHA}{\ODHAT}{*}\\
\hline
\amline{HHA}{\HDHAT}{*}\\
\hline
\amline{SRCA}{\RDHAT}{*}\\
\hline
\amline{FTA}{\SDHAT, \DDHAT}{}\\
\hline
\amline{ETA}{\SDHAT}{}\\
\hline
\amline{FMEA}{\DDHAT}{}\\
\hline
\amline{FaHA}{\DDHAT}{}\\
\hline
\amline{FuHA}{\SDHAT, \DDHAT}{}\\
\hline
\amline{SCA}{\SDHAT, \DDHAT}{}\\
\hline
\amline{PNA}{\SDHAT, \DDHAT}{}\\
\hline
\amline{MA}{\SDHAT, \DDHAT}{}\\
\hline
\amline{BA}{\SDHAT}{}\\
\hline
\amline{BPA}{\DDHAT}{}\\
\hline
\amline{HAZOP}{\SDHAT, \DDHAT}{}\\
\hline
\amline{CCA}{\SDHAT, \DDHAT}{}\\
\hline
\amline{CCFA}{\SDHAT, \DDHAT}{}\\
\hline
\amline{MORT}{\SDHAT, \DDHAT}{}\\
\hline
\amline{SWSCA}{}{}\\
\hline
\amline{SWHA}{}{}\\
\hline
\amline{THA}{}{}\\
\hline
\hline 
\end{tabular}
\caption{Analysis methods}
\label{tbl:analysis-methods}
\end{table}
\end{definition}

\begin{definition}[Time]
Time $\hmdatatypetime$ is abstracted with equal and before relations. 
%
It is defined as a sequence of time tags.
%
$\hmdatatypetime: \hmseq{\hmdatatypetimetag}$.
\end{definition}

\begin{definition}[Time tag]
A time tag represents an instantaneous observation of time.
%
It is defined as an identification: $\hmdatatypetimetag: \hmdatatypeId$
\end{definition}

\begin{definition}[Analysis]
Given an analysis method $M$, such that $M: \hmdatatypeAM$, a time tag $t$, such that $t:\hmdatatypetimetag$, an element $E$, such that $E:\hmdatatypeId$, and a probability $p$, such that $p:\hmdatatypeR$, an analysis $A$ is defined as:
%
\\$A: \hmcartesian{\left(\hmcartesian{\hmdatatypeId}{\hmdatatypetimetag}\right)}{\left(\hmcartesian{\hmdatatypeAM}{\hmdatatypeR}\right)}$
%
\\$A = \left(\left(E, t\right), \left(M, p\right) \right)$

\end{definition}
The probability function is then updated accordingly to an analysis. For example, for the analysis above, the probability function $P$ can be updated as:
$\hmprobabilitysymbol = \hmoverride{\hmprobabilitysymbol}{\left\{\hmmapsto{E}{p}\right\}}$.

\begin{definition}[Hazard activation probability]
Given a probability function $\hmprobabilitysymbol$, such that $P: \hmpartialfunction {\hmdatatypeId}{\hmdatatypeR}$, a hazardous element $HE$, a sequence of $n$ initiating mechanisms $\hmseqenum{IM_1,\ldots,IM_n}$ and a target-threat pair $T/T$, the probability of the activation of the hazard $H = \hmhazard{HE}{\hmseqenum{IM_1,\ldots,IM_n}}{T/T}$ is:
%
\\$\hmprobability{H} = \hmtimes{\hmprobability{HE}}{\hmtimes{\hmtimes{\hmprobability{IM_1}}{\ldots}}{\hmprobability{IM_n}}}$
\end{definition}

\begin{definition}[Initiating mechanism]
An initiating mechanism is a relation between components attributes and their unwanted behaviour or state. 
%
Some component attribute examples~\cite{EricsonII2005}:
\begin{description}
  \item[Hardware] Failure modes, hazardous energy sources, $IM_{H1}=\left(C_1,1\right)$, $IM_{H2}=\left(C_1,2\right)$;
  \item[Software] Design errors, design incompatibilities, $IM_{S1}=\left(C_2,3\right)$, $IM_{S2}=\left(C_2,4\right)$
  \item[Personnel] Human error, human injury, human control interface, $IM_{P1}=\left(C_3,3\right)$;
  \item[Environment] Weather, external equipment;
  \item[Procedures] Instructions, tasks, warning notes;
  \item[Interfaces] Erroneous input/output, unexpected complexities;
  \item[Functions] Fail to perform, performs erroneously;
  \item[Facilities] Building faults, storage compatibility, transportation faults;
\end{description}
\end{definition}

By the relations between components and hazards, and hazards and initiating mechanisms, a dependency graph can be built, so if a components changes, the corresponding hazards changes as well.

\begin{definition}[Mishap]
A mishap is defined as a pair of the probability of a hazard and the severity in case it happens: $M: \hmcartesian{\hmdatatypeR}{\hmdatatypeseverity}$.
\end{definition}
%
\noindent Where $\hmdatatypeseverity$ is a finite set, such that $\hmdatatypeseverity \subset \hmdatatypeN$. It contains the following values:
\begin{description}
  \item[1] $=$ Catastrophic;
  \item[2] $=$ Critical;
  \item[3] $=$ Marginal and
  \item[4] $=$ Negligible.
\end{description}

\begin{definition}[Hazard Model]
\hmreg{hazardmodel}{\mathrm{HM}}
A $\hmuse{hazardmodel}$\ldots
\end{definition}

\section{Refinement}

In this section we define a refinement relation on a hazard model of a system.

\TODO{Add definitions on the hazard model
\begin{itemize}
  \item Set of analyses;
  \item Probability function;
  \item Addition of analysis $\implies$ updates probability function;
  \item Set of hazards;
  \item Dependency graph (connection between components);
\end{itemize}
}

There a few situations in which a refinement is desirable:
\begin{description}
  \item[Hazard analysis.] A hazard model is said to be an analysis of another model if at least one new analysis was added.
  \item[Hazard mitigation.] A hazard model is said to be a mitigation of another if at least one hazard is mitigated by decrement of probability value. If $\hmprobabilitysymbol$ is the probability function of the previous model and $\hmprobabilitysymbol'$ is the probability function of the refined model, then: 
%
\\$\hmdom{\hmprobabilitysymbol} \subseteq \hmdom{\hmprobabilitysymbol'} \land%
\hmexists{e \in \hminter{\hmseqenum{set of hazards}}{\hmdom{\hmprobabilitysymbol}}}{%
\hmprobability[\hmprobabilitysymbol']{e} < \hmprobability{e}%
}$.
  \item[Hazard discovery.] A hazard model is said to be a discovery of another if the hazards set is augmented with a new discovered hazard.
  \item[Component replacement.] A valid component replacement in a model is achieved if all analyses for the replaced component are updated with the new component and the probability of activation is less than or equal to the previous ones.
  \item[Component addition.] A valid component addition in a model is achieved if the component is not in any other component hierarchy or, if it is, then the analyses of all affected component hierarchy and related components are updated and the probability of activation is less than or equal to the previous ones.
  \item[Component removal.] A valid component removal in a model is achieved if the analyses of all affected component hierarchy and related components are updated and the probability of activation is less than or equal to the previous ones. 
\end{description}

