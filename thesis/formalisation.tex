\chapter{Formalisation of Hazard Management}

Hazard properties, extracted from~\cite{ericsonII2005}:
%
\begin{enumerate}
  \item Hazardous element (HE);
  \item Initiating mechanisms (IM): sequence of events to activate the hazard;
  \item Target and Threat (T/T): the person or system and its corresponding threat;
  \item Status: open or closed. A hazard can only be closed if it has been verified through analysis, inspection or testing that the safety requirements are implemented in the design and successfully tested for effectiveness.
  \item Mitigation: lists the recommended actions obtained by all analysis
  \begin{itemize}
    \item Eliminate through design selection;
    \item Incorporate safety devices;
    \item Provide warning devices;
    \item Develop procedures and training;
    \item Control hazard through design methods.
  \end{itemize}
  \item Mishap Risk Index (initial, current, final) -- qualitative measurement:
  \begin{itemize}
    \item Severity
    \begin{enumerate}
      \item Catastrophic;
      \item Critical;
      \item Marginal;
      \item Negligible.
    \end{enumerate}
    \item Probability
    \begin{enumerate}
      \item Frequent;
      \item Probable;
      \item Occasional;
      \item Remote;
      \item Improbable.
    \end{enumerate}
  \end{itemize} 
\end{enumerate}

Scope of system to which Hazard Analysis make sense: those that have signal, material and energy.
%
All system functional elements~\cite[p. 47]{KSS+2011} are:
\begin{itemize}
  \item Signal---generate, transmit, distribute, and receive signals used in passive or active sensing and in communications;
  \item Data---analyse, interpret, organize, query, and/or convert data and information into forms desired by the user or other systems;
  \item Material---provide system structural support or enclosure, or transform the shape, composition, or location of material substances;
  \item Energy---provide and convert energy or propulsive power to the system.
\end{itemize}

\begin{definition}[Identifier]
It can be modelled as any user-defined data type. It can be a sequence of alphanumeric symbols (a descriptive text of the element) or a natural number (for example: an identification of a system element). It is a basic data type $\hmdatatypeId$ used in other definitions.
\end{definition}

\begin{definition}[Component]
A component is any functional or physical block within a system that may contain a hazard. $\hmdatatypecomponent: \hmdatatypeId$
\end{definition}

\begin{definition}[Hazardous element]
A hazardous element represents an element that can potentially harm someone or something.
$\hmdatatypeHE: \hmdatatypeId$.
\end{definition}

\begin{definition}[Initiating mechanism]
An initiating mechanism is an event on the system.
%
$\hmdatatypeIM: \hmdatatypeId$
%it can be a CSP process. 
\end{definition}

\begin{definition}[Target/Threat]
A target and threat pair represents the person or system that is harmed if a hazardous element is activated by the initiating mechanisms and the nature of the harm. 
%
$\hmdatatypeTT: \hmcartesian{\hmdatatypeId}{\hmdatatypeId}$.
\end{definition}

\begin{definition}[Hazard]
Given a hazardous element $HE$, a sequence of $n$ initiating mechanisms $\hmseqenum{IM_1, \ldots, IM_n}$ and a target-threat pair $T/T$, a hazard $H$ is defined as: $H: \hmdatatypeId$ and is obtained by a total, bijective function $\hmhazardsymbol$: 
%
\\$\hmhazardsymbol: \hmtotalbijectivefunction{\hmcartesian{\hmcartesian{\hmdatatypeHE}{\hmdatatypeTT}}{\hmseq{\hmdatatypeIM}}}{\hmdatatypeId}$
%
\\which is written as:
%
\\$H = \hmhazard{HE}{\hmseqenum{IM_1, \ldots, IM_n}}{T/T}$.
\end{definition}

\begin{definition}[Probability function]
A probability function $\hmprobabilitysymbol$ is used to associate values to elements in the system.
%
It is defined through analysis. 
%
$\hmprobabilitysymbol: \hmpartialfunction{\hmdatatypeId}{\hmdatatypeR}$
\end{definition}

\begin{definition}[Analysis method]
An analysis method $\hmdatatypeAM$, such that $\hmdatatypeAM: \hmdatatypeId$, describes a method used to analyse an element. 
%
It is defined as an enumerated set with the following value correspondence:
%
\begin{description}
  \item[PHL] \PHL;
  \item[PHA] \PHA;
  \item[\ldots] \TODO{Add others.}
\end{description}
\end{definition}

\begin{definition}[Time]
Time $\hmdatatypetime$ is abstracted with equal and before relations. 
%
It is defined as a sequence of time tags.
%
$\hmdatatypetime: \hmseq{\hmdatatypetimetag}$.
\end{definition}

\begin{definition}[Time tag]
A time tag represents an instantaneous observation of time.
%
It is defined as an identification: $\hmdatatypetimetag: \hmdatatypeId$
\end{definition}

\begin{definition}[Analysis]
Given an analysis method $M$, such that $M: \hmdatatypeAM$, a time tag $t$, such that $t:\hmdatatypetimetag$, an element $E$, such that $E:\hmdatatypeId$, and a probability $p$, such that $p:\hmdatatypeR$, an analysis $A$ is defined as:
%
\\$A: \hmcartesian{\left(\hmcartesian{\hmdatatypeId}{\hmdatatypetimetag}\right)}{\left(\hmcartesian{\hmdatatypeAM}{\hmdatatypeR}\right)}$
%
\\$A = \left(\left(E, t\right), \left(M, p\right) \right)$

\end{definition}
The probability function is then updated accordingly to an analysis. For example, for the analysis above, the probability function $P$ can be updated as:
$\hmprobabilitysymbol = \hmoverride{\hmprobabilitysymbol}{\left\{\hmmapsto{E}{p}\right\}}$.

\begin{definition}[Hazard activation probability]
Given a probability function $\hmprobabilitysymbol$, such that $P: \hmpartialfunction {\hmdatatypeId}{\hmdatatypeR}$, a hazardous element $HE$, a sequence of $n$ initiating mechanisms $\hmseqenum{IM_1,\ldots,IM_n}$ and a target-threat pair $T/T$, the probability of the activation of the hazard $H = \hmhazard{HE}{\hmseqenum{IM_1,\ldots,IM_n}}{T/T}$ is:
%
\\$\hmprobability{H} = \hmtimes{\hmprobability{HE}}{\hmtimes{\hmtimes{\hmprobability{IM_1}}{\ldots}}{\hmprobability{IM_n}}}$
\end{definition}

\begin{definition}[Mishap]
A mishap is defined as a pair of the probability of a hazard and the severity in case it happens: $M: \hmcartesian{\hmdatatypeR}{\hmdatatypeseverity}$.
\end{definition}
%
\noindent Where $\hmdatatypeseverity$ is a finite set, such that $\hmdatatypeseverity \subset \hmdatatypeN$. It contains the following values:
\begin{description}
  \item[1] $=$ Catastrophic;
  \item[2] $=$ Critical;
  \item[3] $=$ Marginal and
  \item[4] $=$ Negligible.
\end{description}

\section{Refinement}

In this section we define a refinement relation on a hazard model of a system.

\TODO{Add definitions on the hazard model
\begin{itemize}
  \item Set of analysis;
  \item Probability function;
  \item Addition of analysis updates probability function;
\end{itemize}
}

There a few situations in which a refinement is desirable:
\begin{description}
  \item[Hazard analysis.] A hazard model is said to be an analysis of another model if at least one new analysis was added.
  \item[Hazard mitigation.] A hazard model is said to be a mitigation of another if at least one hazard is mitigated through probability decrease. If $\hmprobabilitysymbol$ is the probability function of the previous model and $\hmprobabilitysymbol'$ is the probability function of the refined model, then: 
%
$\hmforall{e \in \hmdatatypeId}{\hmprobability[\hmprobabilitysymbol']{e}\leq\hmprobability{e}}$.
\end{description}

