\chapter{Introduction}
\label{chap:intro}

The development process of critical control systems requires the rigorous execution of guides and regulations~\cite{ANAC2011,FAA1993,FAA2007,SAE1996b}.
Specialized agencies (like FAA, EASA and ANAC in the aviation field) use these guides and regulations to certify such systems.
Only upon certification such systems can be used in the real-world.

Safety is a property (measured both qualitative and quantitatively) of crucial concern on critical systems and it is the responsibility of the safety assessment process.
To employ such a process, dependable systems' taxonomy and safety assessment techniques must be well defined and understood.
Clarification of concepts of dependable systems can be surprisingly difficult when systems are complex, because the determination of possible causes or consequences of failures can be a very subtle process~\cite{ALR+2004}.

ARP-4761~\cite{SAE1996b} defines several techniques to perform safety assessment.
One of them is \ac{FTA}\index{Fault Tree Analysis}.
It is a deductive method that uses trees to model faults and their dependencies and propagation.
In such trees, the premises are the leaves (basic events) and the conclusions are the roots (top events).
Intermediate events use gates to combine basic events and each kind of gate has its own combination semantics definition.
\Acp{FT} that use only \ac{OR} and \ac{AND} gates are called \emph{coherent \aclp{FT}}~\cite{Andrews2001,AB2003,Oliv2006,CCR2008,Vaurio2016}\index{Fault Tree!coherent}.
They combine events as \emph{at least one shall occur} and \emph{all shall occur}, respectively.
To analyse \acp{FT}, their structures are abstracted as Boolean\index{Boolean Algebra} expressions called \emph{structure expressions}\index{structure expression}.
The analysis of coherent \acp{FT} uses a well-defined algorithm based on the Shannon's method\index{Binary Decision Diagrams!Shannon's method} to obtain \acp{MCS} from the structure expressions\index{structure expression}, and a general formula to calculate the probability of top events.
The \acp{MCS} are obtained by reducing structure expressions to a normal form, in which each term is a combination of variables (basic events) with conjunctive (\ac{AND}) gates, and the terms are combined by disjunctive (\ac{OR}) gates.
These minimal terms are also called \emph{prime implicants} or \emph{minterms}.
The Shannon's method\index{Binary Decision Diagrams!Shannon's method} originated a formalism to reduce structure expressions called \ac{BDD}\index{Binary Decision Diagrams}.
Another approach to reduce structure expressions is to use a mathematical model---called \acf{FBA}---that uses sets of sets to represent Boolean expressions.

Although structure expressions are formulas with logical operators, they are formalisms to enable automatic \ac{FTA}.
As shown in~\cite{Ericson2005}, \acp{FT} are a much richer model than structure expressions alone, enabling a visual indication of fault paths, and include description of subsystems as intermediate events.
Redundancy may be present in \acp{FT}, but not usually in structure expressions.
%% Este parágrafo foi incluído após discussão de Paulo sobre avaliação de confiabilidade/disponibilidade.

\Cref{fig:strategy-overview-traditional} shows how \ac{FTA} is traditionally performed.
It starts with an architectural model, then faults are identified and modelled in \iac{FT}.
System requirements are identified and are checked with \ac{FTA} results.
If the requirements are satisfied (accepted), the process ends and the modelled system may be implemented.
Otherwise, fault tolerance patterns are used, adding or modifying the original architecture to improve dependability.
The analyses are executed until system requirements are met.
We call such system requirements of \acg{FT} \emph{acceptance criteria}.

\begin{figure}[htb]
  \centering
  \includegraphicsaspectratio[0.55]{StrategyOverview-traditional-path}
  \caption{Traditional \ac{FTA}}
  \label{fig:strategy-overview-traditional}
\end{figure}

Besides the traditional \ac{OR} and \ac{AND} gates, the \FThandbook defines other gates as well.
For example the \ac{PAND} gate, which considers the order of occurrence of events.
Although the \FThandbook defines new gates, there is no algorithm to perform the analysis of trees that contain such new gates.
This absence together with the need to analyse dynamic aspects of increasingly complex systems motivated the introduction of two new kinds of \aclp{FT}: \acp{DFT} and \acp{TFT}.
These variant trees can capture sequential dependencies of fault events in a system.
The difference from \ac{TFT} to \ac{DFT} is that \acp{TFT} use temporal gates directly, while \ac{DFT} does not---\acp{DFT} gates are an abstraction of temporal gates.
To differentiate the \aclp{FT} as defined in the \FThandbook from the other two, we will call the former as \acp{SFT}.

The work reported in~\cite{WP2009} aims at performing the full implementation of the \FThandbook, adding temporal gates to its \pandora methodology.
It was this implementation that introduced the new concept of \acp{TFT}, cited previously.
In such trees, events ordering is well-defined and an algebraic framework was proposed to reduce structure expressions\index{structure expression} to obtain \acp{MCSeq} and perform probabilistic analysis.
Reducing expressions is also desirable to check for tautologies, for example.

\Acp{DFT} introduce very different gates to capture dynamic configurations of systems.
The main gates are: \ac{CSp}, \ac{FDEP}, and \ac{SEQ}.
The semantics of the first is to add ``backup'' events, so the gate is active if the primary event and all spares are active.
The second adds basic events' dependency from a trigger event.
The third forces the occurrence of events in a particular order.
There is also \iac{WSp} gate that is slightly different from the \ac{CSp} gate.
They differ on the nature of sparing, whether fast (warm, always-on) or slow (cold, stand-by).
The readiness of the backup system in \iac{WSp} gate is higher than in \iac{CSp} gate.
The work reported in~\cite{MRL2011} shows an algebraic framework to compositionally reduce \ac{DFT} gates to order-based gates and perform probabilistic analysis of structure expressions\index{structure expression}. Thus, despite some limitations related to spare gates~\cite{MRL2014}, the structure expressions\index{structure expression} used in \acp{TFT} and \acp{DFT} can be formulated in terms of a generic order-based operator.

\begin{sloppypar}
The \ac{NOT} operator is absent in the algebras reported in~\cite{WP2009,Walker2009,Merle2010,MRL2011b} because,
if it is used without restrictions, it can be misleading, generating non-coherent analysis~\cite{Oliv2006}.
Although such an issue may arise, it can be essential in practical use as demonstrated in~\cite{Andrews2001} with algebraic laws to handle the operator in structure expressions.
%
Our concern is that the decision of the relevance of its use should be neither due to the choice of events-occurrence representation, as it is in~\cite{WP2009,Walker2009,Merle2010,MRL2011b}, nor with algebraic laws to include missing terms as it is in~\cite{Andrews2001}.
The algebra created in this work defines the \ac{NOT} operator such that it can be used without any restriction (freely), as we show in \cref{chap:algebra}.
\end{sloppypar}

\begin{sloppypar}
\Ac{hiphops} is a set of methods and tools to analyse \acp{FT}.
The semi-automatic generation of \acp{FT} has architectural models and failure expressions as inputs.
\emph{The failure expressions are in fact structure expressions of components or subsystems.}
These expressions are annotated in components and subsystems and describe how they fail.
The tool combines these expressions with regard to the architecture of the system to generate \acp{FT}.
The work reported in~\cite{WP2008} shows a strategy to use the semi-automatic \ac{FT} generation of \ac{hiphops} with \pandora to generate structure expressions of \acp{TFT}.
\end{sloppypar}

\Ac{AADL} is a standard language to model (among other features) system structure and component interaction. 
\Ac{AADL} has several tools to perform different analyses to obtain \ac{SFT} to perform \ac{FTA}.
But \acpg{AADL} assertions framework does not express order explicitly as needed for \ac{TFT} and \ac{DFT} analyses.

In previous work~\cite{Didier2012,DM2012}, we proposed a systematic hardware-based faults identification strategy to obtain failure expressions as defined in \acifused{hiphops}{\acs{hiphops}}{\acscite{hiphops}} for \acp{SFT}.
%
We considered faults in components or subsystems to obtain structure expressions and use them as input for \ac{hiphops}.
If, instead, we obtain failure expressions of a whole system, they are in fact structure expressions\index{structure expression} of \iac{FT}.
%
Our previous strategy throws away the ordering information of the fault event sequences to generate failure expressions for components or subsystems for \acp{SFT}.
%
%Using our strategy as input for \ac{hiphops} we obtain a failure expression of a fault tree.
%
We focused on hardware faults because we assume that software does not fail as a function of time (wear, corrosion, etc).
%
We inherited this view from \embraer, which assumes that functional behaviour is completely analysed by functional verification~\cite{SP2011}.
%
We followed industry common practices using \simulink diagrams~\cite{Nise1992} as a starting point.
%
The work reported in~\cite{DM2012} was based on \ac{CSPm} to allow an automatic analysis using the model checker \acs{FDR}.
%
Thus, our strategy required the translation from \simulink to \ac{CSPm}~\cite{JMS+2011}.
%
It then runs \acs{FDR} to obtain several counter-examples (which are fault traces) ending in failures.
%
For two case studies provided by \embraer we showed that our automatically created failure expressions match with the engineer's provided ones or are better because consider additional fault occurrence combinations.

\section{Mathematical models}
\label{sec:mathematical-models}

Both \ac{TFT} and \ac{DFT} lack a first-order logic mathematical model like the one defined for \ac{SFT}.
For \acp{SFT}, mathematical models to reduce structure expressions are either based on set inclusion, with \ac{FBA}, or through tree search, with \ac{BDD}\index{Binary Decision Diagrams}.
One important concern on employing \ac{FTA} is whether \iac{FT} indeed represents a system's operational mode.
The work reported in~\cite{MCS+1999} exposes this concern for \acp{DFT}, and the \ac{hiphops} framework---related to \acp{SFT} and \acp{TFT}---aims at getting this issue sorted out.
Our contribution to this issue for \ac{SFT} is shown in~\cite{DM2012,Didier2012}.

The mathematical model for \ac{TFT} has a discontinuity between states.
The transition from the non-occurrence to an occurrence some time later is different from the occurrence of one event before another one.
Such a discontinuity has some drawbacks as, for example, the impossibility to use \ac{NOT} gates, and handling the specific case of non-occurrence with zeros in \acp{TTT}.
The reduction of structure expressions in \ac{TFT} is based on a combination of:
\begin{alineasinline}
  \item algebraic rewriting---which can unfortunately result in an infinite application of rules,
  \item modularisation of independent subtrees (subtrees not always are independent), and
  \item \ac{DT}---which are limited to seven basic events, due to exponential growth.
\end{alineasinline}


\begin{sloppypar}
Most mathematical models~\cite{LHT2013,CSD2000,BRM+2005} for \ac{DFT} are based on the formalisation of \ac{DTMC} or \ac{CTMC} because \acp{DFT} were initially conceived to be a visual representation of such models.
As both \ac{DTMC} and \ac{CTMC} are state-based, they experience the state-space explosion problem.
The works reported in~\cite{BKK+2003,BHH+2003,SAE1996b} show techniques to overcome the state-explosion problem.
%, but the reduction can be infeasible because it depends on systems' models particularly, whether they are independent or not.
\end{sloppypar}

There are other approaches, as well.
For instance, a modified version of \ac{BDD}\index{Binary Decision Diagrams} to tackle events ordering, called \acf{SBDD}\index{Sequential Binary Decision Diagrams}\index{Binary Decision Diagrams!Sequential}, that can reduce structure expressions, and the work reported in \cite{BRM+2005}, which proposes a conversion of \ac{DFT} into \ac{DBN} to perform probabilistic analysis.

The approach to tackle events ordering with \ac{SBDD}\index{Sequential Binary Decision Diagrams}\index{Binary Decision Diagrams!Sequential}~\cite{XTD2012} has two kinds of nodes: terminals and non-terminals (terminals are nodes with basic events, and non-terminals are nodes with two events and an operator).
Although demonstrated in~\cite{Bryant1986} that these unconventional nodes (non-terminals) generate correct and efficient Boolean analysis, the analysis is still dependent on the order-related operators because the relation of terminals and non-terminals is not established directly (non-terminals are seen as an independent node in~\cite{XTD2012}).
For example, the occurrence of $A \rightarrow B$ is related to the occurrence of $A$ and then $B$, but this relation is obtained in a further step, not in the \ac{SBDD}\index{Sequential Binary Decision Diagrams}\index{Binary Decision Diagrams!Sequential}.

The approach using the construction of \acp{DBN}~\cite{BRM+2005} is automatic and handles time slices as $t + \Delta t$, which implies a notion of events ordering as well.
As it is focused in probabilistic analysis, qualitative analysis is not directly supported.

The works reported in~\cite{Merle2010,XTD2012} show that \acg{DFT} operators can be converted into order-related operators, simplifying \ac{DFT} analysis.
Although the mathematical model presented in~\cite{Merle2010} establishes a denotational semantics for order-related operators, it lacks a formal method for expression reduction based on such a model.
It defines, instead, several algebraic laws to reduce expressions and an algorithm to minimize the structure function.

\begin{sloppypar}
\Ac{hiphops} proposes a hierarchical approach to model systems and perform \ac{FTA} (and \ac{FMEA}).
Although there is a tool to model and analyse systems using \ac{hiphops}, \acp{FT} construction is based on an algorithm.%, without proofs for soundness or completeness.
\end{sloppypar}

%The work reported in~\cite{AH2015} shows the formalisation of probabilistic analysis of \ac{SFT} and uses the same concept of date-of-occurrence shown in \adnote{cite Merle work}.

\begin{sloppypar}
Another concern, left untreated in the literature, is the undesirable possibility of non-determinism in system analyses.
For example, \iac{FT} to analyse a signal omission has the structure expression $A \land B$.
Another \ac{FT} to analyse a commission has the structure expression $\left(A \land B\right) \lor C$.
In this example, if faults $A$ and $B$ are active, then either an omission or a commission is observed for the system.
\end{sloppypar}

\section{Research questions}
\label{sec:research-questions}

From the exposed in this \lcnamecref{sec:mathematical-models}, our research questions are:
\begin{rqenum}[series=researchquestion]
  \item Is there a consistent mathematical model to analyse \acp{TFT} and \acp{DFT} that is set-based and similar to \ac{FBA}?\label{question:mathematical-model}
  \item What guarantees can we provide to detect non-determinism in erroneous behaviour?\label{question:non-determinism}
\end{rqenum}
%
Also, does such a model:
%
\begin{rqenum}[resume*=researchquestion]
  %\item have the capacity of representing events ordering similar to \ac{TFT} and \ac{DFT}\label{question:ordering-representation}?
  \item represent systems behaviour by construction\label{question:gap}?
  \item allow both qualitative and quantitative analyses as supported by \ac{TFT} and \ac{DFT}\label{question:analyses}?
  %\item perform reduction of structure expressions to a normal form at least as efficiently as current approaches\label{question:efficiency}?
\end{rqenum}

%
%In this work we propose a theory that answers \cref{question:mathematical-model,question:ordering-representation,question:gap,question:analyses}. 
%\Cref{question:efficiency} is left as a future work.

\section{Proposed solution}

%In this work we present an algebra, called \acf{algebra}, defined  denota
In this work we present an algebra, called \acf{algebra}, to express ordering of fault events (\ac{TFT} and \ac{DFT}), enabling analysis of	 acceptance criteria of \acp{FT}.
The laws of \ac{algebra} are proven in a denotational semantics based on sets of lists of distinct elements.
\Theac{algebra} aims at answering the \cref{question:mathematical-model}.
The analysis of acceptance criteria is a decision problem and we use first-order logic and \isabellehol as verification tool.
%Indeed, \ac{algebra} is part of a bigger strategy that relates fault injection on nominal models, fault modelling, \ac{FTA}, and fault tolerance patterns.

System and fault modelling is an essential step towards safety analysis.
Architectural modelling is the first step of the strategy and can be executed either in a graphical tool, or as requirements in natural language.
For example, our work reported in~\cite{APR+2013,ADP+2013} uses fault modelling in the \ac{SysML} to verify fault tolerance of \acp{SoS}.

Writing and analysing expressions with order-related operators is more complex than analysing expressions with Boolean\index{Boolean} operators only.
We propose a logic, called \ac{activation}, which works together with an accompanied (attached) algebra to perform analysis of system structure and component interaction with a focus on fault modelling and fault propagation, tackling the complexity introduced by order-related operators.
\ac{activation} receives an algebra and the set of operational modes of a system as parameters.
The choice of algebra defines which structure expressions can be obtained: if Boolean algebra is passed as a parameter, \ac{activation} can generate structure expressions with Boolean operators (\ac{SFT}); if \ac{algebra} is passed as a parameter, \ac{activation} can generate structure expressions with order-related operators (\ac{TFT} and \ac{DFT}).
\Ac{activation} requires that the accompanied algebras satisfy a set of properties (tautology and contradiction) and semantic values.
The use of the \ac{NOT} is essential: besides its use in expressions, we use the complement of structure expressions, normalizing them and making them \emph{healthy}.

To obtain critical event expressions used in \acp{FT} and to denote faults propagation, the \ac{activation} provides a predicate notation and verification of non-determinism. 
We show three different approaches to check non-determinism and answer \cref{question:non-determinism}: 
\begin{alineasinline}
  \item verify its existence, 
  \item indicate which set of operational modes are active for a combination of faults, or 
  \item which combination of faults activates a set of operational modes.
\end{alineasinline}

In our proposed solution, depending on the easiness to identify the faults, the analyst may follow one of the paths: 
\begin{alineasinline}
  \item model the system in \simulink to allow fault injection and discovery, or 
  \item model faults using the \acl{activation}.
\end{alineasinline}
%
Both paths end with structure expressions and the \ac{FTA} is performed using \ac{algebra}.

\Cref{fig:strategy-overview-csp} shows how to perform \ac{FTA} using fault injection.
The ``Faults injection'' block is obtained from part of our work reported in~\cite{DM2012,Didier2012}.
It starts with \simulink modelling, converts the model to \ac{CSPm} and then obtains fault event sequences (also called fault traces).
The fault event sequences are then mapped to \theac{algebra}, which has a denotational semantics based on sets of lists.
This strategy aims at answering the \cref{question:gap}.

Safety requirements are stated in terms of critical failures such as, for example, ``the probability of a complete failure of an airplane engine should be less than $10^{-9}$'' (quantitative), or ``a complete failure of the propulsion system should not be caused by a single failure'' (qualitative).
%In this work we call \emph{acceptance criteria} the verification of a safety requirement on \iac{FT}, where \acg{FT} top event is the undesired failure.
Positive requirements such as, for example, ``the communication system should be operational $99.99\%$ of the cruise phase'' are treated as a complement (the complete failure should have a probability in less than $0.01\%$ of the cruise phase).
The acceptance criteria analysis aims at answering the \ref{question:analyses}.

From the model in \ac{algebra} (\cref{fig:strategy-overview-csp}), the acceptance criteria are then verified.
If the criteria are accepted, the process finishes.
Otherwise, the system is modified, and the process continues, modifying the system architecture, using fault tolerance patterns, improving the system dependability.
%
\begin{figure}[htb]
  \centering
  \includegraphicsaspectratio[0.8]{StrategyOverview-csp-path}
  \caption{Faults injection and \ac{algebra} to perform \ac{FTA}}
  \label{fig:strategy-overview-csp}
\end{figure}

\Cref{fig:strategy-overview-activation} shows a fault modelling strategy directly in the \ac{activation}.
The \ac{activation} associates each operational mode with a fault expression.
After modelling all faults, the top events are extracted in a predicate notation. 
For example, ``is the behaviour of the system in the operational mode $X$?'', where $X$ can be an omission, commission, etc.
Given the flexibility of the \ac{activation} notation, it can be used to reason about basic fault events and top-event failures, which are related to~\ref{question:mathematical-model}.
Each predicate in \ac{activation} generates an expression in \theac{algebra}, which is reduced to obtain a normal form to obtain \acp{MCSeq} and to calculate top-events probability.
With the system modelled in \ac{activation}, the fault tolerance patterns can be applied directly on the model.
%
\begin{figure}[htb]
  \centering
  \includegraphicsaspectratio[0.8]{StrategyOverview-activation-path}
  \caption{\Ac{activation} and \ac{algebra} to perform \ac{FTA}}
  \label{fig:strategy-overview-activation}
\end{figure}

The complete proposed solution is summarized in \cref{fig:strategy-overview}, joining the paths described in \cref{fig:strategy-overview-csp} and \cref{fig:strategy-overview-activation} (paths A and B, respectively).
%
\begin{figure}[htb]
  \centering
  \includegraphicsaspectratio[1]{StrategyOverview}
  \caption{Strategy overview}
  \label{fig:strategy-overview}
\end{figure}

\section{Contributions}

The main contributions of this work are:

\begin{contrenum}[series=contributions]
  \item Define a denotational model and an algebra to express fault events order with \theac{algebra} (\cref{chap:algebra});
  %%%%%
  \item Define a new operator to express order explicitly and proving that the resulting algebra---(\ac{algebra}) using this operator and Boolean\index{Boolean Algebra} operators---is a conservative extension of the Boolean algebra\index{Boolean Algebra} (also published in~\cite{DM2016})---see \cref{chap:algebra};
    %%%%%
  \item Map sequences of fault events into \ac{algebra} (\cref{chap:algebra});
    %%%%%
  \item Reason about fault modelling in \ac{activation} to obtain formal expressions of critical failures (top-event failures, \cref{chap:activation});
    %%%%%
  \item Illustrate both \ac{algebra} and \ac{activation} on a real case study, provided by \embraer (\cref{chap:case-study}), and on a literature case study.
  %\item Generalise laws in~\ac{algebra} in terms of abstract properties, similar to healthiness conditions in \theac{UTP}---to do.
  %\item Formally verify \acg{FT} (\ac{TFT} and \ac{DFT}) acceptance criteria (\cref{chap:acceptance});
\end{contrenum}

We use \isabellehol, theories in \isabellehol{'s} library, and a theory in the AFP library \cite{JM2005} to prove the theorems of \cref{chap:algebra}.

The case studies cover the following scenarios, presented in \cref{chap:case-study}:
\begin{enumerate}
  \item From a model in \simulink, obtain the failure expression of a critical failure, analyse the ordering relation of fault events, and verify its acceptance criteria;
  \item Given a set of \ac{FT} structure expressions, verify which fault combinations analysis are missing;
  \item Perform a probabilistic analysis in \iac{FT} with an explicit \ac{NOT} operator.
\end{enumerate}

\section{Thesis organization}

This thesis is organized as follows: in \cref{chap:basic-concepts,chap:analysis} we show the concepts and tools used as basis for this work.
\Cref{chap:algebra} presents \theac{algebra}, \cref{chap:activation} presents \theac{activation}, \cref{chap:case-study} the case study and the application of the proposed strategy, and we present our conclusions and future work in \cref{sec:conclusion}.
The contributions presented in \cref{chap:algebra} are summarized in terms of proved results.
To facilitate the understanding of the presented strategy, the effort to build laws and theirs (mechanized) proofs are shown in \cref{app:formal-proofs-isabelle-hol}.

\isabellehol{'s} theory files with all proofs are available at \algebraurl.