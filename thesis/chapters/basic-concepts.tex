\chapter{Basic concepts}
\label{chap:basic-concepts}

Means to dependability\index{Dependability}\amnote{Está muito filosófico isto. É assim mesmo ou seja, não há uma definição para dependebilidade?} \adnote{Eu explico dependabilidade a seguir.} are obtained by modelling and analysing a system.
It is strongly related to fault modelling, which depends on the kinds of analyses we want to perform.
%For instance, in \aclp{FT}, even if a fault can be repaired, it is considered as a non-repairable fault.
\Acp{FT} are present in several stages of systems' modelling.
We introduce dependability\index{Dependability} and fault modelling in \cref{sec:dependability}.

\Iac{SFT} is a snapshot\footnote{Whether a top event indeed causes a catastrophic or major failure is out of the scope of this thesis; we consider that, if it is possible that such failure occur, then it will.} of a faults topology of a system, subsystem or component.
%\acp{TFT} and \acp{DFT} uses sequences of fault events because they consider a time relation on fault events.
The time relation of fault events in \acp{TFT} and \acp{DFT} allows the analysis of different configurations (or snapshots) of a system, subsystem or component.
We discuss these time relations in \cref{sec:time-relations}.

\section{Systems, dependability\index{Dependability}, and fault modelling\index{fault modelling}}
\label{sec:dependability}

\begin{sloppypar}
Computing systems are characterized by five properties: functionality, performance, cost, dependability\index{Dependability}, security.
The work reported in~\cite[p. 289--302]{Sommerville2011} explain these properties---including dependability\index{Dependability}---with a focus on software.
Hardware and software are connected, as software faults may cause a failure in a software-controlled hardware, and hardware faults may send incorrect data, causing a failure in the software.
\end{sloppypar}

The work reported in~\cite{ALR+2004} summarizes all concepts of and related to dependability\index{Dependability} for computing systems that contain software and hardware.
In the following, we show these concepts and highlight those used in this work.

\subsection{Systems}

Before introducing systems' dependability\index{Dependability}, we first describe what a system is and its characteristics.
A \emph{system} is an entity that interacts with other systems (software and hardware as well), users (humans), and the physical world.
These other entities are the \emph{environment} of the given system, and its \emph{boundary} is the frontier between the system and its environment.

The \emph{function} of a system is what the system is intended to do, and its \emph{behaviour} is what the system does to implement its function.\amnote{Você continua muito filosófico aqui. Quero ver para onde vamos (uso)...} \adnote{Isso faz parte do contexto para quando mencionar os termos eles terem sido introduzidos.}
The \emph{total state} of a system are the means to implement its function and is defined as the set of the following states: computation, communication, stored information, interconnection, and physical condition.
The \emph{service} delivered by a system is its behaviour as it is perceived by its boundary.
A system can both provide and consume services.

The \emph{structure} of a system is how it is composed: a system is composed of components, and each component is another system, etc.
This concept of hierarchical compositionality in systems is what originated the concept of \ac{SoS} and is the object of analysis in \ac{hiphops}.
Such a recursion (of a system containing other systems) stops when a component---or a constituent system---is considered to be atomic.
A system is the total state of its atomic components.

\subsection{Dependability}
\index{Dependability}

The concepts that create the basis for dependability\index{Dependability} are:
\begin{alineasinline}
  \item threats to,
  \item attributes of, and
  \item means to attain.
\end{alineasinline}

\emph{Threats to dependability}\index{Dependability!threats to} are the so-called \emph{fault-error-failure} chain.
A failure is a service deviation perceived on systems' boundary.
An error is the part of the total state of a system that leads to subsequent service failure.
Depending on how a system tolerate internal errors, many errors may not reach system's boundary.
Finally, a fault is what causes an error.
In this case, we say that the fault \emph{occurred} (the fault is active).
Otherwise, the fault is dormant, and has not occurred (yet).
A \emph{degraded} mode of a system is when there are active faults, so some functions of the system are inoperative, but the system still delivers its service.

There are two acceptable definitions of dependability\index{Dependability} reported in~\cite{ALR+2004}.
One is more general, difficult to measure: ``the ability to deliver service that can justifiably be \emph{trusted}''.
A more precise definition that uses the definition of service failure is: ``the ability to avoid service failures that are more frequent and more severe than is acceptable''.
This definition has two implications about system's requirements: there should be defined how it can fail, and what are the acceptable severity and frequency of its failures.

The following systems' dependability attributes\index{Dependability!attributes of} enlightens such requirements:
\begin{description}
  \item[Availability:] the readiness for correct service;
  \item[Reliability:] continuity of correct service;
  \item[Safety:] absence of catastrophic consequences on the environment (other systems, users, and the physical world).
  Safety can be verified using \acp{FT}, which is part of the objective of this work;
  \item[Integrity:] absence of improper systems alterations;
  \item[Maintainability:] ability to be modified and repaired.
\end{description}
%
A system description should mention all or most of these attributes, at least the first three of them.

The implementation of these attributes requires a deep analysis of system's models.
The \emph{means to attain dependability}\index{Dependability!means to attain} are summarized as follows:

\begin{description}
  \item[Prevention] is about avoiding incorporating faults during development.
  \item[Tolerance] deals with usage of mechanisms to still deliver a---possibly degraded---service even in the presence of faults.
  \item[Removal] is about detecting and removing (or reducing severity of) failures from a system, both in the development and production stages.
  \item[Forecasting] is about predicting likely faults so they can removed, or tackling their effects.
\end{description}

\begin{sloppypar}
The intersection of the current work with dependability\index{Dependability} is in fault removal during development and fault tolerance (analysis).
Following the taxonomy presented in~\cite{ALR+2004}, there are some techniques for fault removal, summarized as follows:
\begin{alineas}
  \item Static verification:
  \begin{subalineas}
    \item Structural model:
    \begin{description}
      \item[Static analysis:] Range from inspection or walk-through, data flow analysis, complexity analysis, abstract interpretation, compiler checks, vulnerability search, etc.
      \item[Theorem proving:] Check properties of infinite state models.
    \end{description}
    \item Behaviour model:
    \begin{description}
      \item[Model checking:] Usually the model is a finite state-transition model (\acp{PN}, finite state automata).
      Model-checking verifies all possible states on a given system's model.
    \end{description}
  \end{subalineas}
  \item Dynamic verification:
  \begin{subalineas}
    \item Symbolic inputs:
    \begin{description}
      \item[Symbolic Execution:] It is the execution with respect to variables (symbols) as inputs.
    \end{description}
    \item Actual inputs:
    \begin{description}
      \item[Testing:] Selected input values are set on system's inputs and their outputs are compared to expected values.
      The verification outcomes are observed faults, in case of hardware testing or software mutation testing, and criteria-based, in case of software testing.
    \end{description}
  \end{subalineas}
\end{alineas}
\end{sloppypar}


Verification methods are often used in combination.
For example, symbolic execution may be used to obtain testing patterns, test inputs can be obtained by model-checking as in~\cite{CBC+2015}, faults can be used as symbolic inputs, and system behaviour can be observed using model-checking as in~\cite{DM2012,Didier2012} (This technique is called fault injection; see also~\cite{AAL+1996}).

The techniques to attain fault tolerance are summarized as follows:
\begin{description}
  \item[Error detection:] is used to identify the presence of an error.
  It can be a concurrent or a preemptive detection.
  Concurrent detection takes place during normal service, while preemptive detection takes place while normal service is suspended.
  \item[Recovery:] transforms a system state that contains errors into a state without them. The behaviour of the system upon recovery is equivalent to the normal behaviour.
  Techniques range from rollback to a previously saved state without errors, error masking, isolation of faulty components, to reconfiguration using spare components.
\end{description}

In this work, we use a combination of:
\begin{alineasinline}
  \item fault-injection,
  \item theorem proving, and
  \item symbolic execution.
\end{alineasinline}
We use these methods to obtain an erroneous behaviour of the system which is compared to the system dependability attributes\index{Dependability!attributes of} (safety).
We explain how these methods are combined in \cref{chap:algebra}.

On the analyses of systems and its constituents, there is a distinction of operational modes and error events.
Operational modes refer to the behaviour that is perceived on the boundaries of a system (\emph{failure}).
Error events, on the other hand, represent the behaviour detected in a constituent of a system.
Such error events may relate to an operational mode, but not necessarily.
Further in \cref{chap:algebra} we abstract these differences and leave the distinction as a parameter.
We refer to such a set as \emph{operational modes}.

\subsection{Fault Modelling}

Fault modelling plays an important role in reasoning about the fault-error-failure chain.
They are the initial steps to perform the verification of a system, starting in the architectural model to reason about the critical failures, which are (in general) the top-events in \acp{FT}.

\Ac{SysML} is a profile for \ac{UML} that provides features to model structure and behaviour of systems.
The works reported in~\cite{APR+2013,ADP+2013} define several structural and behavioural views in \ac{SysML} to model the fault-error-failure chain and fault tolerance.
Fault, error, failures, and fault propagation have structural views, which are related to behavioural views to describe fault activation and recovery.
These works map \ac{SysML} to two formal languages---\ac{CML} and \ac{CSP}, respectively---to verify fault tolerance.

In~\cite{SAE1996b} the safety assessment process for civil airborne systems and equipment describes development cycles and methods to ``clearly identify each failure condition''.
The methods that involve failure identification are:
\begin{alineasinline}
  \item \ac{SFT},
  \item \ac{DD},
  \item \aca{DTMC}, and
  \item \ac{FMEA}.
\end{alineasinline}
The first three are top-down methods, that start with undesired failure conditions and moves to lower levels to obtain more detailed conditions that causes the top-level event.
\Acp{DD} are an alternative method of representing the data in \ac{SFT}.
\Ac{FMEA} is a bottom-up method that identifies failure modes of a component and determines the effects on the upper level.
We detail \ac{SFT} in \cref{sec:static-fault-trees}.

\Acp{DFT} are an extension of \acp{SFT} and models dynamic behaviour of system faults.
Similarly to the relation of \acp{SFT} and \acp{DD}, the work reported in~\cite{DP2009} demonstrates the relation of \acp{DFT} to \acp{DRBD}.
As the models (\ac{DFT} and \ac{DRBD}) are equivalent, this work sticks to \ac{DFT} due to the amount of work already published.
We detail \acp{DFT} in \cref{sec:dynamic-fault-trees}.

\section{Time relation of fault events}
\label{sec:time-relations}

The most general case for time relations is to consider that each fault event has a continuous time duration.
They are the basis on how fault events discretisation are defined.
The point of view in this work is the analysis of the effects caused by a combination of faults in a snapshot of a system state.
In \cref{fig:time-relations} we show all possibilities of events relations in a continuous time line (from A to B; the converse relation is similar):

\begin{enumerate}\renewcommand{\theenumi}{\alph{enumi}}
  \item A starts before and ends after B has started, but before B has ended;
  \item A starts before B and ends after B has ended (A contains B);
  \item B starts after A, but they end at the same time;
  \item A and B start at the same time, but A ends before B;
  \item A and B start and end at the same time;
  \item A starts before B and ends when B starts.
  \item A starts and ends before B starts;
\end{enumerate}

Considering that fault occurrence corresponds to the start of a fault event and its duration, from \cref{fig:time-relations} we clearly identify which event comes first: A comes before B, except in the cases (d) and (e), where they start exactly at the same time.
If fault events are independent (they are not susceptible to have a common cause) then the probability of their occurrences starting at the same time is very low.
The relations (f) and (g) shows the case that the system was repaired, thus A is not active when B starts.
In \cref{sec:strategy} we abstract the relation of events in continuous time as an \emph{exclusive before} relation, based on fault \emph{occurrence} (it is similar---at least implicitly---to what is reported in~\cite{WP2009,MRL2011}).

\begin{figure}[htb]
  \centering
  \includegraphicsaspectratio[0.5]{time-relations}
  \caption{Relation of two events with duration}
  \label{fig:time-relations}
\end{figure}

