\section{Structure expression analysis}
\label{sec:structure-expression-analysis}

In this \lcnamecref{sec:structure-expression-analysis} we explain the difference of stateful and stateless methods to analyse structure expressions and detail the stateless methods.
A common approach to analyse \iac{FT} is to perform structure expression analysis based on algebraic laws.
Boolean\index{Boolean Algebra} laws are well-known and are used for \acp{SFT}, temporal laws~\cite{Walker2009,WP2010} for \acp{TFT}, and the works reported in~\cite{Merle2010,MRL2011} show laws for \acp{DFT}.
An issue with algebraic laws is that, in some cases, the expression needs to be expanded before it gets rewritten.
So, automation with rewriting is not trivial.
For example, the following \acg{TFT} structure expression needs to be expanded \cite{WP2010} before it gets reduced via rewriting:
\begin{align*}
  & \parsin{X \land Y} \lor \parsin{\parsin{X \pand Y} \land Z} &\\
  = & \parsin{X \pand Y} \lor \parsin{X \sand Y} \lor \parsin{Y \pand X} \lor \parsin{\parsin{X \pand Y} \land Z} & \text{by \cref{law:tft-conjunctive-completion-law}}\\
  = & \parsin{X \pand Y} \lor \parsin{X \sand Y} \lor \parsin{Y \pand X} & \text{by Boolean absorption of } X \pand Y\\
  = & X \land Y & \text{by \cref{law:tft-conjunctive-completion-law}}
\end{align*}

A denotational semantics to Boolean\index{Boolean Algebra} expressions---and consequently to \ac{SFT}---is given by \acp{FBA} (\cref{sec:fba}).
Using denotational semantics, instead of axiomatic laws, avoids possible inconsistencies: in an axiomatic presentation, the axioms must be very simple or must be generally accepted, because a subtlety in an axiom (an unfounded axiom or conflicting axioms) may invalidate a whole theory.
Ideally, one would like to have an axiomatic presentation derived from a denotational one, as advocated, for example, in \cite{HH1998}.

There are several works with stateful analysis methods for \acp{FT} (\ac{SFT}, \ac{TFT}, and \ac{DFT}).
We show some of them in \cref{sec:ft-stateful-analysis-methods}.

\subsection{Stateful methods and temporal logic analysis}
\label{sec:ft-stateful-analysis-methods}

The work reported in~\cite{STR2002} shows a formal approach to analyse \ac{SFT} using \ac{ITL}.
Instead of tackling basic events ordering (as in \ac{PAND}), it considers a causal relation over a gate, as for example, a relation of a basic event and a higher-level intermediate event.

For \acp{TFT}, the works reported in~\cite{MPW2010,MWP2012} show an inverse solution.
They map \acp{FSM} to \pandora logic, then verify system properties.
They show that such a mapping simplifies expression reduction, thus improving performance on the analysis.

Although there is formal modelling approaches to \acp{DFT}, they do not implement a direct modelling of the \ac{DFT} itself.
Instead, most of the works propose a formal modelling using a state-based approach.
The work reported in~\cite{CSD2000} shows a formal model of \acap{DTMC} in \theac{Z} and each \ac{DFT} element (basic events and gates).
The analysis uses a quantifier on states of the resulting \aca{DTMC} automaton.
The work reported in~\cite{GD1997} shows a methodology to perform a modular analysis of \acp{DFT} based on \ac{BDD}\index{Binary Decision Diagrams} and \aca{DTMC}.
As \ac{DFT} extends \ac{SFT}, it identifies subtrees that are purely \ac{SFT} and uses \ac{BDD}\index{Binary Decision Diagrams}, otherwise.
It performs \aca{DTMC} analysis.
Still on the state-based approaches, the work reported in~\cite{SLD2011} maps \acp{DFT} to \ac{HLPN} to analyse false alarms.

%The work reported in~\cite{LR1998} uses a formal approach using \ac{FSM} to generate fault trees automatically from system models.

In the following we show specific stateless methods that are designed to reduce structure expressions.
In essence, the methods are very similar.
Structure expressions for \acp{SFT} can be reduced using \acp{BDD} (\cref{sec:bdd}), \acp{TFT} can be reduced using \acp{DT} (\cref{sec:dependency-trees}), \acp{MCSeq} of \acp{DFT} can be obtained using \ac{ZBDD} (\cref{sec:zbdd}), and the works reported in~\cite{TXD2011,XTD2012} show the analysis of standby systems (\ac{CSp} gates) using \acp{SBDD}\index{Sequential Binary Decision Diagrams}\index{Binary Decision Diagrams!Sequential} (\cref{sec:sbdd}).

\subsection[BDD]{\Aclp*{BDD}\index{Binary Decision Diagrams}}
\label{sec:bdd}

\Acp{BDD}\index{Binary Decision Diagrams} are directed acyclic graphs that represent a Boolean\index{Boolean Algebra} expression.
They are still referred to as \ac{BDD}\index{Binary Decision Diagrams}, but the widespread version is the \ac{ROBDD}, which is an optimisation.
There are two ways to generate \iac{BDD}\index{Binary Decision Diagrams} for an expression:
\begin{alineasinline}
  \item derive a diagram from the truth-table, or
  \item expand the paths based on Shannon's method\index{Binary Decision Diagrams!Shannon's method} (described in the \FThandbook).
\end{alineasinline}

To demonstrate the expressiveness of \iac{BDD}, \cref{fig:bdd-diagram-for-a-truth-table} shows a diagram for a truth table with three variables (\cref{tbl:bdd-truth-table-with-three-variable}).
In a node, when a path is chosen, the variable of the node assumes the edge value.
For example, the top-level node variable of \cref{fig:bdd-diagram-for-a-truth-table} is $A$.
Following the right-hand side of the node, all leaf nodes correspond to the lines of the truth table that $A$ has ``0'' values (the first four lines).
The symbol nodes are replaced by the values assumed by a specific formula.

\begin{figure}[htb]
  \centering
  \includegraphicsaspectratio[0.5]{bdd-diagram-for-a-truth-table}
  \caption{A diagram for a truth table}
  \label{fig:bdd-diagram-for-a-truth-table}
\end{figure}

\begin{table}[t]
  \caption{Truth table for a formula outputs with three variables (A, B, and C)}
  \label{tbl:bdd-truth-table-with-three-variable}
  \centering
  {\footnotesize
  \begin{tabular}{cccc}
    \hline\noalign{\smallskip}
    \textbf{A} & \textbf{B} & \textbf{C} & \textbf{Formula}\\
    \hline\noalign{\smallskip}\hline\noalign{\smallskip}
    0 & 0 & 0 & a \\
    0 & 0 & 1 & b \\
    0 & 1 & 0 & c \\
    0 & 1 & 1 & d \\
    1 & 0 & 0 & e \\
    1 & 0 & 1 & f \\
    1 & 1 & 0 & g \\
    1 & 1 & 1 & h \\
    \hline\noalign{\smallskip}
  \end{tabular}
  }
\end{table}

Following Shannon's method\index{Binary Decision Diagrams!Shannon's method}, we choose a variable and build the lower level \ac{BDD}\index{Binary Decision Diagrams} assuming the edge value for the chosen variable.
In the remainder of the path, the variable's value is unchanged.
For example, the expression $A \lor \parsin{\lnot B \land C}$ has value ``0'' in the lines $a$, $c$ and $d$, and value ``1'' in the other lines.
By choosing the variable $A$ first, then $B$ and $C$, the resulting \ac{BDD}\index{Binary Decision Diagrams} with the binary values nodes (called sink nodes ``false'' and ``true'') for this formula is depicted in \cref{fig:bdd-diagram-for-example-expression}.
Starting from the top-level node $A$, the formula expressed by the \ac{BDD}\index{Binary Decision Diagrams} is true if $A$ assumes value true.
If $A$ is false, and $B$ is false, the expression is only true if $C$ is true.

\begin{figure}[htb]
  \centering
  \includegraphicsaspectratio[0.25]{bdd-diagram-for-example-expression}
  \caption{\Iacs{BDD} for the expression $A \lor \parsin{\lnot B \land C}$}
  \label{fig:bdd-diagram-for-example-expression}
\end{figure}

\Cref{fig:bdd-diagram-for-example-expression} is \iac{ROBDD}.
To be considered \iac{ROBDD}, the \ac{BDD}\index{Binary Decision Diagrams} must meet the following constraints~\cite{BRB1990}:
%
\begin{alineas}
  \item the variables are assigned a constant ordering;
  \item every path to sink nodes visit the input variables in ascending order;
  \item each node represents a distinct logic function.
\end{alineas}
%
For a given expression, the size of \acp{BDD} and \acp{ROBDD} depends on the chosen variables ordering.
The work reported in~\cite{Rudell1993} shows initial findings on best variable ordering, and the work reported in~\cite{KH2014} shows heuristics to improve the performance for optimal order search.

For \acp{SFT} the evaluation of \iac{BDD} is the calculation of the probability of the paths ending in \emph{true}.
For example, the probability of the expression in \cref{fig:bdd-diagram-for-example-expression} is obtained from the formula: $\probability{A \lor \parsin{\lnot A \land \lnot B \land C}}$. Note that the formula in the probability calculation is different from the formula that originated the diagram.

\subsection[DT]{\Acl*{DT}}
\label{sec:dependency-trees}

\Acf{DT} is a hierarchical acyclic graph of expressions that shows all possible cut sequences for any given set of events.
It is a graphical view of a \ac{TTT}.
At the top of \iac{DT} are the variables, that is, the single events that occur in an expression.
On the lower levels are the increasingly complex expressions.
Each node represents \iac{MCSeq}.
\Cref{fig:simple-dependency-tree} shows \iac{DT} with all nodes for variables $X$ and $Y$.

\begin{figure}[htb]
  \centering
  \includegraphicsaspectratio[0.5]{simple-dependency-tree}
  \caption{\acs*{DT} for variables $X$ and $Y$}
  \label{fig:simple-dependency-tree}
\end{figure}

The reduction of a structure expression is given by the activation (true values) of nodes.
If a node is active (true), then all child nodes are also active; the converse is also true: if all node's children are active, then it is also active.
The reduced expression is given by the \ac{DNF} created with the expressions of higher active level nodes.
To reduce the formula $\parsin{X \land Y} \lor \parsin{\parsin{X \pand Y} \land Z}$, given on the beginning of this \lcnamecref{sec:dependency-trees}, we create the \ac{DT} depicted in \cref{fig:dependency-tree-reduction}.
Nodes marked with ``1'' are those \acp{MCSeq} given directly by the formula.
Nodes marked with ``2'' are child nodes of the ``1'''s nodes, and so forth.
The node of the expression $\parsin{\parsin{X \pand Y} \land Z}$ is a grandchild of $X \land Y$ and thus it is not necessary.
The final expression is obtained by the active higher level node, which is $X \land Y$.

\begin{figure}[htb]
  \centering
  \includegraphicsaspectratio[0.7]{dependency-tree-reduction}
  \caption{\acs*{DT} for the formula $\parsin{X \land Y} \lor \parsin{\parsin{X \pand Y} \land Z}$}
  \label{fig:dependency-tree-reduction}
\end{figure}

\subsection[ZBDD]{\Aclp*{ZBDD}\index{Zero-suppressed Binary Decision Diagrams}}
\label{sec:zbdd}

The work reported in~\cite{TD2004} proposes \aclp{ZBDD}, which is a variant of \ac{BDD}\index{Binary Decision Diagrams}, and uses set manipulations (union, intersection, difference, and product) to obtain \acp{MCSeq} of \acp{DFT}.

To reduce \iac{BDD} to \iac{ZBDD}, the nodes that have the ``true'' (`1') path pointing to the ``false'' (`0') sink node are removed, and the parent node is connected directly to the ``false'' subgraph of the removed node.
\Cref{fig:zbdd-example} shows an example of \ac{ZBDD} for the combination set $\setsin{a,b}$, as shown in~\cite{TD2004}.
The idea of the reduction is to remove irrelevant variables and nodes.
The irrelevant variables are set to ``false''.
The method described in~\cite{TD2004} obtains the \acp{MCSeq} by navigating the paths to sink node ``true''.

\begin{figure}[htb]
  \centering
  \includegraphicsaspectratio[0.7]{bdd-zbdd-example}
  \caption{\acs*{ZBDD} example of combination set $\setsin{a,b}$}
  \label{fig:zbdd-example}
\end{figure}

Although the work reported in~\cite{TD2004} shows \theac{ZBDD}, the final solution does not use them directly. 
The idea is to transform the \ac{DFT} into \iac{SFT}, in a very similar way as the one shown in~\cite{TXD2011}.
The order-related operators in \iac{DFT} are replaced by a new event, which takes ordering into account.
Finally, the \acp{MCSeq} are obtained using set manipulation with elements that are basic events alone or order-related operators.
These order-related operators are event-to-event only, so they cannot be combined with other sets.

The use of sets in~\cite{TD2004} is very related to our \ac{algebra}.
We use sets of sequences to define the \ac{algebra}, but keep the analysis with set operators.
In \ac{algebra} we do not create new events that represent an order-related operator.
Our order-related operator has a set-based semantics that can be combined with other non-order-related (Boolean\index{Boolean Algebra}) operators.

\subsection[SBDD]{\Aclp*{SBDD}\index{Sequential Binary Decision Diagrams}\index{Binary Decision Diagrams!Sequential}}
\label{sec:sbdd}

\Ac{SBDD}\index{Sequential Binary Decision Diagrams}\index{Binary Decision Diagrams!Sequential} is an extension of \ac{BDD}\index{Binary Decision Diagrams} to tackle ordering of events in \acp{DFT} for \ac{CSp} and \ac{WSp} gates.
Ordering of events in \ac{CSp} gates~\cite{XTD2012} is slightly different from \ac{WSp}~\cite{TXD2011}.
A backup system in \ac{CSp} gets activated slower than in \ac{WSp}, which implies that there are less failure possibilities in \ac{CSp}, but its the readiness is lower than in \ac{WSp}.
\Ac{SBDD}\index{Sequential Binary Decision Diagrams}\index{Binary Decision Diagrams!Sequential} adds a new kind of node that contains a binary operation of fault events, which allows to express the ordering of events.
One kind of operation expresses the slowness of the relation of the fault events of \ac{CSp}, and another one expresses the readiness of the \ac{WSp}.
The latter semantics is similar to the semantics of \ac{PAND} and \ac{IBefore} (combined with \ac{AND}) gates.

\Ac{SBDD}\index{Sequential Binary Decision Diagrams}\index{Binary Decision Diagrams!Sequential} creation has two steps:
\begin{alineasinline}
  \item \ac{CSp} or \ac{WSp} \ac{DFT} conversion, and \label{item:sbdd-creation-spare}
  \item \ac{SBDD}\index{Sequential Binary Decision Diagrams}\index{Binary Decision Diagrams!Sequential} model generation.\label{item:sbdd-model-generation}
\end{alineasinline}
In \ref{item:sbdd-creation-spare}, it is a \ac{DFT}-to-\ac{DFT} conversion.
\Ac{CSp} and \ac{WSp} gates are converted to a new, but equivalent \ac{DFT} without \ac{CSp} and \ac{WSp} gates, where the operations appear as basic events and are combined using other gates.
In \ref{item:sbdd-model-generation}, the \ac{SBDD}\index{Sequential Binary Decision Diagrams}\index{Binary Decision Diagrams!Sequential} model is created.
The model may contain nodes that are contradictory as, for example, nodes that assumes that an event $A$ is false and a binary operation with \ac{AND} semantics that contains $A$ is true.
This step ends when all contradictions are removed.
The evaluation is similar to \acg{BDD}: each path ending in true is a minimal term in the \ac{DNF} that may contain one of the binary operations and individual events.


