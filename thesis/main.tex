%%This is a very basic article template.
%%There is just one section and two subsections.
\newif\ifdraft
\drafttrue
\documentclass[en,twoside,onehalfspacing,phd]{risethesis}
\usepackage[english]{babel}
\usepackage[utf8]{inputenc}
\usepackage[T1]{fontenc}
\usepackage{colortbl}
\usepackage{color}
\usepackage{graphicx}
\usepackage{amsmath}
\usepackage{cspsymb}
\usepackage[table]{xcolor}
\usepackage{microtype}
\usepackage{bibentry}
\usepackage{subfigure}
\usepackage{multirow}
\usepackage{rotating}
\usepackage{booktabs}
\usepackage{pdfpages}
\usepackage{caption}
\usepackage{lipsum}
\DeclareGraphicsExtensions{.png,.jpg,.eps}
\graphicspath{{./figures},{./}}
\def\mytitle{[TBD]}
\def\me{Andr\'{e} Lu\'{i}s Ribeiro Didier}
\usepackage[linkcolor=black,
            citecolor=blue,
            urlcolor=black,
            colorlinks,
            pdfpagelabels,
            pdftitle={\mytitle},
            pdfauthor={\me}]{hyperref}
\usepackage{cleveref}
\usepackage{etoolbox}
\usepackage{hazard-management}
\usepackage{amsthm}

\captionsetup[table]{position=top,justification=centering,width=.85\textwidth,labelfont=bf,font=small}
\captionsetup[lstlisting]{position=top,justification=centering,width=.85\textwidth,labelfont=bf,font=small}
\captionsetup[figure]{position=bottom,justification=centering,width=.85\textwidth,labelfont=bf,font=small}

%% Remove colors from hyperlinks
\makeatletter
\AtBeginDocument{%
  \renewcommand*{\AC@hyperlink}[2]{%
    \begingroup
      \hypersetup{hidelinks}%
      \hyperlink{#1}{#2}%
    \endgroup
  }%
}
\makeatother

\address{Recife}

\universitypt{Universidade Federal de Pernambuco}
\universityen{Federal University of Pernambuco}

\departmentpt{Centro de Informática}
\departmenten{Center of Informatics}

\programpt{Pós-graduação em Ciência da Computação}
\programen{Graduate in Computer Science}

\majorfieldpt{Ciência da Computação}
\majorfielden{Computer Science}

\title{\mytitle}
\date{April 2014}

\author{\me}
\adviser{Alexandre Cabral Mota}
\coadviser{Alexander Romanovsky}

\newtheorem{definition}{Definition}
\newtheorem{lemma}{Lemma}

\ifdraft
\includeonly{temporal-fault-trees}
%\includeonly{formalisation}
\fi
%%%%% Isabelle/HOL
%\newcommand{\DefineSnippet}[2]{%
%   \expandafter\newcommand\csname snippet--#1\endcsname{%
%     \begin{quote}
%     \begin{isabelle}
%     #2
%     \end{isabelle}
%     \end{quote}}}
%\newcommand{\Snippet}[1]{\csname snippet--#1\endcsname}

%\input{fault-modelling-theory/generated/snippets}

%%%%% Temporal gates
\newcount\pgateexp
\newcommand{\gateexp}[1]{\ifnum\the\pgateexp > 1 (#1) \else #1 \fi}

\newcommand{\OR}[2]{\advance\pgateexp 1\ensuremath{\gateexp{#1+#2}}\advance\pgateexp -1}
\newcommand{\AND}[2]{\advance\pgateexp 1\ensuremath{\gateexp{#1.#2}}\advance\pgateexp -1}
\newcommand{\POR}[2]{\advance\pgateexp 1\ensuremath{\gateexp{#1|#2}}\advance\pgateexp -1}
\newcommand{\PAND}[2]{\advance\pgateexp 1\ensuremath{\gateexp{#1<#2}}\advance\pgateexp -1}
\newcommand{\SAND}[2]{\advance\pgateexp 1\ensuremath{\gateexp{#1\&#2}}\advance\pgateexp -1}
\newcommand{\SV}{\ensuremath{\mathrm{SV}}}
\newcommand{\SVtuple}[1][n]{\ensuremath{\underbrace{\SV \cross \ldots \cross \SV}_{#1}}}

\newcommand{\powerset}{\ensuremath{\mathcal P}}

%%%%% Boolean gates
\newcommand{\BOR}[2]{#1\mathrm{+^B}#2}



%%%% Abbreviations

%TODO
\newcommand{\TODO}[1]{{\color{red}\relsize{+1}TODO: #1}}

\newcommand{\Nat}{\ensuremath{\mathbb{N}}}
\newcommand{\Real}{\ensuremath{\mathbb{R}}}

%%%%%%%%%%%%%%%%%%%%%%%%%%%%%%%%%%%%%%%%%%%%%%%%%%%%%%%%%%%%%%%%%%%%%%%%%%%%%%%%
% Fault and system modelling %%%%%%%%%%%%%%%%%%%%%%%%%%%%%%%%%%%%%%%%%%%%%%%%%%%
%%%%%%%%%%%%%%%%%%%%%%%%%%%%%%%%%%%%%%%%%%%%%%%%%%%%%%%%%%%%%%%%%%%%%%%%%%%%%%%%
\newcommand{\Nominal}[1]{\mathrm{N}.#1}
\newcommand{\Omission}{\mathrm{Omission}}
\newcommand{\Values}{\ensuremath{\mathrm{Values}}}
\newcommand{\ValueFunction}{\ensuremath{\mathrm{ValueFunction}}}
\newcommand{\FMode}{\ensuremath{\mathrm{FMode}}}
\newcommand{\MIn}{\ensuremath{\mathrm{M_{inputs}}}}
\newcommand{\MOut}{\ensuremath{\mathrm{M_{outputs}}}}
\newcommand{\cin}[2]{\ensuremath{\mathrm{in}_{#1,#2}}}
\newcommand{\cout}[2]{\ensuremath{\mathrm{out}_{#1,#2}}}
\newcommand{\PortIndex}{\ensuremath{\mathrm{PortIndex}}}
\newcommand{\Undefined}{\ensuremath{\mathrm{Undefined}}}
%\DeclareMathOperator*{\Concat}{\Gamma}
\newcommand{\VTValue}{\ensuremath{\rho}}


\newcommand{\newabbrevdescr}[2]{%
\expandafter\newif\csname if#1CHECK\endcsname%
\expandafter\newcommand\csname #1presentation\endcsname{#1\xspace}%
\expandafter\newcommand\csname #1expanded\endcsname{#2\global\csname#1CHECKtrue\endcsname\xspace}%
\expandafter\newcommand\csname #1\endcsname{%
  \csname if#1CHECK\endcsname%
  \csname #1presentation\endcsname%
  \else
  \csname #1expanded\endcsname (\csname #1presentation\endcsname)%
  \fi
}%
}

\newcommand{\renewabbrevdescr}[2]{%
\expandafter\newif\csname if#1CHECK\endcsname%
\expandafter\newcommand\csname #1presentation\endcsname{#1}%
\expandafter\newcommand\csname #1expanded\endcsname{#2\global\csname #1CHECKtrue\endcsname\xspace}%
\expandafter\renewcommand\csname #1\endcsname{%
  \csname if#1CHECK\endcsname%
  \csname #1presentation\endcsname\xspace
  \else
  \csname #1expanded\endcsname
  \fi
}%
}

\newcommand{\newabbrevdescrext}[3]{%
\expandafter\newcommand\csname #2\endcsname{%
  \csname if#1CHECK\endcsname%
  #2\xspace
  \else
  #3\xspace\expandafter\csname #1CHECKtrue\endcsname
  \fi
}%
}

\newabbrevdescr{CSP}{Communicating Sequential Processes}
\renewabbrevdescr{CSPM}{Machine-readable version of \CSP}
\renewcommand\CSPMpresentation{\CSP{$_M$}}

\newabbrevdescr{TTT}{Temporal Truth Table}
\newabbrevdescrext{TTT}{TTTs}{Temporal Truth Tables}

\newabbrevdescr{TFT}{Temporal Fault Tree}
\newabbrevdescrext{TFT}{TFTs}{Temporal Fault Trees}

\newabbrevdescr{FT}{Fault Tree}
\newabbrevdescrext{FT}{FTs}{Fault Trees}

\newabbrevdescr{CDHAT}{Conceptual design hazard analysis type}
\renewcommand{\CDHATpresentation}{CD-HAT}
\newabbrevdescr{PDHAT}{Preliminary design hazard analysis type}
\renewcommand{\PDHATpresentation}{PD-HAT}
\newabbrevdescr{DDHAT}{Detailed design hazard analysis type}
\renewcommand{\DDHATpresentation}{DD-HAT}
\newabbrevdescr{SDHAT}{System design hazard analysis type}
\renewcommand{\SDHATpresentation}{SD-HAT}
\newabbrevdescr{ODHAT}{Operations design hazard analysis type}
\renewcommand{\ODHATpresentation}{OD-HAT}
\newabbrevdescr{HDHAT}{Health design hazard analysis type}
\renewcommand{\HDHATpresentation}{HD-HAT}
\newabbrevdescr{RDHAT}{Requirements design hazard analysis type}
\renewcommand{\RDHATpresentation}{RD-HAT}

\newabbrevdescr{PHL}{Preliminary Hazard List}
\newabbrevdescr{PHA}{Preliminary Hazard Analysis}
\newabbrevdescr{SSHA}{Subsystem Hazard Analysis}
\newabbrevdescr{SHA}{System Hazard Analysis}
\newabbrevdescr{OSHA}{Operating and Support Hazard Analysis}
\renewcommand{\OSHApresentation}{O\&SHA}
\newabbrevdescr{HHA}{Health Hazard Assessment}
\newabbrevdescr{SRCA}{Safety Requirements/Criteria Analysis}
\newabbrevdescr{FTA}{Fault Tree Analysis}
\newabbrevdescr{ETA}{Event Tree Analysis}
\newabbrevdescr{FMEA}{Failure Mode and Effects Analysis}
\newabbrevdescr{FaHA}{Fault Hazard Analysis}
\newabbrevdescr{FuHA}{Functional Hazard Analysis}
\newabbrevdescr{SCA}{Sneak Circuit Analysis}
\newabbrevdescr{SWSCA}{}
\newabbrevdescr{PNA}{Petri Net Analysis}
\newabbrevdescr{MA}{Markov Analysis}
\newabbrevdescr{BA}{Barrier Analysis}
\newabbrevdescr{HAZOP}{Hazard and Operability Analysis}
\newabbrevdescr{CCA}{Cause--Consequence Analysis}
\newabbrevdescr{CCFA}{Common Cause Failure Analysis}
\newabbrevdescr{MORT}{Management Oversight Risk Tree Analysis}
\newabbrevdescr{SWHA}{Software Safety Assessment}
\newabbrevdescr{BPA}{Bent Pin Analysis}
\newabbrevdescr{THA}{Threat Hazard Assessment}


\begin{document}

\frontmatter
\frontpage

\presentationpage

\ifdraft\else
\begin{fichacatalografica}
  \FakeFichaCatalografica % Comment this line when you have the correct file
%     \includepdf{ficha_catalografica.pdf} % Uncomment this
\end{fichacatalografica}
\banca
\begin{dedicatory}
I dedicate this thesis to Juliana and Luciana.
\end{dedicatory}

\acknowledgements
I would like to thank to the following people:
\begin{enumerate}
  \item Alexandre
  \item Sascha
  \item Augusto Sampaio
  \item Zoe
  \item John Fitz.
\end{enumerate}
\ldots

\begin{epigraph}[title]{author}
[Text]\\
\vspace{0.5cm}
[More text\ldots]
\end{epigraph}
\fi

\resumo
Resumo\ldots
\begin{keywords}
\end{keywords}

\abstract
Abstract\ldots
\begin{keywords}
\end{keywords}


\ifdraft\else
\listoffigures
\listoftables
% Acronyms manual: http://linorg.usp.br/CTAN/macros/latex/contrib/acronym/acronym.pdf
\listofacronyms
%
\ExplSyntaxOn
% this now only works because I've use the same already in the preamble so
% it does nothing here:
\ProvideAcroEnding {possessive} {'s} {'s}
\ProvideAcroCommand \acg
{
	\acro_possessive:
	\acro_use:n {#1}
}
\ProvideAcroCommand \Acg
{
	\acro_possessive:
	\acro_first_upper:
	\acro_use:n {#1}
}
\NewAcroCommand \acpg
{
	\acro_possessive:
	\acro_plural:
	\acro_use:n {#1}
}
\NewAcroCommand \Acpg
{
	\acro_possessive:
	\acro_plural:
	\acro_first_upper:
	\acro_use:n {#1}
}

\NewAcroCommand \acsg
{
	\acro_possessive:
	\acro_short:n {#1}
}

\NewAcroCommand \Acsg
{
	\acro_possessive:
	\acro_first_upper:
	\acro_short:n {#1}
}
\NewAcroCommand \acspg
{
	\acro_possessive:
	\acro_plural:
	\acro_short:n {#1}
}

\NewAcroCommand \theac
{
	\acro_if_acronym_used:nTF {#1}{}{the\xspace}
	\acro_use:n {#1}
}

\NewAcroCommand \Theac
{
	\acro_if_acronym_used:nTF {#1}{}{The\xspace}
	\acro_use:n {#1}
}

\NewAcroCommand \workaroundac
{
	##2
	\todo{Issue-workaround-#1}
}


\NewAcroCommand \accite
{
	\acro_cite:
	\acro_use:n {#1}
}


\NewAcroCommand \acpcite
{
	\acro_cite:
	\acro_plural
	\acro_use:n {#1}
}

\NewAcroCommand \Acpcite
{
	\acro_cite:
	\acro_plural:
	\acro_first_upper:
	\acro_use:n {#1}
}


\NewAcroCommand \acscite
{
	\acro_cite:
	\acro_short:n {#1}
}

\ExplSyntaxOff

\DeclareAcronym{algebra}{%
	short=ATF,
	long=Algebra of Temporal Faults,
	short-indefinite=an,
	long-indefinite=an
}

\DeclareAcronym{activation}{%
	short=AL,
	long=Activation Logic,
	short-indefinite=an,
	long-indefinite=an
}

\DeclareAcronym{BDD}{%
	short=BDD,
	long={Binary Decision Diagram},
	cite={{Akers1978,Boute1976}}
}

\DeclareAcronym{ROBDD}{%
	short=ROBDD,
	long={Reduced Ordered Binary Decision Diagram},
	cite={BRB1990},
	short-indefinite=an
}


\DeclareAcronym{SBDD}{%
	short=SBDD,
	long=Sequential Binary Decision Diagram,
	cite={TXD2011,XTD2012}
}

\DeclareAcronym{ZBDD}{%
	short=ZBDD,
	long=Zero-suppresed Binary Decision Diagram,
	cite={TD2004}
}

\DeclareAcronym{CSP}{%
	short=CSP,
	long=Communicating Sequential Processes,
	cite={Roscoe1997}
}

\DeclareAcronym{CSp}{%
	short=CSp,
	%long=\acsg*{DFT} cold spare gate
	%long=DFT's cold spare gate
	long={cold spare},
	class=gate,
	extra={Used in \acs*{DFT}.}
}

\DeclareAcronym{WSp}{%
	short=WSp,
	%long=\acsg*{DFT} warm spare gate
	%long=DFT's warm spare gate
	long={warm spare},
	class=gate,
	extra={Used in \acs*{DFT}.}
}

\DeclareAcronym{AND}{%
	short=AND,
	long={$\land$},
	class=gate,
	extra={Used in \acs*{SFT}, \acs*{TFT}, and \acs*{DFT}.}
}

\DeclareAcronym{OR}{%
	short=OR,
	long={$\lor$},
	class=gate,
	extra={Used in \acs*{SFT}, \acs*{TFT}, and \acs*{DFT}.}
}

\DeclareAcronym{PAND}{%
	short={PAND},
	long=priority-AND,
	class=gate,
	extra={Used in \acs*{SFT}, \acs*{TFT}, and \acs*{DFT}.}
}

\DeclareAcronym{SAND}{%
	short={SAND},
	long=simultaneous-AND,
	class=gate,
	extra={Used in \acs*{TFT}.}
}

\DeclareAcronym{POR}{%
	short=POR,
	long={priority-OR},
	class=gate,
	extra={Used in \acs*{TFT}.}
}

\DeclareAcronym{NOT}{%
	short=NOT,
	long={$\lnot$},
	class=gate,
	extra={Used in non-coherent trees.}
}

\DeclareAcronym{NIBefore}{%
	short=NIBefore,
	long={non-inclusive-before},
	class=gate,
	extra={Used in structure expressions of \acs*{DFT}.}
}

\DeclareAcronym{IBefore}{%
	short=IBefore,
	long={inclusive-before},
	class=gate,
	extra={Used in structure expressions of \acs*{DFT}.}
}

\DeclareAcronym{SIMLT}{%
	short=SIMLT,
	long={simultaneous},
	class=gate,
	extra={Used in structure expressions of \acs*{DFT}.}
}

\DeclareAcronym{XBefore}{%
	short=XBefore,
	long={exclusive-before},
	class=gate,
	extra={Proposed in this work.}
}

\DeclareAcronym{CSPm}{%
	short=CSP{$_M$},
	long=Communicating Sequential Processes,
	long-post={\footnote{This variant ``M'' is the machine-readable version of \acs*{CSP}.}}
	%long-post={\footnote{This variant ``M'' is the machine-readable version of CSP.}}
}

\DeclareAcronym{DFT}{%
	short=DFT,
	long=Dynamic Fault Tree,
	cite={DBB1992,Boyd1992}
}

\DeclareAcronym{FBA}{%
	short=FBA,
	long=Free Boolean Algebra,
	cite={[pp. 256-266]{GH2009}},
	short-indefinite=an
}

\DeclareAcronym{FDEP}{%
	short=FDEP,
	%long=\acsg*{DFT} functional dependency gate,
	%long=DFT's functional dependency gate,
	long={functional dependency},
	short-indefinite=an,
	class=gate,
	extra={Used in \acs*{DFT}.}
}

\DeclareAcronym{FDR}{%
	short=FDR,
	long=Failures and Divergences Refinement,
	short-indefinite=an
}

\DeclareAcronym{FT}{%
	short=FT,
	long=fault tree,
	%short-possessive=',
	short-indefinite=an
}

\DeclareAcronym{FTA}{%
	short=FTA,
	long=Fault Tree Analysis,
	short-indefinite=an,
	long-plural-form={Fault Tree Analyses}
}

\DeclareAcronym{Isar}{%
	short=Isar,
	long=Intelligible semi-automated reasoning
}

\DeclareAcronym{SEQ}{%
	short=SEQ,
	%long=\acsg*{DFT} sequence enforcing gate
	%long=DFT's sequence enforcing gate
	long={sequence enforcing},
	class=gate,
	extra={Used in \acs*{DFT}.}
}

\DeclareAcronym{SFT}{%
	short=SFT,
	long=Static Fault Tree,
	short-indefinite=an
}

\DeclareAcronym{TFT}{%
	short=TFT,
	long=Temporal Fault Tree,
	cite={WP2008,WP2009,Walker2009}
}

\DeclareAcronym{hiphops}{%
	short=HiP-HOPS,
	long=Hierarchically Performed Hazard Origin and Propagation Studies,
	cite={PMS+2001},
	long-post={\footnote{\url{http://www.hip-hops.eu/}}}
}

\DeclareAcronym{UTP}{%
	short=UTP,
	long=Unifying Theories of Programming,
	cite={HH1998}
}

\DeclareAcronym{HOL}{%
	short=HOL,
	long=higher-order logic,
}

\DeclareAcronym{AFP}{%
	short=AFP,
	long=archive of formal proofs,
	long-post={\footnote{Accessed 27/jan/2016: \url{http://afp.sourceforge.net/}}}
}
\DeclareAcronym{MCS}{%
	short=MCS,
	long=minimal cut set,
	short-possessive=',
	short-indefinite=an
}
\DeclareAcronym{MCSeq}{%
	short=MCSeq,
	long={minimal cut sequence},
	short-indefinite=an
}
\DeclareAcronym{DNF}{%
	short=DNF,
	long=disjunctive normal form
}
\DeclareAcronym{TTT}{%
	short=TTT,
	long=Temporal Truth Table
}

\DeclareAcronym{DTMC}{%
	short=DTMC,
	long={discrete-time Markov chain},
	alt={Markov chain},
	cite={Sericola2013,Ericson2005}
}
\DeclareAcronym{CTMC}{%
	short=CTMC,
	long={continuous-time Markov chain},
	cite={IP2014,Anderson2012}
}
\DeclareAcronym{BN}{%
	short=BN,
	long={Bayesian network},
	cite={Pearl1985}
}

\DeclareAcronym{SWN}{%
	short=SWN,
	long={stochastic well-formed net},
	cite={CDF+1993}
}

\DeclareAcronym{CPN}{%
	short=CPN,
	long={coloured \acl*{PN}},
	cite={Jensen1987}
}

\DeclareAcronym{PN}{%
	short=PN,
	long={Petri-net}
}

\DeclareAcronym{HLPN}{%
	short=HLPN,
	long={high-level \acl*{PN}},
	cite={Jensen1983}
}

\DeclareAcronym{HCAS}{%
	short=HCAS,
	long={cardiac assist system}
}

\DeclareAcronym{LTL}{%
	short=LTL,
	long={linear temporal logic}
}

\DeclareAcronym{SoS}{%
	short=SoS,
	long={System of Systems},
	long-plural-form={Systems of Systems},
	cite={Maier1998}
}

\DeclareAcronym{FMEA}{%
	short=FMEA,
	long={Failure Modes and Effects Analysis},
	cite={MMB2008}
}

\DeclareAcronym{DD}{%
	short=DD,
	long={Dependence Diagram},
	alt={RBD},
	%alt-long={Reliability Block Diagram},
	cite={[p. 198]{MKK2009}},
	long-post={\footnote{Also known as Reliability Block Diagram (\aca*{DD}). }}
	%long-post={\footnote{Also known as Reliability Block Diagram (RBD). }}
}

\DeclareAcronym{DRBD}{%
	short=DRBD,
	long={Dynamic Reliability Block Diagram},
	cite={DP2009}
}

\DeclareAcronym{SysML}{%
	short=SysML,
	long={Systems Modelling Language},
	cite={OMG2012}
}

\DeclareAcronym{UML}{%
	short=UML,
	long={Unified Modelling Language}
}

\DeclareAcronym{CML}{%
	short=CML,
	long={COMPASS Modelling Language},
	cite={BCW2014}
}

\DeclareAcronym{Z}{%
	short=Z,
	long={Z Notation},
	cite={Spivey1998}
}

\DeclareAcronym{ITL}{%
	short=ITL,
	long={Interval Temporal Logic},
	cite={Moszkowski1982}
}

\DeclareAcronym{FSM}{%
	short=FSM,
	long={Finite State Machine}
}

\DeclareAcronym{DT}{%
	short=DT,
	long={dependency tree},
	cite={WP2010}
}

\DeclareAcronym{DBN}{%
	short=DBN,
	long={dynamic bayesian network},
	cite={Murphy2002}
}

\DeclareAcronym{AADL}{%
	short=AADL,
	long={Architecture Analysis and Design Language},
	cite={FGH2006}
}


%\acresetall
%\end{acronym}
%\lstlistoflistings
\fi
\tableofcontents

\mainmatter

\chapter{Introduction}
\label{chap:intro}

The development process of critical control systems requires the rigorous execution of guides and regulations~\cite{ANAC2011,FAA1993,FAA2007,SAE1996b}.
Specialized agencies (like FAA, EASA and ANAC in the aviation field) use these guides and regulations to certify such systems.
Only upon certification such systems can be used in the real-world.

Safety is a property (measured both qualitative and quantitatively) of crucial concern on critical systems and it is the responsibility of the safety assessment process.
To employ such a process, dependable systems' taxonomy and safety assessment techniques must be well defined and understood.
Clarification of concepts of dependable systems can be surprisingly difficult when systems are complex, because the determination of possible causes or consequences of failures can be a very subtle process~\cite{ALR+2004}.

ARP-4761~\cite{SAE1996b} defines several techniques to perform safety assessment.
One of them is \ac{FTA}\index{Fault Tree Analysis}.
It is a deductive method that uses trees to model faults and their dependencies and propagation.
In such trees, the premises are the leaves (basic events) and the conclusions are the roots (top events).
Intermediary events use gates to combine basic events and each kind of gate has its own combination semantics definition.
\Acp{FT} that use only \ac{OR} and \ac{AND} gates are called \emph{coherent \aclp{FT}}~\cite{Andrews2001,AB2003,Oliv2006,CCR2008,Vaurio2016}\index{Fault Tree!coherent}.
They combine events as \emph{at least one shall occur} and \emph{all shall occur}, respectively.
To analyse \acp{FT}, their structures are abstracted as Boolean\index{Boolean Algebra} expressions called \emph{structure expressions}\index{structure expression}.
The analysis of coherent \acp{FT} uses a well-defined algorithm based on the Shannon's method\index{Binary Decision Diagrams!Shannon's method} to obtain \acp{MCS} from the structure expressions\index{structure expression}, and a general formula to calculate the probability of top events.
The \acp{MCS} are obtained by reducing structure expressions to a normal form, in which each term is a combination of variables (basic events) with conjunctive (\ac{AND}) gates, and the terms are combined by disjunctive (\ac{OR}) gates.
These minimal terms are also called \emph{prime implicants} or \emph{minterms}.
The Shannon's method\index{Binary Decision Diagrams!Shannon's method} originated a formalism to reduce structure expressions called \ac{BDD}\index{Binary Decision Diagrams}.
Another approach to reduce structure expressions is to use a mathematical model---called \acf{FBA}---that uses sets of sets to represent Boolean expressions.

Although structure expressions are formulas with logical operators, they are formalisms to enable automatic \ac{FTA}.
As shown in~\cite{Ericson2005}, \acp{FT} are a much richer model enabling a visual indication of fault paths, and include description of subsystems as intermediate events.
%% Este parágrafo foi incluído após discussão de Paulo sobre avaliação de confiabilidade/disponibilidade.

\Cref{fig:strategy-overview-traditional} shows how \ac{FTA} is traditionally performed.
It starts with an architectural model, then faults are identified and modelled in \iac{FT}.
System requirements are identified and are checked with \ac{FTA} results.
If the requirements are satisfied (accepted), the process ends and the modelled system may be implemented.
Otherwise, fault tolerance patterns are used, adding or modifying the original architecture to improve dependability.
The analyses are executed until system requirements are met.

\begin{figure}[htb]
  \centering
  \includegraphicsaspectratio[0.55]{StrategyOverview-traditional-path}
  \caption{Traditional \ac{FTA}}
  \label{fig:strategy-overview-traditional}
\end{figure}

Besides the traditional \ac{OR} and \ac{AND} gates, the \FThandbook defines other gates.
For example the \ac{PAND} gate, which considers the order of occurrence of events.
Although the \FThandbook defines new gates, there is no algorithm to perform the analysis of trees that contain such new gates.
This absence together with the need to analyse dynamic aspects of increasingly complex systems motivated the introduction of two new kinds of \aclp{FT}: \acp{DFT} and \acp{TFT}.
These variant trees can capture sequential dependencies of fault events in a system.
The difference from \ac{TFT} to \ac{DFT} is that \acp{TFT} use temporal gates directly, while \ac{DFT} does not---\acp{DFT} gates are an abstraction of temporal gates.
To differentiate the \aclp{FT} as defined in the \FThandbook from the other two, we will call the former as \acp{SFT}.

The work reported in~\cite{WP2009} aims at performing the full implementation of the \FThandbook, adding temporal gates to its \pandora methodology.
It was this implementation that introduced the new concept of \acp{TFT}, cited previously.
In such trees, events ordering is well-defined and an algebraic framework was proposed to reduce structure expressions\index{structure expression} to obtain \acp{MCSeq} and perform probabilistic analysis.
Reducing expressions is also desirable to check for tautologies, for example.

\Acp{DFT} introduce very different gates to capture dynamic configurations of systems.
The main gates are: \ac{CSp}, \ac{FDEP}, and \ac{SEQ}.
The semantics of the first is to add ``backup'' events, so the gate is active if the primary event and all spares are active.
The second adds basic events' dependency from a trigger event.
The third forces the occurrence of events in a particular order.
There is also \iac{WSp} gate that is slightly different from the \ac{CSp} gate.
They differ on the nature of sparing, whether fast (warm, always-on) or slow (cold, stand-by).
The readiness of the backup system in \iac{WSp} gate is higher than in \iac{CSp} gate.
The work reported in~\cite{MRL2011} shows an algebraic framework to compositionally reduce \ac{DFT} gates to order-based gates and perform probabilistic analysis of structure expressions\index{structure expression}. Thus, despite some limitations related to spare gates~\cite{MRL2014}, the structure expressions\index{structure expression} used in \acp{TFT} and \acp{DFT} can be formulated in terms of a generic order-based operator.

\begin{sloppypar}
The \ac{NOT} operator is absent in the algebras reported in~\cite{WP2009,Walker2009,Merle2010,MRL2011b} because,
if it is used without restrictions, it can be misleading, generating non-coherent analysis~\cite{Oliv2006}.
Although such an issue may arise, it can be essential in practical use as demonstrated in~\cite{Andrews2001} with algebraic laws to handle the operator in structure expressions.
%
Our concern is that the decision of the relevance of its use should be neither due to the choice of events-occurrence representation, as it is in~\cite{WP2009,Walker2009,Merle2010,MRL2011b}, nor with algebraic laws to include missing terms as it is in~\cite{Andrews2001}.
The algebra created in this work defines the \ac{NOT} operator and allows its use, as we show in \cref{chap:algebra}.
\end{sloppypar}

\begin{sloppypar}
\Ac{hiphops} is a set of methods and tools to analyse \acp{FT}.
The semi-automatic generation of \acp{FT} has architectural models and failure expressions as inputs.
\emph{The failure expressions are in fact structure expressions of components or subsystems.}
These expressions are annotated in components and subsystems and describe how they fail.
The tool combines these expressions with regard to the architecture of the system to generate \acp{FT}.
The work reported in~\cite{WP2008} shows a strategy to use the semi-automatic \ac{FT} generation of \ac{hiphops} with \pandora to generate structure expressions of \acp{TFT}.
\end{sloppypar}

\Ac{AADL} is a standard language to model (among other features) system structure and component interaction. 
\Ac{AADL} has several tools to perform different analyses to obtain \ac{SFT} to perform \ac{FTA}.
But \acpg{AADL} assertions framework does not express order explicitly as needed for \ac{TFT} and \ac{DFT} analyses.

In previous work~\cite{Didier2012,DM2012}, we proposed a systematic hardware-based faults identification strategy to obtain failure expressions as defined in \acifused{hiphops}{\acs{hiphops}}{\acscite{hiphops}} for \acp{SFT}.
%
We considered faults in components or subsystems to obtain structure expressions and use them as input for \ac{hiphops}.
If, instead, we obtain failure expressions of a whole system, they are in fact structure expressions\index{structure expression} of \iac{FT}.
%
Our previous strategy throws away the ordering information of the fault event sequences to generate failure expressions for components or subsystems for \acp{SFT}.
%
%Using our strategy as input for \ac{hiphops} we obtain a failure expression of a fault tree.
%
We focused on hardware faults because we assume that software does not fail as a function of time (wear, corrosion, etc).
%
We inherited this view from \embraer, which assumes that functional behaviour is completely analysed by functional verification~\cite{SP2011}.
%
We followed industry common practices using \simulink diagrams~\cite{Nise1992} as a starting point.
%
The work reported in~\cite{DM2012} was based on \ac{CSPm} to allow an automatic analysis using the model checker \acs{FDR}.
%
Thus, our strategy required the translation from \simulink to \ac{CSPm}~\cite{JMS+2011}.
%
It then runs \acs{FDR} to obtain several counter-examples (which are fault traces) ending in failures.
%
For two case studies provided by \embraer we showed that our automatically created failure expressions match with the engineer's provided ones or are better because consider additional fault occurrence combinations.

\section{Research questions}
\label{sec:research-questions}

Both \ac{TFT} and \ac{DFT} lack a first-order logic mathematical model like the one defined for \ac{SFT}.
For \acp{SFT}, mathematical models to reduce structure expressions are either based on set inclusion, with \ac{FBA}, or through tree search, with \ac{BDD}\index{Binary Decision Diagrams}.
One important concern on employing \ac{FTA} is whether \iac{FT} indeed represents a system behaviour.
The work reported in~\cite{MCS+1999} exposes this concern for \acp{DFT}, and the \ac{hiphops} framework---related to \acp{SFT} and \acp{TFT}---aims at getting this issue sorted out.
Our contribution to this issue for \ac{SFT} is shown in~\cite{DM2012,Didier2012}.

The mathematical model for \ac{TFT} has a discontinuity between two activation states:
\begin{alineasinline}
  \item non-occurrence, and
  \item occurrence some time later.
\end{alineasinline}
Such a discontinuity has some drawbacks as, for example, the impossibility to use \ac{NOT} gates, and handling the specific case of non-occurrence with zeros in \acp{TTT}.
The reduction of structure expressions in \ac{TFT} is based on a combination of:
\begin{alineasinline}
  \item algebraic reduction---which can unfortunately result in an infinite application of rules---,
  \item modularisation of independent subtrees (subtrees not always are independent), and
  \item \ac{DT}---which are limited to seven basic events, due to exponential growth.
\end{alineasinline}


\begin{sloppypar}
Most mathematical models~\cite{LHT2013,CSD2000,BRM+2005} for \ac{DFT} are based on the formalisation of \ac{DTMC} or \ac{CTMC} because \acp{DFT} were initially conceived to be a visual representation of such models.
As both \ac{DTMC} and \ac{CTMC} are state-based, they experience the state-space explosion problem.
The works reported in~\cite{BKK+2003,BHH+2003,SAE1996b} show techniques to overcome this problem, but the reduction can be infeasible because it depends on systems' models particularly, whether they are independent or not.
\end{sloppypar}

There are other approaches, however.
For instance, a modified version of \ac{BDD}\index{Binary Decision Diagrams} to tackle events ordering, called \acf{SBDD}\index{Sequential Binary Decision Diagrams}\index{Binary Decision Diagrams!Sequential}, to reduce structure expressions, and the work reported in \cite{BRM+2005}, which proposes a conversion of \ac{DFT} into \ac{DBN} to perform probabilistic analysis.

The approach to tackle events ordering with \ac{SBDD}\index{Sequential Binary Decision Diagrams}\index{Binary Decision Diagrams!Sequential}~\cite{XTD2012} has two kinds of nodes: terminals and non-terminals (terminals are nodes with basic events, and non-terminals are nodes with two events and an operator).
Although demonstrated in~\cite{Bryant1986} that these unconventional nodes (non-terminals) generate correct and efficient Boolean analysis, the analysis is still dependent on the order-related operators because the relation of terminals and non-terminals is not established directly (non-terminals are seen as an independent node in~\cite{XTD2012}).
For example, the occurrence of $A \rightarrow B$ is related to the occurrence of $A$ and $B$, but this relation is obtained in a further step, not in the \ac{SBDD}\index{Sequential Binary Decision Diagrams}\index{Binary Decision Diagrams!Sequential}.

The approach using the construction of \acp{DBN}~\cite{BRM+2005} is automatic and handles time slices as $t + \Delta t$, which implies a notion of events ordering as well.
As it is focused in probabilistic analysis, qualitative analysis is not directly supported.

The works reported in~\cite{Merle2010,XTD2012} show that \acg{DFT} operators can be converted into order-related operators, simplifying \ac{DFT} analysis.
Although the mathematical model presented in~\cite{Merle2010} establishes a denotational semantics for order-related operators, it lacks a formal method for expression reduction based on such a model.
It defines, instead, several algebraic laws to reduce expressions and an algorithm to minimize the structure function.

\begin{sloppypar}
\Ac{hiphops} proposes a hierarchical approach to model systems and perform \ac{FTA} (and \ac{FMEA}).
Although there is a tool to model and analyse systems using \ac{hiphops}, \acp{FT} construction is based on an algorithm.%, without proofs for soundness or completeness.
\end{sloppypar}

%The work reported in~\cite{AH2015} shows the formalisation of probabilistic analysis of \ac{SFT} and uses the same concept of date-of-occurrence shown in \adnote{cite Merle work}.

\begin{sloppypar}
Another concern, left untreated in the literature, is the undesirable possibility of non-determinism in structure expressions.
For example, if a commission is observed when fault A is active and an omission is observed when faults A and B are active, then the system may behave non-deterministically with a commission or omission when both A and B are active (A and B implies A).
\end{sloppypar}

From the exposed in this \lcnamecref{sec:research-questions}, our research questions are:
\begin{rqenum}[series=researchquestion]
  \item Is there a consistent mathematical model to analyse \acp{TFT} and \acp{DFT} that is set-based and similar to \ac{FBA}?\label{question:mathematical-model}
  \item What guarantees can we provide to detect non-determinism in erroneous behaviour?\label{question:non-determinism}
\end{rqenum}
%
Also, does such a model:
%
\begin{rqenum}[resume*=researchquestion]
  \item have the capacity of representing events ordering similar to \ac{TFT} and \ac{DFT}\label{question:ordering-representation}?
  \item represent systems behaviour by construction\label{question:gap}?
  \item allow both qualitative and quantitative analyses as supported by \ac{TFT} and \ac{DFT}\label{question:analyses}?
  %\item perform reduction of structure expressions to a normal form at least as efficiently as current approaches\label{question:efficiency}?
\end{rqenum}

%
%In this work we propose a theory that answers \cref{question:mathematical-model,question:ordering-representation,question:gap,question:analyses}. 
%\Cref{question:efficiency} is left as a future work.

\section{Proposed solution}

%In this work we present an algebra, called \acf{algebra}, defined  denota
In this work we present an algebra, called \acf{algebra}, to express ordering of fault events (\ac{TFT} and \ac{DFT}), enabling analysis of acceptance criteria of \acp{FT}.
The laws of \ac{algebra} are given in a denotational semantics based on sets of lists of distinct elements.
\Theac{algebra} aims at answering the \cref{question:mathematical-model,question:ordering-representation}.
The analysis of acceptance criteria is a decision problem and we use first-order logic and \isabellehol as verification tool.
Indeed, \ac{algebra} is part of a bigger strategy that relates fault injection on nominal models, fault modelling, \ac{FTA}, and fault tolerance patterns.

System and fault modelling is an essential step towards safety analysis.
Architectural modelling is the first step of the strategy and can be executed either in a graphical tool, or as requirements in natural language.
For example, our work reported in~\cite{APR+2013,ADP+2013} uses fault modelling in \ac{SysML} to verify fault tolerance of \acp{SoS}.

Writing and analysing expressions with order-related operators is more complex than analysing expressions with Boolean\index{Boolean} operators only.
We propose a logic, called \ac{activation}, which works together with an inner algebra to perform analysis of system structure and component interaction with a focus on fault modelling and fault propagation, tackling the complexity introduced by order-related operators.
\ac{activation} receives an algebra and the set of operational modes of a system as parameters.
The choice of algebra defines which structure expressions can be obtained: if Boolean algebra is passed as a parameter, the \ac{activation} can generate structure expressions with Boolean operators (\ac{SFT}); if the \ac{algebra} is passed as a parameter, the \ac{activation} can generate structure expressions with order-related operators (\ac{TFT} and \ac{DFT}).
The \ac{activation} requires that the inner algebras provide a set of properties (tautology and contradiction) and semantic values.
The use of the \ac{NOT} is essential: besides its use in expressions, we use the complement to normalise the expressions to provide \emph{healthy} expressions.

To obtain critical event expressions used in \acp{FT} and to denote faults propagation, the \ac{activation} provides a predicates notation and verification of non-determinism. 
We show three different approaches to check the non-determinism and answer \cref{question:non-determinism}: 
\begin{alineasinline}
  \item verify its existence, 
  \item indicate which set of operational modes are active for a combination of faults, or 
  \item what is the combination of faults that activates a set of operational modes.
\end{alineasinline}

In our proposed solution, depending on how much clear are the faults, the analyst may follow one of the paths: 
\begin{alineasinline}
  \item model the system in \simulink to allow fault injection and discovery, or 
  \item model faults using the \acl{activation}.
\end{alineasinline}
%
Both paths ends with structure expressions and the \ac{FTA} is performed using \ac{algebra}.

\Cref{fig:strategy-overview-csp} shows how to perform \ac{FTA} using faults injection.
The ``Faults injection'' block is obtained from part of our work reported in~\cite{DM2012,Didier2012}.
It starts with \simulink modelling, converts the model to \ac{CSPm} and then obtains fault event sequences.
The fault event sequences are then mapped to \theac{algebra}, which have a denotational semantics based on sets of lists.
This strategy aims at answering the \cref{question:gap}.

Safety requirements are stated in terms of critical failures as, for example, ``the probability of a complete failure of an airplane engine should be less than $10^{-9}$'' (quantitative), or ``a complete failure of the propulsion system shouldn't be caused by a single failure'' (qualitative).
In this work we call \emph{acceptance criteria} the verification of a safety requirement on \iac{FT}, where \acg{FT} top event is the undesired failure.
Positive requirements as, for example, ``the communications system should be operational $99.99\%$ of the cruise phase'' are treated as a complement (the complete failure should have a probability in less than $0.01\%$ of the cruise phase).
The acceptance criteria analysis aims at answering the \ref{question:analyses}.

From the model in \ac{algebra} (\cref{fig:strategy-overview-csp}), the acceptance criteria are then verified.
If the criteria are accepted, the process finishes.
Otherwise, the system is modified, and the process continues, modifying system's architecture, using fault tolerance patterns, improving system's dependability.
%
\begin{figure}[htb]
  \centering
  \includegraphicsaspectratio[0.8]{StrategyOverview-csp-path}
  \caption{Faults injection and \ac{algebra} to perform \ac{FTA}}
  \label{fig:strategy-overview-csp}
\end{figure}

\Cref{fig:strategy-overview-activation} shows a faults modelling strategy directly in the \ac{activation}.
The logic associates each operational mode with a faults expression.
After modelling all faults, the top events are extracted in a predicates notation. 
For example, ``does the behaviour of the system is the operational mode $X$?'', where $X$ can omission, commission, etc.
Given the flexibility of the \ac{activation} notation it is used to reason about basic fault events and top-event failures, which are related to~\ref{question:ordering-representation}.
Each predicate in \ac{activation} generates an expression in \theac{algebra}, which are reduced to obtain a normal form to obtain \acp{MCSeq} and to calculate top-events probability.
With the system modelled in \ac{activation}, the fault tolerance patterns can be applied directly on the model.
%
\begin{figure}[htb]
  \centering
  \includegraphicsaspectratio[0.8]{StrategyOverview-activation-path}
  \caption{\Ac{activation} and \ac{algebra} to perform \ac{FTA}}
  \label{fig:strategy-overview-activation}
\end{figure}

The complete proposed solution is summarized in \cref{fig:strategy-overview}, joining the paths described in \cref{fig:strategy-overview-csp} and \cref{fig:strategy-overview-activation} (paths A and B, respectively).
%
\begin{figure}[htb]
  \centering
  \includegraphicsaspectratio[1]{StrategyOverview}
  \caption{Strategy overview}
  \label{fig:strategy-overview}
\end{figure}

\section{Contributions}

The main contributions of this work are:

\begin{contrenum}[series=contributions]
  \item Define a denotational and algebraic model to express fault events order with \theac{algebra} (\cref{chap:algebra});
  \item Define a new operator to express order explicitly and proving that the resulting algebra---(\ac{algebra}) using this operator and Boolean\index{Boolean Algebra} operators---is a conservative extension of the Boolean algebra\index{Boolean Algebra} (also published in~\cite{DM2016})---see \cref{chap:algebra};
  \item From \simulink models, obtain fault event sequences and mapping to \ac{algebra} (\cref{chap:algebra});
  \item Reason about fault modelling in \ac{activation}, to obtain formal expressions of critical failures (top-event failures, \cref{chap:activation});
  \item Illustrate the application of the laws on a real case study, provided by \embraer (\cref{chap:case-study}), and on a literature case study.
  %\item Generalise laws in~\ac{algebra} in terms of abstract properties, similar to healthiness conditions in \theac{UTP}---to do.
  %\item Formally verify \acg{FT} (\ac{TFT} and \ac{DFT}) acceptance criteria (\cref{chap:acceptance});
\end{contrenum}

We use \isabellehol, theories in \isabellehol{'s} library, and a theory in the AFP library \cite{JM2005} to prove all theorems presented in this work.

The case studies cover the following scenarios, presented in \cref{chap:case-study}:
\begin{enumerate}
  \item From a model in \simulink, obtain the failure expression of a critical failure, analyse the ordering relation of fault events, and verify its acceptance criteria;
  \item Given a set of \ac{FT} structure expressions, verify which fault combinations analysis are missing;
  \item Perform a probabilistic analysis in a tree with explicit \ac{NOT} operator.
\end{enumerate}

\section{Thesis organization}

This thesis is organized as follows: in \cref{chap:basic-concepts,chap:analysis} we show the concepts and tools used as basis for this work.
\Cref{chap:algebra} presents \theac{algebra}, \cref{chap:activation} presents \theac{activation}, \cref{chap:case-study} the case study and the application of the proposed strategy, and we present our conclusions and future work in \cref{sec:conclusion}.
The contributions presented in \cref{chap:algebra} are summarized in terms of proved results.
To facilitate the understanding of the presented strategy, the effort to build laws and theirs (mechanized) proofs are shown in \cref{app:formal-proofs-isabelle-hol}.

\isabellehol{'s} theory files with all proofs are available at \algebraurl.

\chapter{Temporal Fault Trees}
\label{sec:tft}

\begin{quotation}[Título]{Autor}
Texto \\
Text
\end{quotation}

Compared to traditional \FTs, \TFTs can describe the order of occurrence of events. A specific order of occurrence causes a top event, thus the concept of failure expressions is augmented. Instead of using traditional Boolean gates, they use special gates named temporal gates (see \cref{tab:temporal-gates}). They differ on what kind of input they compare: Boolean gates compare Boolean values whilst temporal gates compare \emph{sequence values}. A sequence value is a natural number in which each basic event happens on a specific value, but there are no gaps on the sequence (see \cref{tab:temporal-gates-sv-formulas}). Zero values indicate that the event has not happened. As in Boolean logic, temporal logic also has truth tables, which are named \TTTs.

\TFTs can be written as an expression where each basic event in a tree is a variable in the expression. \Cref{tab:temporal-gates-ttt} shows a \TTT---a result of direct application of the equations shown in \cref{tab:temporal-gates-sv-formulas}---for each temporal gate basic formula, using events A and B. 

\begin{table}
\caption{Temporal gates}
\label{tab:temporal-gates}
\center
\begin{tabular}{|l|c|c|p{6cm}|}
\hline
\textbf{Name} & \textbf{Abbrev.} & \textbf{Operator} & \textbf{Description}\\
\hline
\hline
Or & OR & + & Similarly to the Boolean operator, whenever one of its inputs is higher than $0$, it outputs its value.\\
\hline
And & AND & . & When both inputs are higher than zero, it outputs the value of the maximum sequence value.\\
\hline
Priority Or& POR & | & When the first input is higher than zero, it outputs its value and the second output may or may not occur. It will not output a value when only the second output is higher than the first output.\\
\hline
Priority And & PAND & < & It outputs the value of the second output if the first output occurs strictly before the second.\\
\hline
Synchronous And & SAND & \& & It outputs the value of the sequence value when both inputs occur.\\
\hline
\end{tabular}
\end{table}

\begin{table}
\caption{Sequence value (SV) equations for each temporal gate}
\label{tab:temporal-gates-sv-formulas}
\center
\begin{tabular}{|c|l|}
\hline
\textbf{Gate} & \textbf{Sequence value formula} \\
\hline
\hline
OR & $\SV(\OR{A}{B}) = \left\{
\begin{array}{ll}
  \min(\SV(A), \SV(B)) &, \SV(A) > 0 \land \SV(B) > 0\\
  \max(\SV(A), \SV(B)) & \text{otherwise}
\end{array}
\right.$\\
\hline
AND & $\SV(\AND{A}{B}) = \left\{
\begin{array}{ll}
  \max(\SV(A), \SV(B)) &, \SV(A) > 0 \land \SV(B) > 0\\
  0 & \text{otherwise}
\end{array}
\right.$\\
\hline
POR & $\SV(\POR{A}{B}) = \left\{
\begin{array}{ll}
  \SV(A) &, \SV(B) = 0 \lor \SV(A) < \SV(B) \\
  0 & \text{otherwise}
\end{array}
\right.$\\
\hline
PAND & $\SV(\PAND{A}{B}) = \left\{
\begin{array}{ll}
  \SV(B) &, \SV(A) > 0 \land \SV(A) < \SV(B) \\
  0 & \text{otherwise}
\end{array}
\right.$\\
\hline
SAND & $\SV(\SAND{A}{B}) = \left\{
\begin{array}{ll}
  \SV(A) &, \SV(A) = \SV(B) \\
  0 & \text{otherwise}
\end{array}
\right.$\\
\hline
\end{tabular}
\end{table}

\begin{table}
\caption{\TTT for each temporal gate}
\label{tab:temporal-gates-ttt}
\center
\begin{tabular}{|c|c|c|c|c|c|c|}
\hline
\textbf{A} & \textbf{B} & {$\OR{A}{B}$} & $\AND{A}{B}$ & $\POR{A}{B}$ & $\PAND{A}{B}$ & $\SAND{A}{B}$ \\
\hline
\hline
0 & 0 & 0 & 0 & 0 & 0 & 0\\
0 & 1 & 1 & 0 & 0 & 0 & 0\\
1 & 0 & 1 & 0 & 1 & 0 & 0\\
1 & 1 & 1 & 1 & 1 & 0 & 1\\
1 & 2 & 1 & 2 & 1 & 2 & 0\\
2 & 1 & 1 & 2 & 0 & 0 & 0\\
\hline
\end{tabular}
\end{table}

\section{Formalisation of Temporal Fault Trees}
\label{sec:tft-formalisation}

In this section, I propose a formalisation in \CSP of \TFT to enable a comparison of \TFTs and the calculation of the probability of occurrence of top events.

The formalisation occurs in three steps: (i) formalising sequence values calculations to achieve \TTTs for a given temporal expression (\cref{sec:sv-calculus}), (ii) converting these \TTTs to sequences of events (\cref{sec:ttt-to-seqs}), and (iii) building a process to call these events to verify refinements (\cref{sec:seqs-to-process}).

\subsection{Sequence Value Calculus}
\label{sec:sv-calculus}

Given two sequence values as inputs, each temporal gate expression is encoded as a \CSP function as following:
\begin{align}
OR(a,b) & = 
  \begin{cases}
  min(a,b) & \text{if } a > 0 \land b > 0\\
  max(a,b) & \text{otherwise}
  \end{cases}\\
AND(a,b) & =
  \begin{cases}
  max(a,b) & \text{if } a > 0 \land b > 0\\
  0 & \text{otherwise}
  \end{cases}\\ 
POR(a,b) & =
  \begin{cases}
  a & \text{if } a > 0 \land (b = 0 \lor a < b)\\
  0 & \text{otherwise}
  \end{cases}\\ 
PAND(a,b) & =
  \begin{cases}
  b & \text{if } a > 0 \land a < b\\
  0 & \text{otherwise}
  \end{cases}\\ 
SAND(a,b) & =
  \begin{cases}
  a & \text{if } a = b\\
  0 & \text{otherwise}
  \end{cases}
\end{align}

Combining these functions, any temporal expression can be written. For example, given the expression $\PAND{\OR{A}{B}}{C}$, it can be written as: $PAND(OR(A,B),C)$. \Cref{tab:example-expression-ttt} shows the truth table for this example formula, by applying the function definitions for each row.

\begin{table}
\caption{\TTT for the expression $\PAND{\OR{A}{B}}{C}$}
\label{tab:example-expression-ttt}
\center
{\scriptsize
\begin{tabular}{|c|c|c|c|c|}
\hline
\textbf{A} & \textbf{B} & \textbf{C} & $\OR{A}{B}$ & $\PAND{\OR{A}{B}}{C}$ \\
\hline
\hline
0 & 0 & 0 & 0 & 0\\
0 & 0 & 1 & 0 & 0\\
0 & 1 & 0 & 1 & 0\\
0 & 1 & 1 & 1 & 0\\
0 & 1 & 2 & 1 & 2\\
0 & 2 & 1 & 2 & 0\\
1 & 0 & 0 & 1 & 0\\
1 & 0 & 1 & 1 & 0\\
1 & 0 & 2 & 1 & 2\\
1 & 1 & 0 & 1 & 0\\
1 & 1 & 1 & 1 & 0\\
1 & 1 & 2 & 1 & 2\\
1 & 2 & 0 & 1 & 0\\
1 & 2 & 1 & 1 & 0\\
1 & 2 & 2 & 1 & 2\\
1 & 2 & 3 & 1 & 3\\
1 & 3 & 2 & 1 & 2\\
2 & 0 & 1 & 2 & 0\\
2 & 1 & 0 & 1 & 0\\
2 & 1 & 1 & 1 & 0\\
2 & 1 & 2 & 1 & 2\\
2 & 1 & 3 & 1 & 3\\
2 & 2 & 1 & 2 & 0\\
2 & 3 & 1 & 2 & 0\\
3 & 1 & 2 & 1 & 2\\
3 & 2 & 1 & 2 & 0\\
\hline
\end{tabular}
}
\end{table} 

\subsection{From \TTTs to sequences of events}
\label{sec:ttt-to-seqs}

Obtaining (a set of) sequences of events is an intermediate step to get a process that represents a \TFT. This step is an optimisation to remove non-determinism, trimming the paths that lead to the top-level event. It combines paths that have the same initial event, building an hierarchical structure of choices.

The function $TTT$ below creates a set o tuples of size $n+1$, where $n$ is the number of basic events on the expression and the value on position $n+1$ is the result of the application of the temporal expression:
\begin{align}
TTT & :: TExp \Longrightarrow \powerset\left(\SVtuple[n+1]\right)\nonumber\\
TTT(expression) & = \left\{ (a_1, \ldots, a_n, expression(a_1, \ldots, a_n)) \vphantom{\clause}\nonumber\right.\\
& \qquad\left.\clause (a_1, \ldots, a_n) \in TTT_{inputs}(n)) \right\}
\end{align}
\noindent where $1,\ldots,n$ are the indexes of basic events, $\SV = \left\{0,\ldots,n\right\}$ are sequence values and $TTT_{inputs}$ defines a set of tuples of size $n$ that represents the events (inputs) for each row in a \TTT:
\begin{align}
TTT_{inputs} & :: I \Longrightarrow \powerset\left(\SVtuple\right)\\
TTT_{inputs}(n) & = \left\{ (a_1, \ldots, a_n) \vphantom{\clause}\right.\nonumber\\
 & \qquad\left. \clause max(a_1,\ldots,a_n) = card\left(\left\{a_1, \ldots, a_n\right\} \setminus \left\{0\right\}\right) \right\} \label{eq:ttt-inputs}
\end{align}

\noindent where $I = \left\{1,\ldots,n\right\}$. Note that the clause $max(a_1,\ldots,a_n) = card\left(\left\{a_1, \ldots, a_n\right\} \setminus \left\{0\right\}\right)$ guarantees that there are no gaps between two $a_i$'s, satisfying the \TFT property for sequence values.

For the example expression $\PAND{\OR{A}{B}}{C}$, the functions $TTT$ and $TTT_{inputs}$ return a set with cardinality $26$.

Finally, we create a data structure to avoid non-determinism and optimise the final process creation. This data structure is created in two steps: 
\begin{enumerate}
  \item Each tuple is converted into a set of pairs recursively where the first element is an available event and the second element is a set of options of events to choose. This set of options may contain others pairs recursively.
  \begin{align}
  SoP\left(sync_{ev}, \left(a_1,\ldots,a_n\right)\right) &= 
  \end{align}
  \noindent where $sync_{ev}$ is an event that indicates that the following events occur with the same sequence value.
  \item These pairs are then merged with respect to the first element. If there two pairs with the same first element, they are merged, making a union of the second element sets. 
\end{enumerate}


\subsection{Checking process refinements}
\label{sec:seqs-to-process}

Using the tuples defined by \cref{eq:ttt-inputs}, we now define a process that represents the execution of a temporal expression.

\chapter{Formalisation of Hazard Management}

Hazard properties, extracted from~\cite{EricsonII2005}:
%
\begin{enumerate}
  \item Hazardous element (HE);
  \item Initiating mechanisms (IM): sequence of events to activate the hazard;
  \item Target and Threat (T/T): the person or system and its corresponding threat;
  \item Status: open or closed. A hazard can only be closed if it has been verified through analysis, inspection or testing that the safety requirements are implemented in the design and successfully tested for effectiveness.
  \item Mitigation: lists the recommended actions obtained by all analysis
  \begin{itemize}
    \item Eliminate through design selection;
    \item Incorporate safety devices;
    \item Provide warning devices;
    \item Develop procedures and training;
    \item Control hazard through design methods.
  \end{itemize}
  \item Mishap Risk Index (initial, current, final) -- qualitative measurement:
  \begin{itemize}
    \item Severity
    \begin{enumerate}
      \item Catastrophic;
      \item Critical;
      \item Marginal;
      \item Negligible.
    \end{enumerate}
    \item Probability
    \begin{enumerate}
      \item Frequent;
      \item Probable;
      \item Occasional;
      \item Remote;
      \item Improbable.
    \end{enumerate}
  \end{itemize} 
\end{enumerate}

Scope of system to which Hazard Analysis make sense: those that have signal, material and energy.
%
All system functional elements~\cite[p. 47]{KSS+2011} are:
\begin{description}
  \item[Signal.] Generate, transmit, distribute, and receive signals used in passive or active sensing and in communications;
  \item[Data.]Analyse, interpret, organize, query, and/or convert data and information into forms desired by the user or other systems;
  \item[Material.] Provide system structural support or enclosure, or transform the shape, composition, or location of material substances;
  \item[Energy.] Provide and convert energy or propulsive power to the system.
\end{description}

\begin{definition}[Identifier]
\hmreg{Id}{\mathrm{Id}}
It is modelled as any user-defined data type. It can be a sequence of alphanumeric symbols (a descriptive text of the element) or a natural number (for example: an identification of a system element). It is a basic data type $\hmuse{Id}$ used in other definitions.
\end{definition}

\begin{definition}[Component]
\hmreg{Component}{\mathrm{C}}
\hmreg{subcomponents}{\mathop{\mathrm{subcomponents}}}
A component is any functional or physical block within a system that may contain a hazard. 
%
It is defined as an identifier and a set of identifiers: 
%
\\$\hmuse{Component}: \hmcartesian{\hmuse{Id}}{\hmpowerset{\hmuse{Id}}}$.
%
\\Every subcomponent cannot contain any ancestor.
%
Given a component $C=\left(c_0,CS_0\right)$, the subcomponents set is obtained by a folding operation.
%
\\$\hmsubcomponents{C} = \hmsetcomprehension{\hmunion{\hmsetenum{\hmfirst{C_i}}}{\hmsubcomponents{C_i}}}{C_i \in \hmsecond{C}}$.
%
\\For every component $C$: $\hmforall{C_i \in CS_0}{c_0 \notin \hmsubcomponents{C_i}}$ 
\end{definition}

\begin{definition}[Hazardous element]
A hazardous element represents an element that can potentially harm someone or something.
$\hmdatatypeHE: \hmdatatypeId$.
\end{definition}

%\hmregisterbinop{OPA}{\mathrm{OPA}}
%$\hmop{OPA}{X}{Y}$

\begin{definition}[Initiating mechanism]
An initiating mechanism is an event on the system.
%
$\hmdatatypeIM: \hmdatatypeId$
%it can be a CSP process. 
\end{definition}

\begin{definition}[Target/Threat]
A target and threat pair represents the person or system that is harmed if a hazardous element is activated by the initiating mechanisms and the nature of the harm. 
%
$\hmdatatypeTT: \hmcartesian{\hmdatatypeId}{\hmdatatypeId}$.
\end{definition}

\begin{definition}[Hazard]
Given a hazardous element $HE$, a sequence of $n$ initiating mechanisms $\hmseqenum{IM_1, \ldots, IM_n}$ and a target-threat pair $T/T$, a hazard $H$ is defined as: $H: \hmdatatypeId$ and is obtained by a total, bijective function $\hmhazardsymbol$: 
%
\\$\hmhazardsymbol: \hmtotalbijectivefunction{\hmcartesian{\hmcartesian{\hmdatatypeHE}{\hmdatatypeTT}}{\hmseq{\hmdatatypeIM}}}{\hmdatatypeId}$
%
\\which is written as:
%
\\$H = \hmhazard{HE}{\hmseqenum{IM_1, \ldots, IM_n}}{T/T}$.
\end{definition}

\begin{definition}[Probability function]
A probability function $\hmprobabilitysymbol$ is used to associate values to elements in the system.
%
It is defined through analysis. 
%
$\hmprobabilitysymbol: \hmpartialfunction{\hmdatatypeId}{\hmdatatypeR}$
\end{definition}

\begin{definition}[Hazard analysis type]
A hazard analysis type is a categorization of the analysis methods as follows:
%
\newcommand{\HATdescr}[1]{\item[\expandafter\csname #1presentation\endcsname.] \expandafter\csname #1expanded\endcsname\csname #1CHECKtrue\endcsname}
%
\begin{description}
  \HATdescr{CDHAT}
  \HATdescr{PDHAT}
  \HATdescr{DDHAT}
  \HATdescr{SDHAT}
  \HATdescr{ODHAT}
  \HATdescr{HDHAT}
  \HATdescr{RDHAT} 
\end{description}
\end{definition}

\begin{definition}[Analysis method]
An analysis method $\hmdatatypeAM$, such that $\hmdatatypeAM: \hmdatatypeId$, describes a method used to analyse an element. 
%
It is defined as an enumerated set with the value correspondence shown in \cref{tbl:analysis-methods}.
%
\newcounter{analysismethods}
\newcommand{\amcount}{\addtocounter{analysismethods}{1}\arabic{analysismethods}}
\newcommand{\amline}[3]{\amcount & \csname #1presentation\endcsname#3 & #2 & \csname #1expanded\endcsname}
%
\begin{table}
\centering
\begin{tabular}{c|l|l|l}
Seq. & Abbrev. & Hazard analysis type & Description\\
\hline
\hline
\amline{PHL}{\CDHAT}{*}\\
\hline
\amline{PHA}{\PDHAT}{*}\\
\hline
\amline{SSHA}{\DDHAT}{*}\\
\hline
\amline{SHA}{\SDHAT}{*}\\
\hline
\amline{OSHA}{\ODHAT}{*}\\
\hline
\amline{HHA}{\HDHAT}{*}\\
\hline
\amline{SRCA}{\RDHAT}{*}\\
\hline
\amline{FTA}{\SDHAT, \DDHAT}{}\\
\hline
\amline{ETA}{\SDHAT}{}\\
\hline
\amline{FMEA}{\DDHAT}{}\\
\hline
\amline{FaHA}{\DDHAT}{}\\
\hline
\amline{FuHA}{\SDHAT, \DDHAT}{}\\
\hline
\amline{SCA}{\SDHAT, \DDHAT}{}\\
\hline
\amline{PNA}{\SDHAT, \DDHAT}{}\\
\hline
\amline{MA}{\SDHAT, \DDHAT}{}\\
\hline
\amline{BA}{\SDHAT}{}\\
\hline
\amline{BPA}{\DDHAT}{}\\
\hline
\amline{HAZOP}{\SDHAT, \DDHAT}{}\\
\hline
\amline{CCA}{\SDHAT, \DDHAT}{}\\
\hline
\amline{CCFA}{\SDHAT, \DDHAT}{}\\
\hline
\amline{MORT}{\SDHAT, \DDHAT}{}\\
\hline
\amline{SWSCA}{}{}\\
\hline
\amline{SWHA}{}{}\\
\hline
\amline{THA}{}{}\\
\hline
\hline 
\end{tabular}
\caption{Analysis methods}
\label{tbl:analysis-methods}
\end{table}
\end{definition}

\begin{definition}[Time]
Time $\hmdatatypetime$ is abstracted with equal and before relations. 
%
It is defined as a sequence of time tags.
%
$\hmdatatypetime: \hmseq{\hmdatatypetimetag}$.
\end{definition}

\begin{definition}[Time tag]
A time tag represents an instantaneous observation of time.
%
It is defined as an identification: $\hmdatatypetimetag: \hmdatatypeId$
\end{definition}

\begin{definition}[Analysis]
Given an analysis method $M$, such that $M: \hmdatatypeAM$, a time tag $t$, such that $t:\hmdatatypetimetag$, an element $E$, such that $E:\hmdatatypeId$, and a probability $p$, such that $p:\hmdatatypeR$, an analysis $A$ is defined as:
%
\\$A: \hmcartesian{\left(\hmcartesian{\hmdatatypeId}{\hmdatatypetimetag}\right)}{\left(\hmcartesian{\hmdatatypeAM}{\hmdatatypeR}\right)}$
%
\\$A = \left(\left(E, t\right), \left(M, p\right) \right)$

\end{definition}
The probability function is then updated accordingly to an analysis. For example, for the analysis above, the probability function $P$ can be updated as:
$\hmprobabilitysymbol = \hmoverride{\hmprobabilitysymbol}{\left\{\hmmapsto{E}{p}\right\}}$.

\begin{definition}[Hazard activation probability]
Given a probability function $\hmprobabilitysymbol$, such that $P: \hmpartialfunction {\hmdatatypeId}{\hmdatatypeR}$, a hazardous element $HE$, a sequence of $n$ initiating mechanisms $\hmseqenum{IM_1,\ldots,IM_n}$ and a target-threat pair $T/T$, the probability of the activation of the hazard $H = \hmhazard{HE}{\hmseqenum{IM_1,\ldots,IM_n}}{T/T}$ is:
%
\\$\hmprobability{H} = \hmtimes{\hmprobability{HE}}{\hmtimes{\hmtimes{\hmprobability{IM_1}}{\ldots}}{\hmprobability{IM_n}}}$
\end{definition}

\begin{definition}[Initiating mechanism]
An initiating mechanism is a relation between components attributes and their unwanted behaviour or state. 
%
Some component attribute examples~\cite{EricsonII2005}:
\begin{description}
  \item[Hardware] Failure modes, hazardous energy sources, $IM_{H1}=\left(C_1,1\right)$, $IM_{H2}=\left(C_1,2\right)$;
  \item[Software] Design errors, design incompatibilities, $IM_{S1}=\left(C_2,3\right)$, $IM_{S2}=\left(C_2,4\right)$
  \item[Personnel] Human error, human injury, human control interface, $IM_{P1}=\left(C_3,3\right)$;
  \item[Environment] Weather, external equipment;
  \item[Procedures] Instructions, tasks, warning notes;
  \item[Interfaces] Erroneous input/output, unexpected complexities;
  \item[Functions] Fail to perform, performs erroneously;
  \item[Facilities] Building faults, storage compatibility, transportation faults;
\end{description}
\end{definition}

By the relations between components and hazards, and hazards and initiating mechanisms, a dependency graph can be built, so if a components changes, the corresponding hazards changes as well.

\begin{definition}[Mishap]
A mishap is defined as a pair of the probability of a hazard and the severity in case it happens: $M: \hmcartesian{\hmdatatypeR}{\hmdatatypeseverity}$.
\end{definition}
%
\noindent Where $\hmdatatypeseverity$ is a finite set, such that $\hmdatatypeseverity \subset \hmdatatypeN$. It contains the following values:
\begin{description}
  \item[1] $=$ Catastrophic;
  \item[2] $=$ Critical;
  \item[3] $=$ Marginal and
  \item[4] $=$ Negligible.
\end{description}

\begin{definition}[Hazard Model]
\hmreg{hazardmodel}{\mathrm{HM}}
A $\hmuse{hazardmodel}$\ldots
\end{definition}

\section{Refinement}

In this section we define a refinement relation on a hazard model of a system.

\TODO{Add definitions on the hazard model
\begin{itemize}
  \item Set of analyses;
  \item Probability function;
  \item Addition of analysis $\implies$ updates probability function;
  \item Set of hazards;
  \item Dependency graph (connection between components);
\end{itemize}
}

There a few situations in which a refinement is desirable:
\begin{description}
  \item[Hazard analysis.] A hazard model is said to be an analysis of another model if at least one new analysis was added.
  \item[Hazard mitigation.] A hazard model is said to be a mitigation of another if at least one hazard is mitigated by decrement of probability value. If $\hmprobabilitysymbol$ is the probability function of the previous model and $\hmprobabilitysymbol'$ is the probability function of the refined model, then: 
%
\\$\hmdom{\hmprobabilitysymbol} \subseteq \hmdom{\hmprobabilitysymbol'} \land%
\hmexists{e \in \hminter{\hmseqenum{set of hazards}}{\hmdom{\hmprobabilitysymbol}}}{%
\hmprobability[\hmprobabilitysymbol']{e} < \hmprobability{e}%
}$.
  \item[Hazard discovery.] A hazard model is said to be a discovery of another if the hazards set is augmented with a new discovered hazard.
  \item[Component replacement.] A valid component replacement in a model is achieved if all analyses for the replaced component are updated with the new component and the probability of activation is less than or equal to the previous ones.
  \item[Component addition.] A valid component addition in a model is achieved if the component is not in any other component hierarchy or, if it is, then the analyses of all affected component hierarchy and related components are updated and the probability of activation is less than or equal to the previous ones.
  \item[Component removal.] A valid component removal in a model is achieved if the analyses of all affected component hierarchy and related components are updated and the probability of activation is less than or equal to the previous ones. 
\end{description}



\chapter{Related work}

\begin{quotation}[Título]{Autor}
Texto \\
Text
\end{quotation}

\section{Title}

\subsection{Subtitle}

Plain text.

\subsection{Another subtitle}

More plain text.


%\bibliographystyle{natbib}
%\addcontentsline{toc}{chapter}{\bibliographytocname}
\begin{references}
\bibliography{references}
\end{references}

% Appendix
%\clearpage
%\addappheadtotoc
%\appendix
%\appendixpage
\theappendix
%\include{appendix/experiment-instruments}

\end{document}
