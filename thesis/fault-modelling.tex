\chapter{Modelling faults}
\label{sec:fault-modelling}

\begin{quotation}[Título]{Autor}
Texto \\
Text
\end{quotation}

In this section we show how to model faults using functions in a component-level failure logic.
%
We define two relations to establish our axioms: (i) failure value and failure state, (ii) order of occurrence.
%
Using these definitions we are able to create a library of basic components that can be used as building blocks to create new components or to compose a system model.

\section{Modelling a system}
\label{sec:modelling-a-system}

A system model is a block diagram (nodes) with directed connected edges. 
%
An edge's origin is the output of some component and the destination is one or more inputs of other component.
%
Two output ports cannot be directly connected, nor does two inputs.
%
It is required an output and at least one input to connect two components.

Edges and nodes contain fault-related information.
%
An edge contains the current value of the output of the originating node.
%
It is a Boolean indicating a nominal or an erroneous value. 
%
In the last case, the edge also contains the error mode.
%
A node contains: (i) state information, indicating whether the component is in failure, and (ii) a logic for each output that expresses how the output reacts to internal failures and failures on the inputs. 
%
Item (ii) is very similar to the cause of output deviation in \HIPHOPS[TODO: citar def. HH].
%
It also includes the order of occurrence of the failures.

Differently from \HIPHOPS, an output expression also contains the nominal case to cover all possibilities on the components, to avoid missing a case.

\begin{definition}[Component]
$C = \left(I, \left( O_1\left(I,F\right),\ldots,O_{n_o}\left(I,F\right) \right) \right)$

\noindent where $I = \left(i_1,\ldots,i_{n_i}\right)$ are inputs of type $FailureMode$, $O_k$ are functions that describe outputs, $F$ indicates an internal failure of the component, and $n_i$ and $n_o$ are the number of inputs and outputs.

The type $FailureMode$ contains enumerated values of failure modes, but contains at least a value $N$ to express a nominal value.
\end{definition}

\begin{definition}[System model]
$S = \left(Cs, A\right)$
\noindent where $Cs$ is a set of components and $A$ is a function $A : (Component \cross \nat) \cross (Component \cross \nat) \rightarrow Boolean$\footnote{Or it is a relation of type $(Component \cross \nat) \cross (Component \cross \nat)$.} that relates a component output to another component's input.
\end{definition}

%%%%%%%%%%%%%%%%%%%%%%%%%%%%%%%%%%%%%%%%%%%%%%%%%%%%%%%%%%%%%%%%%%%%%%%%%%%%%%%%
% Failure value and failure state%%%%%%%%%%%%%%%%%%%%%%%%%%%%%%%%%%%%%%%%%%%%%%%
%%%%%%%%%%%%%%%%%%%%%%%%%%%%%%%%%%%%%%%%%%%%%%%%%%%%%%%%%%%%%%%%%%%%%%%%%%%%%%%%
\subsection{Failure value and failure state}

%A function $F_i : Input \rightarrow Boolean$ expresses whether the current value on an input is a failure value. 
%
%Similarly, a function $F_c : Component \rightarrow Boolean$ expresses whether the current state of a component is a failure state.
%
%The type of a failure value is defined by a unbounded value.

To illustrate these definitions, suppose a component $C$ that contains only one output, and its failure logic depends only on its internal state: if it is faulty, it produces an omission value $O_M$, otherwise it produces a nominal value $N$. 
%
The failure logic for its output $O$ is: $O = F \land O_M \lor \lnot F \land N$.

TODO: all Boolean operators relate to $FailureMode$ as: if the Boolean value is $true$, then it results in the $FailureMode$ value, otherwise it results in $false$. It creates an invariant that all output functions are tautologies.

\subsection{Order of occurrence}

The order of occurrence of events is defined as a sequence value as it is in \HIPHOPS: it is a natural value assigned to each input.
%
The value does not contain gaps, so always there is a pair of input that the modulus of the difference of their sequence values is 1, or the modulus of the difference is 0 for all pairs of inputs.
%
For example, if we have two variables, they can be assigned values 0 and 1 or 1 and 2, but not 0 and 2. 

We define a relation $S : Input \rightarrow \nat $ as the sequence value for the given input.