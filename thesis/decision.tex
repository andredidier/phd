
\begin{table}
\begin{center}
\begin{tabular}{l|p{6.5cm}|p{6.5cm}}
Item & Hazard & Dynamic FT\\
\hline
\hline
Objetivo 
  & Realizar a evolução do gerenciamento de Hazard de maneira controlada através de métodos formais. 
  & Analisar o sistema qualitativamente, com relações de causa e consequência (bottom-up), e quantitativamente com cálculo da probabilidade do evento raiz.\\
\hline
Análises 
  & Dado um sistema com Hazards, probabilidades calculadas e mudanças a serem feitas, verificar o quanto as mudanças impactam o sistema. Uma mudança não impacta se a probabilidade de falha do sistema não é alterada após as mudanças. As probabilidades de falhas das partes do sistema são calculadas da maneira convencional, usando uma das técnicas existentes. 
  & Busca dos conjuntos de eventos mínimos que causam o evento raiz e aderência da árvore ao sistema ou a seu modelo. \\
\hline
Tese 
  & Existem ao menos uma modelagem do sistema e uma relação de refinamento que garante a evolução do gerencimaneto. 
  & A partir do modelo nominal arquitetural do sistema é possível injetar falhas a fim de obter lógica de falhas (equivalentes a árvores de falhas), garantindo a aderência de uma árvore de falha ao modelo do sistema. OU A partir de um modelo nominal arquitetural, é possível injetar falhas a fim de obter a probabilidade de falhas do sistema.\\
\hline
\end{tabular}
\end{center}
\caption{Opções de tese}
\label{tbl:thesis-decision}
\end{table}

Opção escolha a partir da tabela \cref{tbl:thesis-decision}: Dynamic FT.

Solução:
\begin{enumerate}
  \item A escolha da arquitetura define combinações de falha. Com as probabilidade individuais de cada falha, é possível calcular a probabilidade. 
  \item Como resolver o problema de ordem de falha? Ex.: comissão de sinal. Se em uma replicação ambos sofrem comissão de sinal ao mesmo tempo, consiste-se em uma falha. Se um sofre comissão e em um momento seguinte, o outro, então não se configura uma falha.
  \item É preciso indicar como cada um dos componentes responde aos tipos de falhas:
  \begin{itemize}
    \item Omissão, comissão, degradação, EMI/P (\emph{Eletromagnetic Interference or Pulse}).
    \item O tipo de falha é definido pelo designer. A cada tipo adicionado, é preciso revisar os itens já definidos.
  \end{itemize}

\end{enumerate}